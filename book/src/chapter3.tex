% ALGORITHMS FOR MODULAR ELLIPTIC CURVES       Last change: 10/96
%
% CHAPTER 3  ELLIPTIC CURVE ALGORITHMS
%
% Use AmSTeX 2.0
%
\input book.def
\advance\pageno by\chaponepages
\advance\pageno by\chaptwopages
\advance\pageno by\appendixpages
%
\topmatter
%
\title\chapter{3} Elliptic curve algorithms \endtitle
%
\endtopmatter
%
\document
%\openup 2pt 
%\raggedbottom % for drafts only
%
\def\chapno{3}
\newsec{\Notation}   % 3.1 Terminology and notation
\newsec{\TateAlg}    % 3.2 The Kraus--Laska--Connell algorithm and 
                     %     Tate's algorithm
\newsec{\Torsion}    % 3.3 The Mordell--Weil group I: finding torsion points
\newsec{\Heights}    % 3.4 Heights and the height pairing
\newsec{\Findinf}    % 3.5 The Mordell--Weil group II: generators
\newsec{\Rank}       % 3.6 The Mordell--Weil group III: the rank
\newsec{\Periods}    % 3.7 The period lattice
\newsec{\Isogenies}  % 3.8 Finding isogenous curves
\newsec{\TwistsEtc}  % 3.9 Twists and complex multiplication

%
% CHAPTER 3 SECTION 1
%
\beginsection{\Notation}
\head\Notation\ Terminology and notation \endhead

For reference in the following sections, we collect here the notation,
terminology and formulae concerning elliptic curves which we will use
throughout this chapter.

An elliptic curve $E$ defined over $\Q$ has an equation or \und{model} 
of the form \neweq\curveeq
$$
  E\colon\qquad   y^2+a_1xy+a_3y = x^3+a_2x^2+a_4x+a_6  \tag \curveeq
$$
where the coefficients $a_i\in\Q$.   We call such an equation a
\und{\W\ equation} for $E$, and denote this model by 
$[a_1,a_2,a_3,a_4,a_6]$.  We say that (\curveeq) is \und{integral} or
\und{defined over~$\Z$} if all the $a_i$ are in~$\Z$.  From these
coefficients we derive the auxiliary quantities
$$\align
                       b_2&=a_1^2+4a_2,                     \\
                       b_4&=a_1a_3+2a_4,                    \\
                       b_6&=a_3^2+4a_6,                     \\
                       b_8&=a_1^2a_6-a_1a_3a_4+4a_2a_6+a_2a_3^2-a_4^2, \\
\xtext{the \und{invariants}}
                       c_4&=b_2^2-24b_4,                    \\
                       c_6&=-b_2^3+36b_2b_4-216b_6,         \\
\xtext{the \und{discriminant}}
                      \Delta&=-b_2^2b_8-8b_4^3-27b_6^2+9b_2b_4b_6,   \\
\xtext{and the \und{$j$-invariant}}
                       j&=c_4^3/\Delta,                     \\
  \endalign
$$  
which are related by the identities
$$
   4b_8=b_2b_6-b_4^2 \qquad\text{and}\qquad 1728\Delta=c_4^3-c_6^2.
$$
The discriminant $\Delta$ must be non-zero for the curve defined by
equation (\curveeq) to be non-singular and hence an elliptic curve.
The $j$-invariant is (as its name suggests) invariant under
isomorphism; elliptic curves with the same $j$ are called
\und{twists}: they are isomorphic over an algebraic extension, but not
necessarily over $\Q$.  The invariants $c_4$ and $c_6$ are sufficient
to determine $E$ up to isomorphism (over $\Q$) since $E$ is isomorphic
to
$$
   Y^2 = X^3 - 27c_4X - 54c_6.
$$

The most general isomorphism from $E$ to a second curve $E'$ given by an 
equation of the form (\curveeq), which we usually think of as a change of 
coordinates on $E$ itself, is $T(r,s,t,u)$, given by \neweq{\transformxy}
$$
    \aligned
                x &= u^2x' + r          \\
                y &= u^3y' + su^2x' + t \\        
    \endaligned                         \tag\transformxy 
$$
where $r,s,t\in\Q$ and $u\in\Q^*$.  The effect of $T(r,s,t,u)$ on the
coefficients $a_i$ is given by \neweq{\transabc}
$$
  \aligned
        u  a_1' &= a_1+2s                               \\
        u^2a_2' &= a_2-sa_1+3r-s^2                      \\
        u^3a_3' &= a_3+ra_1+2t                          \\
        u^4a_4' &= a_4-sa_3+2ra_2-(t+rs)a_1+3r^2-2st    \\
        u^6a_6' &= a_6+ra_4+r^2a_2+r^3-ta_3-t^2-rta_1   \\
%\noalign{\smallskip}
%        u^6b_6' &= b_6+2rb_4+r^2b_2+4r^3                \\
%        u^4b_4' &= b_4+rb_2+6r^2                        \\
%        u^2b_2' &= b_2+12r                              \\
%\noalign{\smallskip}
%        u^4c_4' &= c_4                                  \\
%        u^6c_6' &= c_6                                  \\
%        u^{12}\Delta' &= \Delta                         \\
%        j' &= j.                                        \\
  \endaligned                                            \tag\transabc  
$$ 
so that 
$$ 
   u^4c_4' = c_4, \qquad 
   u^6c_6' = c_6, \qquad 
   u^{12}\Delta' = \Delta \quad\text{and}\quad 
   j' = j.  
$$ 
The transformations $T(0,0,0,u)$ we will refer to as scaling
transformations; these have the effect of dividing each coefficient
$a_i$ by $u^i$, and similarly for each of the other quantities,
according to its weight.  Here $a_i$, $b_i$ and $c_i$ have weight~$i$,
while $\Delta$ has weight~12 and $j$ has weight~0.  By applying
$T(0,0,0,u)$ for suitable $u$ we can always transform to an integral
model; all the invariants are then integral, except (possibly) for
$j$.  Among such integral models, those for which the positive integer
$|\Delta|$ is minimal are called \und{global minimal models} for $E$.
We will give in the next section a simple algorithm for finding such a
model, given the invariants $c_4$ and $c_6$ of any model.  Clearly,
isomorphisms between minimal models must have $u=\pm1$ and
$r,s,t\in\Z$.  We may normalize so that $a_1, a_3\in\{0,1\}$ and
$a_2\in\{-1,0,1\}$, by suitable choice of $s$, $r$ and $t$ (in that
order), as may be seen from (\transabc).  Such an equation will be
called \und{reduced}, and it is not hard to show that it is unique:
the only transformation other than the identity $T(0,0,0,1)$ from a
reduced model to any another reduced model is the transformation
$T(0,-a_1,-a_3,-1)$, which takes any model to itself; this is just the
negation map $(x,y)\mapsto (x,y-a_1x-a_3)$ from the curve to itself.
Thus every elliptic curve $E$ defined over $\Q$ has a \und{unique}
reduced minimal model.  This fact makes it very easy to recognize
curves: in Table~1 we give the coefficients of such a model for each
of the curves there.

Given integers $c_4$ and $c_6$, two questions arise: is there a curve
over $\Q$ with these invariants, and is it minimal?  Clearly we must
have $c_4^3-c_6^2=1728\Delta$ with $\Delta\not=0$.  A solution to the
first problem is given by Kraus in Proposition~2 of \cite\Kraus, which
states the following.  \newprop{\conditions}

\proclaim{Proposition \conditions} Let $c_4$, $c_6$ be integers such that
$\Delta=(c_4^3-c_6^2)/1728$ is a non-zero integer.  In order for there
to exist an elliptic curve $E$ with a model {\rm(\curveeq)} defined over
$\Z$ having invariants $c_4$ and $c_6$, it is necessary and sufficient
that
\part{1}$c_6\not\equiv\pm9\pmod{27}$; 
\part{2}either $c_6\equiv-1\pmod4$, or $c_4\equiv0\pmod{16}$ and 
$c_6\equiv0,8\pmod{32}$.
\endproclaim

The conditions of Proposition~\conditions\ will be referred to as
\und{Kraus's conditions}.  If we are given integers $c_4$ and $c_6$
satisfying these conditions, we can recover the coefficients $a_i$ of
the reduced model of the curve with $c_4$ and~$c_6$ as invariants,
using the formulae already given in Chapter~2, Section~14, which we
repeat here for convenience:
$$\align
  b_2 &= -c_6 \mod 12 \in \{-5,\ldots,6\};\\
  b_4 &= (b_2^2-c_4)/24;\\
  b_6 &= (-b_2^3+36b_2b_4-c_6)/216;\\
\noalign{\smallskip}
  a_1 &= b_2 \mod2\in\{0,1\};\\
  a_3 &= b_6 \mod2\in\{0,1\};\\
  a_2 &= (b_2-a_1)/4;\\
  a_4 &= (b_4-a_1a_3)/2;\\
  a_6 &= (b_6-a_3)/4.\\
  \endalign
$$
To see this, we may assume that we are seeking coefficients of a
reduced model;  then $b_2\in\{-4,-3,0,1,4,5\}$, and we have
$-c_6\equiv b_2^3\equiv b_2\pmod{12}$.  The rest is easy; provided
that $c_4$ and~$c_6$ satisfy Kraus's conditions, all the divisions
will be exact.

In the following section we answer the second question by giving an
algorithm for computing the reduced coefficients of a \und{minimal}
model for any curve $E$, given either integral invariants satisfying
Kraus's conditions, or any integral model for $E$.  We simply determine
the maximal integer $u$ such that $c_4'=c_4/u^4$ and $c_6'=c_6/u^6$
satisfy Kraus's conditions, and then compute the reduced coefficients
$a_i'$ from these.  As with many questions concerning elliptic curves,
most of the work goes into determining the powers of 2 and 3 which
divide $u$.

We will assume without further discussion that on any given curve $E$,
points may be added and multiples taken, using standard formulae.  The
Mordell--Weil group of all rational points on $E$ will be denoted
$E(\Q)$ as usual.  If $n$ is a positive integer, we denote by
$E(\Q)[n]$ the subgroup of rational points of order dividing $n$,
which is the kernel of the multiplication map from $E$ to itself.

%
% CHAPTER 3 SECTION 2
%
\beginsection{\TateAlg}
\head\TateAlg\ The Kraus--Laska--Connell algorithm and Tate's algorithm 
\endhead
 
In this section we give two algorithms. The first was originally given
by Laska in \cite{\Laska}, and finds a minimal model for a curve $E$,
starting from an integral equation.  Essentially the algorithm was to
test all positive integers $u$ such that $u^{-4}c_4$ and $u^{-6}c_6$
are integral, to see if they are the invariants of a curve defined
over $\Z$.  Using Kraus's conditions (see Proposition~\conditions\
above), this procedure can be simplified, since it is possible to
compute in advance the exponent $d_p$ of each prime $p$ in the minimal
discriminant, and hence compute $u$ at the start.  The usual formulae
then give the coefficients $a_i$ of the reduced model.  Our
formulation of the resulting algorithm over $\Z$ is similar to that
given in \cite\Connell, where more general rings are considered: in
particular an explicit algorithm is given there for finding local
minimal models over arbitrary number fields, and hence global minimal
models where they exist.  Over $\Z$, the algorithm is extremely
simple.

In the pseudocode below,
%\footnote{The syntax for our pseudocode is essentially that of
%Algol68, and should be self-explanatory.} 

{\tt ord(p,n)} gives the power of the prime {\tt p} which divides the 
non-zero integer {\tt n}; 

{\tt floor(x)} gives the integral part of the real number {\tt x};

{\tt a mod p} gives the residue of {\tt a} modulo {\tt p} lying in the
range $-\frac12p<a\le\frac12p$; in particular, when $p=2$ or~3 this
gives a residue in $\{0,1\}$ or $\{-1,0,1\}$ respectively.  Also {\tt
inv(a,p)} gives the inverse of {\tt a} modulo {\tt p}, assuming that
{\tt gcd(a,p)=1}.

\beginalg{The Laska--Kraus--Connell Algorithm}

\+INPUT:   &&&c4, c6 (integer invariants of an elliptic curve E).\cr
\+OUTPUT:  &&&a1, a2, a3, a4, a6 (coefficients of a reduced minimal model for E).\cr
%
\smallskip \lineno=0
%
\nline BEGIN\cr
\nline \D=(c4\3-c6\2)/1728;\cr
%
\smallskip
\comm{Compute scaling factor $u$}
\smallskip
%
\nline u = 1; g = gcd(c6\2,\D);\cr
\nline p\_list = prime\_divisors(g);\cr
\nline FOR p IN p\_list DO\cr
\nline BEGIN\cr
\nline &d = floor(ord(p,g)/12);\cr
\nline &IF p=2 THEN\cr
\nline &&a = c4/2\pow{(4*d)} mod 16; b = c6/2\pow{(6*d)} mod 32;\cr
\nline &&IF (b mod 4 \NEQ\ -1) AND NOT (a=0 AND (b=0 OR b=8)) \cr
\nline &&THEN d = d-1\cr
\nline &&FI\cr
\nline &ELIF p=3 THEN IF ord(3,c6)=6*d+2 THEN d = d-1 FI \cr
\nline &FI;\cr
\nline &u = u*p\pow{d}\cr
\nline END;\cr
%
\smallskip
\comm{Compute minimal equation}
\smallskip
%
\nline c4 = c4/u\4; c6 = c6/u\6;\cr
\nline b2 = -c6 mod 12;  b4 = (b2\2-c4)/24; b6 = (-b2\3+36*b2*b4-c6)/216;\cr
\nline a1 = b2 mod 2;\cr
\nline a3 = b6 mod 2;\cr
\nline a2 = (b2-a1)/4;\cr
\nline a4 = (b4-a1*a3)/2;\cr
\nline a6 = (b6-a3)/4\cr
\nline END\cr

\endalg

Next we turn to Tate's algorithm itself.  The standard reference for
this is Tate's `letter to Cassels' \cite\Tate, which appeared in the
Antwerp IV volume \cite\Antwerp.  There is also a full account in the
second volume of Silverman's book \cite{\Silverd, Section IV.9}.  It may be
applied to an integral model of a curve $E$ and a prime $p$, to give
the following data:

$\bullet$ The exponent $f_p$ of $p$ in the conductor $N$ of $E$ (see below);

$\bullet$ the Kodaira symbol of $E$ at $p$, which classifies the type
of reduction of $E$ at $p$ (see \cite{\Neron} or \cite{\Silverd,
Section IV.9}); these are: I$_0$ for good reduction; I$_n$ ($n>0$) for
bad multiplicative reduction; and types I$^*_n$, II, III, IV, II$^*$,
III$^*$ and IV$^*$ for bad additive reduction.

$\bullet$ the local index $c_p=[E(\Q_p):E^0(\Q_p)]$, where $E^0(\Q_p)$
is the subgroup of the group $E(\Q_p)$ of $p$-adic points of $E$,
consisting of those points whose reduction modulo $p$ is non-singular.
(That this index is finite is implied by the correctness of the
algorithm, as observed by Tate in
\cite\Tate.)

In addition, the algorithm detects whether the given model is non-minimal 
at $p$, and if so, returns a model which is minimal at $p$.  Thus by
applying it in succession with all the primes dividing the discriminant of 
the original model, one can compute a minimal model at the same time as 
computing the conductor and the other local reduction data.  In practice 
this makes the Laska--Kraus--Connell algorithm redundant, though much 
simpler to implement and use if all one needs is the standard model for a 
curve $E$.

The conductor $N$ of an elliptic curve $E$ defined over $\Q$ is defined to
be 
$$  N = \prod_pp^{f_p}
$$
where $f_p=\ord_p(\Delta)+1-n_p$ and $n_p$ is the number of
irreducible components on the special fibre of the minimal N\'eron
model of $E$ at $p$.  This N\'eron model is a more sophisticated
object than we wish to discuss here (see \cite{\Neron} or
\cite\Silverd\ for details): one has to consider $E$ as a scheme over
Spec$(\Z_p)$, and then resolve the singularity at $p$, to obtain a
scheme whose generic fibre is $E/\Q_p$ and whose special fibre is a
union of curves over $\Z/p\Z$.  In terms of a minimal model for $E$
over $\Z$, all may be computed very simply except when $p=2$ or $p=3$
as follows:

$f_p=0$ if $p\ndiv\Delta$;

$f_p=1$ if $p\div\Delta$ and $p\ndiv c_4$ (then $n_p=\ord_p(\Delta)$);

$f_p\ge2$ if $p\div\Delta$ and $p\div c_4$; moreover, $f_p=2$ in this case when
$p\not=2,3$.

To obtain the value of $f_p$ in the remaining cases, and to obtain the Kodaira
symbol and the local index $c_p$, we use Tate's algorithm itself.

In \cite\Tate, the algorithm is given for curves defined over an
arbitrary discrete valuation ring.  To apply it to a curve defined
over the ring of integers $R$ of a number field $K$ at a prime ideal
$\frak p$, one would in general have to work in the localization of
$R$ at $\frak p$; here we can work entirely over $\Z$, since $\Z$ is a
principal ideal domain.  We have added to the presentation in
\cite\Tate\ the explicit coordinate transformations $T(r,s,t,u)$ which
are required during the course of the algorithm to achieve
divisibility of the coefficients $a_i$ by various power of $p$. In
practice one would ignore the transformations which had taken place
while processing each $p$, unless a scaling by $p$ had taken place on
discovering that the model was non-minimal.  The most complicated part
of the algorithm is the branch for reduction type I$^*_m$, where one
successively refines the model $p$-adically until certain auxiliary
quadratics have distinct roots modulo $p$.  This requires careful
book-keeping.  The presentation given here closely follows our own
implementation of the algorithm, which in turn owes much to an
earlier Fortran program written by Pinch.  The following
sub-procedures are used:

{\tt compute\_invariants} computes the $b_i$, $c_i$ and $\Delta$ from the 
coefficients $a_i$.  Note that $c_4$, $c_6$ and $\Delta$ do not change 
unless a scaling is required, since all other transformations have $u=1$.

{\tt transcoord(r,s,t,u)} applies the coordinate transformation formulae of 
the previous section to obtain new values for the $a_i$ and other 
quantities.  All calls to this procedure have $u=1$ except when rescaling
a non-minimal equation.  In each case we first compute suitable values of 
$r$, $s$ and $t$; usually this requires a separate branch if $p=2$ or 
$p=3$.

{\tt quadroots(a,b,c,p)} returns {\tt TRUE} if the quadratic congruence
$ax^2+bx+c\equiv0\pmod{p}$ has a solution, and {\tt FALSE} otherwise.  
This is used in determining the value of the index $c_p$.

{\tt nrootscubic(b,c,d,p)} returns the number of roots of the 
cubic congruence $x^3+bx^2+cx+d\equiv0\pmod{p}$.

\beginalg{Tate's Algorithm}

\+INPUT:  &&&a1, a2, a3, a4, a6 (integer coefficients of E); p (prime).\cr
\+OUTPUT: &&&Kp&(Kodaira symbol)\cr
\+        &&&fp&(Exponent of p in conductor)\cr
\+        &&&cp&(Local index)\cr
%
\smallskip \lineno=0
%
\nline BEGIN\cr
\nline compute\_invariants(b2,b4,b6,b8,c4,c6,\D);\cr
\nline n = ord(p,\D);\cr
%
\smallskip
\comm{Test for type I$_0$}
\smallskip
%
\nline IF n=0 THEN Kp = "I0"; fp = 0; cp = 1; EXIT FI;\cr
%
\smallskip
\comm{Change coordinates so that $p\mid a_3, a_4, a_6$}
\smallskip
%
\nline IF p=2 THEN\cr
\nline &IF p\DIV b2 \cr
\nline &THEN r = a4 mod p; t = r*(1+a2+a4)+a6 mod p\cr
\nline &ELSE r = a3 mod p; t = r+a4 mod p\cr
\nline &FI\cr
\nline ELIF p=3 THEN\cr
\nline &IF p\DIV b2 THEN r = -b6 mod p ELSE r = -b2*b4 mod p FI;\cr
\nline &t = a1*r+a3 mod p\cr
\nline ELSE\cr
\nline &IF p\DIV c4 THEN r = -inv(12,p)*b2 ELSE r = -inv(12*c4,p)*(c6+b2*c4) FI;\cr
\nline &t = -inv(2,p)*(a1*r+a3);\cr
\nline &r = r mod p; t = t mod p\cr
\nline FI;\cr
\nline transcoord(r,0,t,1);\cr
%
\smallskip
\comm{Test for types I$_n$, II, III, IV}
\smallskip
%
\nline IF p\NDIV c4 THEN\cr
\nline &IF quadroots(1,a1,-a2,p) THEN cp = n ELIF 2\DIV n THEN cp = 2 ELSE cp = 1 FI;\cr 
\nline &Kp = "In"; fp = 1; EXIT\cr
\nline FI;\cr
\nline IF p\2\NDIV a6 THEN Kp = "II";  fp = n;   cp = 1; EXIT; \cr
\nline IF p\3\NDIV b8 THEN Kp = "III"; fp = n-1; cp = 2; EXIT; \cr
\nline IF p\3\NDIV b6 THEN \cr
\nline &IF quadroots(1,a3/p,-a6/p\2,p) THEN cp = 3 ELSE cp = 1 FI;\cr
\nline &Kp = "IV";  fp = n-2; EXIT\cr
\nline FI;\cr
%
\smallskip
\comm{Change coordinates so that $p\mid a_1,a_2$; $p^2\mid a_3,a_4$; $p^3\mid a_6$}
\smallskip
%
\nline IF p=2\cr
\nline THEN s = a2 mod 2; t = 2*(a6/4 mod 2)\cr
\nline ELSE s = -a1*inv(2,p); t = -a3*inv(2,p)\cr
\nline FI;\cr
\nline transcoord(0,s,t,1);\cr
%
\smallskip
\comm{Set up auxiliary cubic $T^3 + bT^2 + cT + d$}
\smallskip
%
\nline b = a2/p; c = a4/p\2; d = a6/p\3;\cr
\nline w = 27*d\2-b\2*c\2+4*b\3*d-18*b*c*d+4*c\3;\cr
\nline x = 3*c-b\2;\cr
%
\smallskip
\comm{Test for distinct roots: type I$^*_0$}
\smallskip
%
\nline IF p\NDIV w THEN Kp = "I*0"; fp = n-4; cp = 1+nrootscubic(b,c,d,p); EXIT\cr
%
\smallskip
\comm{Test for double root: type I$^*_m$}
\smallskip
%
\nline ELIF p\NDIV x THEN \cr
%
\smallskip
\comm{Change coordinates so that the double root is $T\equiv0$}
\smallskip
%
%
\nline &IF p=2 THEN r = c ELIF p=3 THEN r = b*c ELSE r = (b*c-9*d)*inv(2*x,p) FI;\cr
\nline &r = p*(r mod p);\cr
\nline &transcoord(r,0,0,1);\cr
%
\smallskip
\comm{Make $a_3$, $a_4$, $a_6$ repeatedly more divisible by $p$}
\smallskip
%
\nline &m = 1; mx = p\2; my = p\2; cp = 0;\cr
\nline &WHILE cp=0 DO\cr
\nline &BEGIN\cr
\nline &&xa2 = a2/p; xa3 = a3/my; xa4 = a4/(p*mx); xa6 = a6/(mx*my);\cr
\nline &&IF p\NDIV (xa3\2+4*xa6) THEN\cr
\nline &&&IF quadroots(1,xa3,-xa6,p) THEN cp = 4 ELSE cp = 2 FI\cr
\nline &&ELSE \cr
\nline &&&IF p=2 THEN t = my*xa6 ELSE t = my*((-xa3*inv(2,p)) mod p) FI;\cr
\nline &&&transcoord(0,0,t,1);\cr
\nline &&&my = my*p; m = m+1;\cr
\nline &&&xa2 = a2/p; xa3 = a3/my; xa4 = a4/(p*mx); xa6 = a6/(mx*my);\cr
\nline &&&IF p\NDIV (xa4\2-4*xa2*xa6) THEN\cr
\nline &&&&IF quadroots(xa2,xa4,xa6,p) THEN cp = 4 ELSE cp = 2 FI\cr
\nline &&&ELSE\cr
\nline &&&&IF p=2 THEN r = mx*(xa6*xa2 mod 2) \cr
\nline &&&&ELSE r = mx*(-xa4*inv(2*xa2,p) mod p) \cr
\nline &&&&FI;\cr
\nline &&&&transcoord(r,0,0,1);\cr
\nline &&&&mx = mx*p; m = m+1\cr
\nline &&&FI\cr
\nline &&FI\cr
\nline &END;\cr
\nline &fp = n-m-4; Kp = "I*m"; EXIT\cr
%
\smallskip
%
\nline ELSE \cr
%
\smallskip
\comm{Triple root case: types II$^*$, III$^*$, IV$^*$ or non-minimal}
\comm{Change coordinates so that the triple root is $T\equiv0$}
\smallskip
%
\nline &IF p=3 THEN rp = -d ELSE rp = -b*inv(3,p) FI;\cr
\nline &r = p*(rp mod p);\cr
\nline &transcoord(r,0,0,1);\cr
\nline &x3 = a3/p\2; x6 = a6/p\4; \cr
%
\smallskip
\comm{Test for type IV$^*$}
\smallskip
%
\nline &IF p\NDIV (x3\2+4*x6) THEN\cr
\nline &&IF quadroots(1,x3,-x6,p) THEN cp = 3 ELSE cp = 1 FI;\cr
\nline &&Kp = "IV*"; fp = n-6; EXIT\cr
\nline &ELSE\cr
%
\smallskip
\comm{Change coordinates so that $p^3\mid a_3$, $p^5\mid a_6$}
\smallskip
%
\nline &&IF p=2 THEN t = x6 ELSE t = x3*inv(2,p) FI;\cr
\nline &&t = -p\2*(t mod p);\cr
\nline &&transcoord(0,0,t,1);\cr
%
\smallskip
\comm{Test for types III$^*$, II$^*$}
\smallskip
%
\nline &&IF p\4\NDIV a4 THEN Kp = "III*"; fp = n-7; cp = 2; EXIT\cr
\nline &&ELIF p\6\NDIV a6 THEN Kp = "II*"; fp = n-8; cp = 1; EXIT\cr
\nline &&ELSE\cr
%
\smallskip
\comm{Equation non-minimal: divide each $a_i$ by $p^i$ and start again}
\smallskip
%
\nline &&&transcoord(0,0,0,p); restart\cr
\nline &&FI\cr
\nline FI\cr
\nline END\cr

\endalg

In Table 1 we will give the local reduction data for each curve at each `bad'
prime (dividing the discriminant of the minimal model).  We also give the
factorization of the minimal discriminant and of the denominator of $j$, as in
the earlier tables.  To save space we omit the $c_4$ and $c_6$
invariants, which are easily computable from the coefficients $a_i$.

%
% CHAPTER 3 SECTION 3
%
\beginsection{\Torsion}
\head\Torsion\ Computing the Mordell--Weil group I: finding torsion points 
\endhead

In this and the next three sections we will discuss the question of
determining the Mordell--Weil group $E(\Q)$ of rational points on an
elliptic curve $E$ defined over $\Q$.  This group is finitely
generated, by Mordell's Theorem, and hence has the structure
$$
  E(\Q) = T \times F
$$
where $T$ is the finite torsion subgroup $E(\Q)_{\text{tors}}$ of
$E(\Q)$ consisting of the points of finite order, and $F$ is free
abelian of some rank $r\ge0$:
$$
     F \cong \Z^r.
$$
The problem of computing $E(\Q)$ thus subdivides into several parts:

$\bullet$ computing the torsion $T$;

$\bullet$ computing the rank $r$;

$\bullet$ finding $r$ independent points of infinite order;

$\bullet$ computing a $\Z$-basis for the free part $F$.

A related task is to compute the regulator $R(E(\Q))$ (defined below);
for this and for the latter two steps we will also need to compute the
canonical height $\canht(P)$ of points $P\in E(\Q)$, and hence the
height pairing $\canht(P,Q)$.

In this section we will treat the easiest of these problems, that of
finding the torsion points.  In fact, these can be found as a
byproduct of the more general search for points on the curve, since
their naive height can be bounded (see the remark before
Lemma~3.5.2).  However, it is also useful to have a self-contained
method for determining the torsion.

Using the fact that $E(\R)$ is isomorphic either to the circle group
$S^1$ (when $\Delta<0$) or to $S^1\times C_2$ (when $\Delta>0$), where
$C_k$ denotes a cyclic group of order $k$, together with the fact that
all finite subgroups of $S^1$ are cyclic, we see that $T$ is
isomorphic either to $C_k$ or to $C_{2k}\times C_2$ for some $k\ge1$,
the latter only being possible when $\Delta$ is positive.  The number
of possible values of $k$ is finite: by a theorem of Mazur
\cite{\MazurTa},\cite{\MazurTb}, a complete list of possible
structures of $T$ is
$$\align
    C_k &\qquad\text{for}\qquad 1\le k\le10 \quad\text{or}\quad k=12; \cr
    C_{2k}\times C_2 &\qquad\text{for}\qquad 1\le k\le 4.
  \endalign
$$

To determine the torsion subgroup of an elliptic curve defined over
$\Q$, we may use a form of the Lutz--Nagell Theorem.  (The situation
is more complicated over number fields other than $\Q$, on account of
the ramified primes.)  The first step is to find a model for the curve
in which all torsion points are integral.  For this it suffices to
complete the square (if necessary) to eliminate the $xy$ and $y$
terms, at the expense of a scaling by $u=2$.  Then for $P=(x,y)$ a
torsion point, we can use the fact that both $P$ and $2P$ are integral
to bound $y$.  For the first step, the following result may be found
in \cite{\Langc, Section III.1} and \cite{\Knapp, Theorem 5.1}.  The
original form of this result, due independently to Lutz \cite{\Lutz}
and Nagell \cite{\Nag}, was for curves of the form $y^2=x^3+ax+b$,
with no $x^2$ term.  While such an equation may be obtained by
completing the cube, this would involve a further scaling of
coordinates, and so would lead to larger numbers.  If $a_1=a_3=0$ we
can apply the following result directly; otherwise, put $a=b_2$,
$b=8b_4$ and $c=16b_6$.  \newprop{\LutzNaga}

\proclaim{Proposition \LutzNaga}Let $E$ be an elliptic curve defined over $\Q$,
given by an equation \neweq{\eqnabc}
$$
  y^2 = f(x) = x^3+ax^2+bx+c  \tag\eqnabc
$$
where $a,b,c\in\Z$.  If $P=(x,y)\in E(\Q)$ has finite order, then
$x, y\in\Z$.
\endproclaim

Next we bound the $y$ coordinate of a torsion  point $P=(x,y)$ (see
\cite{\Langc, Theorem 1.4}).  \newprop{\LutzNagb}

\proclaim{Proposition \LutzNagb}Let $E$ be as in {\rm (\eqnabc)}.
If $P=(x_1,y_1)$ has finite order in $E(\Q)$ then either $y_1=0$ or
$y_1^2\div\Delta_0$, where
$$
   \Delta_0 = 27c^2+4a^3c+4b^3-a^2b^2-18abc.
$$
\endproclaim
\demo{Proof} If $2P=0$ then $y_1=0$, since $-P=(x_1,-y_1)$.  Otherwise
$2P=(x_2,y_2)$ with $x_2,y_2\in\Z$ by Proposition \LutzNaga.  Using
the addition formula on $E$ we find that $2x_1+x_2=m^2-a$ where
$m=f'(x_1)/2y_1$ is the slope of the tangent to $E$ at $P$.  Hence
$m\in\Z$, so that $y_1\div f'(x_1)$.  Using $y_1^2=f(x_1)$, this
implies that $y_1^2|\Delta_0$, since
$$
  \Delta_0 = (-27f(x)+54c+4a^3-18ab)f(x) + (f'(x)+3b-a^2)f'(x)^2. \qed
$$
\enddemo

This gives us a finite number of values of $y$ to check; for each, we
attempt to solve the cubic for $x\in\Z$, to obtain all torsion points
on $E$. Note that we are actually determining all points $P$ such that
both $P$ and $2P$ are integral (in the possibly scaled model for $E$),
which includes all torsion points, but may also include points of
infinite order.  To determine whether a given integral point has
finite or infinite order, we simply compute multiples $mP$
successively until either $mP=0$, in which case $P$ has order $m$, or
$mP$ is not integral, in which case $P$ has infinite order.  This does
not take long, as the maximum possible order for a torsion point is 12
by Mazur's theorem.  If we find points of infinite order at this stage
we keep a note of them for later use (see Section \Findinf).

The quantity $\Delta_0$ is related to the discriminant $\Delta$ of the
curve (\eqnabc) by $\Delta=-16\Delta_0$.  If this is large, there may
be many values of $y_0$ to check when we apply the preceding
Proposition to determine the torsion on a given curve.  It is possible
to save time by using a further result, which states that for an odd
prime~$p$ of good reduction (that is, $p\ndiv2\Delta$), the reduction
map from $E(\Q)_{\tor}$ to $E(\Z/p\Z)$ is injective.  For more
details, and worked examples, see either \cite{\Silvera, Section
VIII.7} or \cite{\Knapp, Section V.1}.

If we want to know the structure of $T$ and not just its order, note
that from Mazur's theorem the only ambiguous cases are when $T$ has
order $4k=4$, 8 or 12 and $\Delta>0$; we can always tell apart the
groups $C_{4k}$ and $C_2\times C_{2k}$ as the former has only one
element of order 2 while the latter has three, and this number is the
number of rational (integer) roots of $f(x)$.

To solve the cubic equations $f(x)=y^2$ for $x$, given $y$, we use the
classical formula of Cardano (see any algebra textbook) to find the
complex roots (which we also need in computing the periods in section
\Periods\ below), and if any of these are real and close to integers
we check them using exact integer arithmetic.  Testing all divisors of
the constant term can be too time-consuming, as it involves
factorization of the numbers $y^2-c$ which may be very large.

Here is the algorithm in pseudocode; for simplicity we only give it
for curves with no $xy$ or $y$ terms; in the general case, one works
internally with points on a scaled model (including the calculation of
the order), converting back to the original model on output.  Since we
know in advance that no point will have order greater than 12, when
computing the order of a point we simply use repeated addition until
we reach a non-integral point or the identity {\tt O}.  The subroutine
{\tt order(P)} returns 0 for a point of infinite order.  Also: {\tt
square\_part(\D)} returns the largest integer whose square divides \D;
{\tt integer\_roots} returns a list of the integer roots of a cubic
with integral coefficients; and {\tt integral(x)} tests whether its
(rational) argument is integral.


\beginalg{Algorithm for finding all torsion points}
%
\+INPUT:   &&&a,b,c (integer coefficients of a nonsingular cubic).\cr
\+OUTPUT:  &&&A list of all torsion points on y\2=x\3+ax\2+bx+c, with orders.\cr % \smallskip \lineno=0
\smallskip\lineno=0
%
\nline BEGIN\cr
\nline \D=27*c\2+4*a\3*c+4*b\3-a\2*b\2-18*a*b*c; \cr
\nline y\_list=positive\_divisors(square\_part(\D)) $\cup$ \{0\}; \cr
\nline FOR y IN y\_list DO \cr
\nline BEGIN \cr
\nline &x\_list=integer\_roots(x\3+a*x\2+b*x+c-y\2); \cr
\nline &FOR x IN x\_list DO \cr
\nline &BEGIN \cr
\nline &&P=point(x,y); \cr
\nline &&n=order(P); \cr
\nline &&IF n>0 THEN OUTPUT P,n FI \cr
\nline &END \cr
\nline END \cr
\nline END \cr
%
\smallskip
\comm{Subroutine to compute order of a point}
\smallskip\lineno=0
%
\+SUBROUTINE order(P) \cr
\nline BEGIN \cr
\nline n=1; Q=P; \cr
\nline WHILE integral(x(Q)) AND Q$\not=$O DO \cr
\nline BEGIN \cr
\nline &n = n+1; Q = Q+P \cr
\nline END; \cr
\nline IF Q$\not=$O THEN n=0 FI; \cr
\nline RETURN n \cr
\nline END \cr

\endalg

%
% CHAPTER 3 SECTION 4
%
\beginsection{\Heights}
\head\Heights\ Heights and the height pairing\endhead


In this section we will show how to compute the canonical height
$\canht(P)$ of a point $P\in E(\Q)$, and hence the height pairing 
$$
   \canht(P,Q)=\frac{1}{2}(\canht(P+Q)-\canht(P)-\canht(Q)).
$$
We will use this in the following section to find dependence relations
among finite sets of points of infinite order, when we are computing a 
$\Z$-basis $\{P_1,\ldots,P_r\}$ for the free abelian group $E(\Q)/T$.
Also, the regulator $R(E)$ is given by the determinant of the height
pairing matrix:
$$
   R(E) = \left|\det(\canht(P_i,P_j))\right|.
$$
The canonical height $\canht$ is a real-valued quadratic form on $E(\Q)$. It
differs by a bounded amount (with a bound dependent on $E$ but not on the point
$P$) from the naive or Weil height $h(P)$.  For a point
$P=(x,y)=(a/c^2,b/c^3)\in E(\Q)$ with $a,b,c\in\Z$ and $\gcd(a,c)=1=\gcd(b,c)$,
the latter is defined to be 
$$
    h(P) = \log \max\{|a|,c^2\}.
$$
Now the canonical height may be defined as $\canht(P)=\lim_{n
\to\infty}4^{-n}h(2^nP)$, but this is not practical for  computational
purposes.  For the theory of heights on elliptic curves, see \cite{\Silvera,
Chapter VIII}.  Later (in the next section) we will need an explicit bound on
the difference between $\canht(P)$ and $h(P)$.

The height algorithms in this section are taken from Silverman's paper
\cite\Silverb.  The \und{global height} $\canht(P)$ is defined as a
sum of \und{local heights}:\neweq{\htsuma}
$$
   \canht(P) = \sum_{p\le\infty}\canht_p(P). \tag\htsuma
$$
Here the sum is over all finite primes $p$ and the `infinite prime' $\infty$
coming from the real embedding of $\Q$.  (Over a general number field, there
would in general be several of these infinite primes, including complex ones,
and the local heights need to be multiplied by certain multiplicities: see
\cite\Silverb).  

A remark about normalization\footnote{I am grateful to Gross for
explaining this to me, after I found that apparently the two sides of
the \BSD\ conjecture disagreed by a factor of $2^r$!}: the canonical
height must be suitably normalized.  In the literature there are two
normalizations used, one of which is double the other and is the one
appropriate for the \BSD\ conjecture (resulting in a regulator $2^r$
times as large).  In Silverman's paper he uses the other (smaller)
normalization.  Thus all the formulae here are double those in the
paper \cite\Silverb.

The following proposition, which is Theorem 5.2 of \cite{\Silverb}
(for curves over general number fields) specialized to the case of a
curve defined over $\Q$, also applies to a curve defined over $\Q_p$
and to a point $P=(x,y)\in E(\Q_p)$. In the proposition, we refer to
the functions $\psi_2$ and $\psi_3$ defined on $E$ by
$$
  \psi_2(P) = 2y+a_1x+a_3, \qquad\text{and}\qquad
  \psi_3(P) = 3x^4+b_2x^3+3b_4x^2+3b_6x+b_8;
$$
thus, $\psi_2$ vanishes at the 2-torsion points of $E$ and $\psi_3$ at
the 3-torsion. \newprop{\SilverTheorem}

\proclaim{Proposition \SilverTheorem}  Let $E$ be an elliptic curve defined
over $\Q$ given by a standard Weierstrass equation {\rm(\curveeq)}
which is minimal at $p$, and let $P=(x,y)\in E(\Q)$.
\part{a} If 
$$
  \ord_p(3x^2+2a_2x+a_4-a_1y)\le0 \qquad\text{or}\qquad
  \ord_p(2y+a_1x+a_3)\le0
$$
then 
$$
  \canht_p(P) = \max\{0,-\ord_p(x)\}\log p.
$$
\part{b} Otherwise, if $\ord_p(c_4)=0$ then set $N=\ord_p(\Delta)$ and
$M=\min\{\ord_p(\psi_2(P),\frac12N\}$; then
$$
   \canht_p(P) = \frac{M(M-N)}{N}\log p.
$$
\part{c}Otherwise, if $\ord_p(\psi_3(P))\ge3\ord_p(\psi_2(P))$ then
$$
   \canht_p(P) = -\frac23\ord_p(\psi_2(P))\log p.
$$
\part{d}Otherwise
$$
   \canht_p(P) = -\frac14\ord_p(\psi_3(P))\log p.
$$
\endproclaim

The first case in Proposition \SilverTheorem\ covers primes $p$ where
the point $P$ has good reduction (including all primes where $E$ has
good reduction, as well as those where the reduced curve is singular
but $P$ does not reduce to the singular point).  In the other three
cases, $P$ has singular reduction, and the reduction of $E$ at $p$ is
multiplicative, additive of types IV or IV$^*$, and additive of types
III, III$^*$ and I$^*_m$ respectively.

Hence for each point $P$, the local height $\canht_p(P)=0$ if $p$
divides neither the discriminant~$\Delta$ nor~$c$, where $c^2$ is the
denominator of the $x$-coordinate of the point $P$.  In all cases,
$\canht_p(P)$ is a rational multiple of $\log(p)$.  The total
contribution from the primes dividing $c$ in the global height
$\canht(P)$ is therefore (from case (a) of the Proposition) simply
$2\log(c)$, and we have the following formula, better for practical
computation than (\htsuma) since we do not have to factorize
$c$:\neweq{\htsumb}
$$
   \canht(P) = \canht_{\infty}(P) + 2\log(c) 
                + \sum_{p\div\Delta, p\ndiv c}\canht_p(P).\tag\htsumb
$$
This formula appears in \cite\Silvere, where it is shown how to
compute $\canht(P)$ using little (or no) factorization of $\Delta$,
which can be useful in certain situations.  We refer the reader to
\cite\Silvere\ for details.

An algorithm for computing the local height at a finite prime~$p$ is
given by the following:

\beginalg{Silverman's algorithm for computing local heights: finite primes}
 
\+INPUT:   && a1, a2, a3, a4, a6 (integer coefficients of a minimal model for
$E$).\cr
\+         && x,y (rational coordinates of a point $P$ on $E$).\cr
\+         && p (a prime).\cr
\+OUTPUT:  && the local height of $P$ at p.\cr
%
\smallskip \lineno=0
%
\nline BEGIN\cr
\nline compute\_invariants(b2,b4,b6,b8,c4,\D);\cr
\nline N = ord(p,\D);\cr
\nline A = ord(p,3*x\2+2*a2*x+a4-a1*y);\cr
\nline B = ord(p,2*y+a1*x+a3);\cr
\nline C = ord(p,3*x\4+b2*x\3+3*b4*x\2+3*b6*x+b8);\cr
\nline M = min(B,N/2);\cr
%
\smallskip
%
\nline IF A $\le$ 0 OR B $\le$ 0 THEN L = max(0,-ord(p,x))\cr
\nline ELSE IF ord(p,c4)=0 THEN L = M*(M-N)/N\cr
\nline ELSE IF C $\ge$ 3*B THEN L = -2*B/3\cr
\nline ELSE L = -C/4\cr
\nline FI;\cr
\nline RETURN L*log(p)\cr
\nline END\cr

\endalg

We must also compute the local component of the height at the infinite
prime, $\canht_{\infty}(P)$.  The method here originated with Tate,
but was amended by Silverman in \cite\Silverb\ to improve convergence,
and to apply also to complex valuations.  Tate in \cite\Tateht\
expressed $\canht_{\infty}(P)$ as a series
$$
  \canht_{\infty}(P) = \log|x| + \frac14\sum_{n=0}^{\infty}4^{-n}c_n
$$
where the coefficients $c_n$ are bounded provided that no point on
$E(\R)$ has $x$-coordinate zero. Of course, over $\R$ one can shift
coordinates to ensure that this condition holds, but the resulting
series can have poor convergence properties, and this trick will not
work over $\C$.  Silverman's solution is to use alternately the
parameters $x$ and $x'=x+1$, switching between them (and between the
two associated series $c_n$ and $c_n'$) whenever $|x|$ or $|x'|$
becomes small (less than $1/2$).  The series of coefficients $c_n$ is
obtained by repeated doubling of the point $P$, working with $t=1/x$
or $t'=1/x'$ as local parameter.  The result is a new series of the
above type in which the error in truncating before the $N$th term is
$O(4^{-N})$, with an explicit constant.  In fact (see \cite{\Silverb,
Theorem 4.2}) the error is less than $\frac1210^{-d}$, giving a result
correct to $d$ decimal places, if
$$
  N \ge \frac53d + \frac12 + 
 \frac34\log(7+\frac43\log H+\frac13\log\max\{1,|\Delta|^{-1}\})
$$
where
$$
  H = \max\{4,|b_2|,2|b_4|,2|b_6|,|b_8| \}.
$$
The last term vanishes for curves defined over $\Z$, since then we have 
$|\Delta|>1$.

In the algorithm which we now give, the quantities {\tt b2', b4', b6'}
and {\tt b8'} are those associated with the shifted model of $E$ with 
$x'=x+1$; the switching flag {\tt beta} indicates which model we are 
currently working on; {\tt mu} holds the current partial sum; {\tt f} holds the
negative power of~4.

\beginalg{Silverman's algorithm for computing local heights: real component}

\+INPUT:   && a1, a2, a3, a4, a6 (integer coefficients of a minimal model for
$E$).\cr
\+         && x (x-coordinate of a point $P$ on $E$).\cr
\+         && d (number of decimal places required).\cr
\+OUTPUT:  && the real local height of $P$.\cr
%
\smallskip \lineno=0
%
\nline BEGIN\cr
\nline compute\_invariants(b2,b4,b6,b8);\cr
\nline H = max(4,|b2|,2*|b4|,2*|b6|,|b8|);\cr
\nline b2' = b2-12; b4' = b4-b2+6; b6' = b6-2*b4+b2-4; b8' = b8-3*b6+3*b4-b2+3;\cr
\nline N = ceiling((5/3)*d + (1/2) + (3/4)*log(7+(4/3)*log(H)));\cr
\nline IF |x|<0.5 THEN t = 1/(x+1); beta = 0 ELSE t = 1/x; beta = 1 FI;\cr
\nline mu = -log|t|; f = 1;\cr
%
\smallskip
%
\nline FOR n = 0 TO N DO\cr
\nline BEGIN\cr
\nline &f = f/4; \cr
\nline &IF beta=1 THEN \cr
\nline &&w = b6*t\4+2*b4*t\3+b2*t\2+4*t; \cr
\nline &&z = 1-b4*t\2-2*b6*t\3-b8*t\4; \cr
\nline &&zw = z+w \cr
\nline &ELSE \cr
\nline &&w = b6'*t\4+2*b4'*t\3+b2'*t\2+4*t; \cr
\nline &&z = 1-b4'*t\2-2*b6'*t\3-b8'*t\4; \cr
\nline &&zw = z-w \cr
\nline &FI;\cr
\nline &IF |w| $\le$ 2*|z| \cr
\nline &THEN mu = mu+f*log|z|;  t = w/z \cr
\nline &ELSE mu = mu+f*log|zw|; t = w/zw; beta = 1-beta \cr
\nline &FI\cr
\nline END;\cr
\nline RETURN mu\cr
\nline END\cr

\endalg

Finally, to compute the global height $\canht(P)$, we simply add to
the infinite local height $\canht_{\infty}(P)$ the finite local
heights $\canht_p(P)$ for all primes $p$ dividing either $\Delta$ or
the denominator of $x(P)$.  Using (\htsumb) this leads to the
following algorithm.

\beginalg{Algorithm for computing global canonical heights}
 
\+INPUT:   && a1, a2, a3, a4, a6 (integer coefficients of a minimal model for
$E$).\cr
\+         && P=(x,y) (a rational point $P$ on $E$).\cr
\+OUTPUT:  && the global canonical height $\canht(P)$ of $P$.\cr
%
\smallskip \lineno=0
%
\nline BEGIN\cr
\nline \D = discr(a1,a2,a3,a4,a6);\cr
\nline d = denom(x);\cr
\nline h = real\_height(P) + log(d);\cr
\nline p\_list = prime\_divisors(\D);\cr
\nline FOR p IN p\_list DO\cr
\nline BEGIN\cr
\nline &IF p\NDIV d THEN h = h + local\_height(p,P) FI\cr
\nline END;\cr
\nline RETURN h\cr
\nline END\cr

\endalg


%
% CHAPTER 3 SECTION 5
%
\beginsection{\Findinf}
\head\Findinf\ The Mordell--Weil group II: generators\endhead
\newprop{\Silverbound}

In this section we will show how we look for rational points of
infinite order on an elliptic curve $E$.  In compiling the tables, we
usually knew the rank $r$ in advance so that we knew how many
independent points to expect to find (and only looked for such points
when we knew that $r>0$); however, this procedure is also useful as an
open-ended search when we do not know the rank, as obviously it can
provide us with a lower bound for $r$.

The procedure divides into two parts.  First, we have a searching
routine which looks for points up to some bound on the naive height
(equivalently, some bound on the numerator and denominator of the
$x$-coordinate).  As this routine finds points, it gives them to the
second routine, which has at each stage a $\Z$-basis for a subgroup
$A$ of $E(\Q)/T$: initially $A=0$.  This second routine uses the
height pairing to determine one of three possibilities: the new point
$P$ may be independent of those already found and can then be added to
our cumulative list of independent points; the rank of $A$ is thus
increased by~1. Secondly, $P$ may be an integral combination of the
current basis (modulo torsion) and can then be ignored. Finally, if a
multiple $kP$ of $P$ is an integral combination of the current basis
for some $k>1$, we can find a basis for a new subgroup $A$ which
contains the old $A$ with index $k$.  Even when we know the rank $r$
in advance, we do not stop as soon as we have a subgroup $A$ of rank
$r$, since $A$ might still have finite index in $E(\Q)/T$.  To close
this final gap we use explicit bounds for the difference between the
naive and canonical heights, such as Silverman's result (Proposition
\Silverbound) below.

The algorithm we use for the second procedure is a very general one,
which can be used in many other similar situations; for example, as
part of an algorithm for finding the unit group of a number field,
where the first routine somehow finds units.  Our algorithm is
essentially the same as the `Algorithm for enlarging sublattices' in
the book by Pohst and Zassenhaus \cite{\PZass, Chapter 3.3}.

A rational point $P$ on $E$ (given by a standard Weierstrass equation)
may be written uniquely as $P = (x,y)=(a/c^2,b/c^3)$ with integers
$a$, $b$, and $c$ satisfying $\gcd(a,c)=\gcd(b,c)=1$ and $c\ge1$.  The
naive or Weil height of $P$ is $h(P)=\log\max\{|a|,c^2\}$. Initially,
we find the point of order~2 in $E(\R)$ with minimal $x$-coordinate
$x_0$; this gives a lower bound for the $x$-coordinates of all real
points on $E$.  We then search for points $P$ with naive height up to
some bound $B$ by looping through positive integers $c\le\exp(B/2)$
and through $a$ coprime to $c$ in the range
\footnote{If $E(\R)$ has three points of order two, with
$x$-coordinates $x_0<x_1<x_2$, then we also omit those $a$ for which
$c^2x_1<a<c^2x_2$.}  $\max\{c^2x_0,-\exp(B)\}\le a\le\exp(B)$.  Given
$a$ and $c$, we attempt to solve the appropriate quadratic equation
for $b\in\Z$.  To speed up this procedure, we use a quadratic sieve:
for each denominator $c$ we precompute for about 10 auxiliary sieving
primes~$p$ the residue classes modulo~$p$ to which $a$ must belong if
the equation for $b$ is to be soluble modulo~$p$. Each candidate value
of $a$ can then first be checked to see if it is admissible modulo
each sieving prime before the more time-consuming step of attempting
to solve for $b$.  This improvement to the search results in a major
time saving in most cases, though for most of the curves in our tables
on which we expected to find points of infinite order, such a point
was found very quickly anyway. (In some cases we had already found
such a point during the search for torsion points.)   In practice it
may be better to use composite moduli for the sieving.

Each point $P$ found by this search is passed to the second procedure,
which tests whether it has infinite order, discarding it if not.  At
the general stage we will have $k$ independent points $P_i$ for $1\le
i\le k$ (initially $k=0$) which generate a subgroup~$A$ of rank~$k$,
and will have stored the $k\times k$ height pairing matrix
$M=(h(P_i,P_j))$ and its determinant $R$.  Now we set $P_{k+1}=P$ and
compute $\canht(P_i,P_{k+1})$ for $i\le k+1$ to obtain a new height
pairing matrix of order $k+1$.  If the determinant of this new matrix
is non-zero, the new point is independent of the previous ones and we
add it to the current list of generators, increment $k$, replace $R$
by the new determinant, and go on with the point search.  If the new
determinant is zero, however, we use the values $h(P_i,P)$ to express
$P_{k+1}$ as a linear combination of the $P_i$ for $i\le k$, with
approximate real coefficients: in fact we have
$$
   a_1P_1 + a_2P_2 + \ldots + a_kP_k + a_{k+1}P_{k+1} = 0 
                                         \quad\text{(modulo torsion)}
$$ 
where for $1\le i\le k+1$ the coefficient $a_i$ is the $(i,k+1)$
cofactor of the enlarged matrix, which we will have stored during the
computation of the new determinant.  In particular, $a_{k+1}$ is (up
to sign) the previous value of~$R$, and hence is non-zero.  
%
Next we find rational approximations to these floating-point
coefficients $a_i$ (using continued fractions, or MLLL if available),
%
and clear denominators to obtain a new equation of the same form with
coprime integer coefficients $a_i$, which we can check holds exactly.
In this relation we still have $a_{k+1}\not=0$ (the first $k$ points
are independent).  The simplest case now is when $a_{k+1}=\pm1$, for
then $P_{k+1}$ is redundant and can be discarded.  Similarly, if
$a_i=\pm1$ for some $i\le k$, then we may discard $P_i$, replacing it
by $P_{k+1}$, and gaining index $|a_{k+1}|$.  In general let $a_i$ be
the minimal non-zero coefficient (in absolute value); if $|a_i|>1$, we
find a coefficient $a_j$ not divisible by $a_i$ (which must exist
since the coefficients are coprime) and write $a_j=a_iq+b$ where
$0<b<|a_i|$.  Now since
$$
  a_iP_i+a_jP_j = a_iP_i + (a_iq+b)P_j = a_i(P_i+qP_j) + bP_j 
$$ 
we may replace the generator $P_i$ by $P_i+qP_j$, replace the
coefficient $a_j$ by $b$ (which is smaller than $|a_i|$), and replace
$i$ by $j$.  After a finite number of steps we obtain a minimal
coefficient $a_i=1$ and can discard the current generator $P_i$,
leaving a new set of $k$ independent generators which generate a group
larger than before by a finite index equal to the original value of
$|a_{k+1}|$.

In this way, we will be able to find a $\Z$-basis for the subgroup $A$
of the Mordell--Weil group (modulo torsion) which is generated by the
points of naive height less than the bound~$B$.  Often we know the
rank $r$ of our curve in advance, so that we can increase $B$ until
$A$ has rank~$r$.  Then $A$ has finite index in $E(\Q)$, and we must
enlarge it to give the whole of $E(\Q)$.  There are various methods
one can use here, all of which rely on having explicit bounds for the
difference between the naive and canonical heights on the curve $E$.
The simplest general bound here is a result of Silverman (see
\cite{\Silverc}).  One can certainly often obtain better bounds for
individual curves, and there are also more complicated results which
apply in general and which usually give much better bounds, such as
the main result of \cite{\SSpaper}.

For simplicity we will only give Silverman's version of the bound.  In
the following proposition,
\footnote{When referring to \cite\Silverc, recall that our $\canht$ is
double Silverman's; also, the constant 1.922 appearing here is a
(normalized) correction, due to Bremner, of the constant in Silverman's paper.}
the height of a rational number $a/b$ with $\gcd(a,b)=1$ is
$h(a/b)=\log\max\{|a|,|b|\}$, and $\log^+(x)=\log\max\{1,|x|\}$ for
$x\in\R$.

\proclaim{Proposition \Silverbound}
Let $E$ be an elliptic curve defined by a standard Weierstrass equation over 
$\Z$, with discriminant $\Delta$ and $j$-invariant $j$.  Set $2^*=2$ if 
$b_2\not=0$, or $2^*=1$ if $b_2=0$.  Define
$$
 \mu(E) = \frac{1}{6}\left(\log|\Delta| + \log^+(j)\right) + \log^+(b_2/12) + \log(2^*).
$$
Then for all $P\in E(\Q)$,
$$
    -\frac1{12}h(j)-\mu(E)-1.922 \le \canht(P)-h(P) \le \mu(E)+2.14.
$$
\endproclaim

This result is easiest to apply in the rank 1 case, as follows.
Suppose we have a rational point $P$ of infinite order on $E$, of
height $\canht(P)$. If $P$ is not a generator it is a multiple $P=kQ$
(modulo torsion) of some generator $Q$, where $k\ge2$, so that
$\canht(Q)\le\frac{1}{4}\canht(P)$. By the preceding proposition we
can bound the naive height of $Q$ and adjust the bound $B$ in our
search accordingly. If a further search up to this bound finds no more
points, then $P$ was a generator after all; otherwise we are sure to
find a generator.

Similar techniques are possible in higher rank situations, using
estimates from the geometry of numbers.  See the papers \cite{\Zimmer}
and \cite{\SSpaper} for more details.

We may also remark that since $P$ has finite order if and only if
$\canht(P)=0$, the proposition implies that all torsion points have
naive height $h(P)\le\frac{1}{12}h(j)+\mu(E)+1.922$, giving us another
way of finding all the rational torsion points.

For the general case, the following simple result\footnote{Attributed
in \cite\Silverc\ to Zagier, it is also exercise 5 on page 84 of
Cassels' book \cite{\Cassels}.} may be used.  \newprop{\Zbasisbound}

\proclaim{Lemma \Zbasisbound} Let $B>0$ be such that
$$
  S  =  \{ P\in E(\Q) \mid \canht(P) \le B \}
$$
contains a complete set of coset representatives for $2E(\Q)$ in
$E(\Q)$.  Then $S$ generates $E(\Q)$.  \endproclaim \demo{Proof} Let
$A$ be the subgroup of $E(\Q)$ modulo torsion generated by the points
in $S$.  Suppose that $A$ is a proper subgroup; then we may choose
$Q\in E(\Q)-A$ with $\canht(Q)$ minimal, since $\canht$ takes a
discrete set of values.  By hypothesis, there exist $P\in A$ and $R$
such that $Q=P+2R$; certainly $R\not\in A$, so that $\canht(R)\ge
\canht(Q)$ by minimality.  Now using the fact that $\canht$ is
quadratic and non-negative we obtain a contradiction:
$$\align
          \canht(P) &=   \frac12(\canht(Q+P)+\canht   (Q-P))-\canht(Q)  \\
                    &\ge \frac12\canht(2R)-\canht(Q)                    \\
                    &=   2\canht(R)-\canht(Q) \ge \canht(Q) > B.\qed     \\
  \endalign
$$
\enddemo

We have two ways of using this in practice.  First of all, it is
possible to obtain from the two-descent procedure which we use to
determine the rank (see the next section), a set of coset
representatives for $E(\Q)$ modulo $2E(\Q)$.  Computing the heights of
these points we can find a $B$ for which the Lemma holds, to which we
add the maximum difference between naive and canonical heights from
the preceding proposition to get a bound on the naive heights of a set
of generators.

Alternatively, assuming that we know the rank $r$, we first run our
search until we find $r$ independent points $P_i$.  Now it is easy to
check whether a point $P$ is twice another: if any subset of the $P_i$
sums to $2Q$ for some $Q$ we replace one of the $P_i$ in the sum by
$Q$ and gain index~2.  After a finite number of steps (since we are in
a finitely-generated group) we obtain independent points which are
independent modulo~2, and proceed as before.

Again, we have only presented here the most straightforward strategies
for enlarging a set of~$r$ independent points in $E(\Q)$ to a full
$\Z$-basis; this is a topic of active research, with new ideas being
developed rapidly: see the paper \cite{\SSpaper} for some recent
advances.

Putting the pieces together, we can determine a set of generators for
$E(\Q)$ modulo torsion, and then compute the regulator, provided that
we know its rank.  If we do not know the rank, we at least can obtain
lower bounds for the rank.  Together with the torsion points found in
section \Torsion, we will have determined the Mordell--Weil group
$E(\Q)$ explicitly.  Computing the rank is the subject of the next
section.

%
% CHAPTER 3 SECTION 6
%
\beginsection{\Rank}
\head\Rank\ The Mordell--Weil group III: the rank\endhead

For an elliptic curve $E$ defined over the rationals, the rank of the
Mordell--Weil group $E(\Q)$ is by far the hardest of the elementary
quantities associated with $E$ to compute, both theoretically and in
terms of implementation.  Strictly speaking, the two-descent
algorithms we will describe are not algorithms at all, as they are not
guaranteed to terminate in all cases.  One part of the procedure
involves establishing whether or not certain curves of genus one have
rational points, when they are known to have points everywhere locally
(that is, over $\R$ and over the $p$-adic field $\Q_p$ for all primes
$p$): there is no known algorithm to decide this in general.
Moreover, even without this difficulty, for curves with large
coefficients and no rational points of order two, the general
two-descent algorithm takes too long to run in practice.  For
simplicity, we will refer to the procedures as rank algorithms,
although their output in certain cases will be bounds on the rank
rather than its actual value.

We originally decided to implement a general two-descent procedure in
order to check that the modular curves we had computed did have their
rank equal to the analytic rank, which we knew, as described in the
previous chapter.  This was a somewhat thankless task, as it involved
a large programming effort, and a large amount of computer time to run
the resulting program, in order to verify that approximately 2500
numbers did in fact have the values 0, 1 or~2 which we were already
sure were correct. Since the project started, the major theoretical
advances by Kolyvagin, Rubin and others meant that all the cases of
rank~0 or~1 were known anyway, which left just 18 cases of conjectured
rank~2 to verify.  In the end we were able to verify these cases, and
to check all but a few dozen of the rank~0 or~1 curves; we also
obtained extra information by the two-descent procedure, such as the
$2$-rank of the \TS\ group \Sha, and a set of coset representatives for
$E(\Q)/2E(\Q)$.

Since the original implementation, the algorithm has been much
improved in many ways (notably the syzygy sieve in the search for
quartics, the systematic use of group structure in the $2$-isogeny case,
and the use of quadratic sieving in searching for rational points on
homogeneous spaces: see below for details).  Our program {\tt mwrank},
\footnote{Available from the author's ftp site: see the Introduction
for details.}  based on the algorithm, now works well on a much larger
set of curves, including some of fairly high rank such as a curve of
Fermigier \cite\Ferm\ with rank~13 and $2$-torsion (see the example
below), and several curves with no $2$-torsion and ranks 6,~7 and~8.
However, curves with extremely large coefficients, such as Nagao's
curve of rank (at least)~21 (see \cite\Nagao), are beyond the reach of
this algorithm owing to the enormous search regions required.  One can
also use the program {\tt mwrank} to find points on curves which are
too large to find by the search methods of the previous section.

We will not describe here the theory of two-descent, which is the
basis of the algorithm, in great detail.  Roughly speaking, one has an
injective homomorphism from $E(\Q)/2E(\Q)$ into a finite elementary
abelian $2$-group, the $2$-Selmer group, and attempts to determine the
image; if this has order $2^t$ then the rank of $E(\Q)$ is $t$, $t-1$
or $t-2$ according to whether the number of points of order $2$ in
$E(\Q)$ is 0, 1 or 3 (respectively).  This procedure applies to
arbitrary curves, and is called \und{general two-descent}.  When $E$
has a rational point $P$ of order two, there is a rational $2$-isogeny
$\phi:E\to E' = E/\left<P\right>$ and a dual isogeny $\phi':E'\to E$.
We may then proceed differently, using a procedure we call
\und{two-descent via $2$-isogeny}: we embed each of $E'/\phi(E)$ and
$E/{\phi'}(E')$ into finite subgroups of $\Q^*/(\Q^*)^2$, which are
easy to write down.  This is in contrast to the general two-descent,
where one has to work hard to find the Selmer group itself.  A full
description of two-descent can be found in the standard references
such as the books by Silverman
\cite\Silvera, Husem\"oller
\cite\Husemoller, Knapp
\cite\Knapp, or Cassels 
\cite\Cassels, but the descriptions given there
are only easy to apply when $E$ has all its $2$-torsion rational.  For
the general case where there are no rational points of order~2, the
main reference is one of the original papers \cite\BirchSD\ by Birch
and Swinnerton--Dyer on their Conjecture, and we followed that paper
closely in writing the first version of our program.  More detail on
the invariant theory, which has resulted in substantial improvements
to the general two-descent algorithm, can be found in the paper
\cite\JCinvariants; a very full description of the algorithm, together
with its extension to real quadratic number fields (see also
\cite\JCPS), can be found in Serf's thesis \cite\PSthesis.

Both algorithms involve the classification of certain curves,
associated with the given elliptic curve $E$, called \und{principal
homogeneous spaces}.  These are twists of $E$: curves of
genus~1 isomorphic to~$E$ over an extension field, but not
(necessarily) over $\Q$ itself; they need not have rational points, so
need not themselves be elliptic curves.  When they do have rational
points, these map to rational points on $E$; the maps $H\to E$ are
called \und{$2$-coverings} and have degree 4 (in the general
two-descent) or 2 (in the $2$-isogeny descent).  The homogeneous spaces
which arise in both algorithms have equations of the form
\neweq\phseqn
$$
   H: \qquad   y^2 = g(x) = ax^4+bx^3+cx^2+dx+e \tag\phseqn 
$$
where $g(x)$ is a quartic polynomial with rational coefficients.  For
brevity we will usually refer to these principal homogeneous spaces simply as
\und{quartics}.  The invariants $I$ and $J$ of $g(x)$ (see below for
their definition) are related to the invariants $c_4$ and $c_6$ of
either $E$ or the $2$-isogenous curve~$E'$.  In the case of descent via
$2$-isogeny, $g(x)$ will in fact be a quadratic in $x^2$.  We will be
interested in whether the quartic $H$ has points over $\Q$ or
one of its completions, the $p$-adics $\Q_p$ or the reals $\R$.  Such
a point will either be an affine point $(x,y)$ satisfying the equation
(\phseqn), or one of the two points at infinity on the projective
completion of $H$, which are rational if and only if $a$ is a square.

In all cases, a quartic with a (global) rational point $(x,y)$ will
lead to a rational point on the original curve~$E$, and the set of all
the rational points thus obtained will cover the cosets of $2E(\Q)$ in
$E(\Q)$; thus we will be able to determine the rank of $E(\Q)$, and at
the same time obtain a set of points which generates a subgroup of the
Mordell-Weil group $E(\Q)$ of odd, finite index.  Quartics with no
global rational point which are everywhere locally soluble arise from
non-trivial elements in the \TS\ group of~$E$ (or of $E'$); if these
exist, we will only obtain upper and lower bounds for the rank.  This
is because we currently have no general procedure for proving that a
quartic with no rational points does have none.  In practice,
moreover, it is often impossible to distinguish between such a quartic
and one with rational points which are all very large, and hence
outside the search region; this happens when a curve has some very
large generators, and in such cases also we may only be able to give
bounds for the rank.  Further work on these questions is clearly
needed, and is currently the focus of much active research.

Since the covering maps $H\to E$ have degree 2 or 4, the rational
points on $H$ tend to be smaller (in the sense of naive height) than
the rational points they map to on $E$; this makes them easier to find
by search.  Here is an example of this: the curve $y^2=x^3-673$ has
rank~2, with generators $P_1=(29,154)$ and $P_2=(33989323537/61761^2,
-1384230292401340/61761^3)$.  The second generator, which would take a
very long time to find by searching on the curve itself, is obtained
from the rational point $(x,y) = (191/97,123522/97^2)$ on the quartic
with coefficients $(a,b,c,d,e) = (-2,4,-24,164,-58)$.  This is much
easier to find: our program takes less than a second to find the rank
and both generators of this curve (but in this time it does not prove
that they are generators, only that they generate a subgroup of finite
odd index in the Mordell-Weil group).

Before we describe the two main two-descent algorithms, we will
present algorithms for determining local solubility and for
attempting to determine global solubility of a quartic equation such
as (\phseqn), as these are used in both the algorithms.

\subhead Checking local solubility \endsubhead

Here we present an algorithm for determining the local solubility of a
curve of the form (\phseqn), where $g(x)$ is a square-free quartic
polynomial with integer coefficients.  It is easy to generalize this
algorithm in two ways: firstly, one might be interested in polynomials
of higher degree (when studying curves of higher genus, for example);
secondly, working over a general number field $K$, one would replace
the $p$-adic field $\Q_p$ here with the appropriate completion of~$K$.
These extensions are quite straightforward.

Solubility at the infinite prime (that is, over the reals) is easily
determined.  If $g(x)$ has a real root then it certainly takes
positive values, so that $H$ has real points; if $g(x)$ has no real
roots, then the values of $g(x)$ have constant sign, and we merely
have to check that~$a>0$.

Regarding the finite primes, we first observe that there are only a
finite number which need checking in each case, for if $p$ is an odd
prime not dividing the discriminant of $g$, then $H$ certainly has
points modulo~$p$ which are nonsingular and hence lift to $p$-adic
points.  For the other primes, we present an algorithm first given in
\cite\BirchSD.  

It suffices to determine solubility in $\Z_p$ for either $g(x)$ or
$g^*(x)=ex^4+dx^3+cx^2+bx+a$, and in the latter case we may assume
$x\in p\Z_p$.  Given $x_k$ modulo $p^k$, one tries to lift to a
$p$-adic point $(x,y)$ with $x\equiv x_k\pmod{p^k}$.  In
\cite\BirchSD, conditions are given for this to be possible; more
precisely, one of three possibilities may occur (given $k$ and~$x_k$):
either a lifting is definitely possible, and we may terminate the
algorithm with a positive result; or it is definitely not possible,
and we reject this value of~$x_k$; or it is impossible to decide
without considering $x_k$ modulo a higher power of $p$.  The test for
this lifting is given below in the two subroutines called {\tt lemma6}
and {\tt lemma7}, named after the corresponding results in
\cite\BirchSD.  This leads to a recursive algorithm which is
guaranteed to terminate since in any given case there is an exponent
$k$ such that it is possible
\footnote{In fact, if $k>\ord_p(\disc(g))$, and also $k\ge2$ when
$p=2$, then the third possibility cannot occur in algorithms {\tt
lemma6} and {\tt lemma7}.} to determine $p$-adic solubility by
considering solubility modulo $p^k$.  All this is an exercise in
Hensel's Lemma; the prime $p=2$ needs to be considered separately.
For the details, we refer to the pseudocode below, or to
\cite\BirchSD.  Further information on local solubility may be found
in \cite\SSthesis\ and \cite\SSpaper.

Here is the pseudocode for these algorithms.  Note that for any given
elliptic curve, all the homogeneous spaces considered will have the
same discriminant as the curve (up to a power of~2), so that in
practice we would not need to factorize the discriminant of each
quartic.

\beginalg{Algorithm for determining local solubility of a quartic}
\+INPUT:   &&&a, b, c, d, e &&&(integer coefficients of a quartic g(x))\cr %
\+OUTPUT:  &&&TRUE/FALSE &&&(solubility of y\2=g(x) in $\R$ and in
$\Q_p$ for all $p$)\cr 
\smallskip
\lineno=0 
\nline BEGIN\cr
\nline IF NOT R\_soluble(a,b,c,d,e) THEN RETURN FALSE FI;\cr
\nline IF NOT Qp\_soluble(a,b,c,d,e,2) THEN RETURN FALSE FI;\cr
\nline $\Delta$ = discriminant(a,b,c,d,e);\cr
\nline p\_list = odd\_prime\_factors($\Delta$);\cr
\nline FOR p IN p\_list DO\cr
\nline BEGIN\cr
\nline &IF NOT Qp\_soluble(a,b,c,d,e,p) THEN RETURN FALSE FI\cr
\nline END;\cr
\nline RETURN TRUE\cr
\nline END\cr
%
\medskip
\comm{Subroutine for determining real solubility}
\smallskip\lineno=0 
%
\+SUBROUTINE R\_soluble(a,b,c,d,e)\cr
\+INPUT:   &&&a, b, c, d, e &&&(integer coefficients of a quartic g(x))\cr %
\+OUTPUT:  &&&TRUE/FALSE &&&(solubility of y\2=g(x) in $\R$)\cr 
\smallskip
%
\nline BEGIN\cr
\nline IF a>0 THEN RETURN TRUE FI;\cr
\nline x\_list = real\_roots(a*x\4+b*x\3+c*x\2+d*x+e=0);\cr
\nline IF length(x\_list)>0 THEN RETURN TRUE FI;\cr
\nline RETURN FALSE\cr
\nline END\cr
%
\medskip
\comm{Subroutine for determining $p$-adic solubility}
\smallskip\lineno=0 
%
\+SUBROUTINE Qp\_soluble(a,b,c,d,e,p)\cr
\+INPUT:   &&&a, b, c, d, e &&&(integer coefficients of a quartic g(x))\cr %
\+         &&&p &&&(a prime)\cr
\+OUTPUT:  &&&TRUE/FALSE &&&(solubility of y\2=g(x) in $\Q_p$)\cr 
\smallskip
%
\nline BEGIN\cr
\nline IF Zp\_soluble(a,b,c,d,e,0,p,0) THEN RETURN TRUE FI;\cr
\nline IF Zp\_soluble(e,d,c,b,a,0,p,1) THEN RETURN TRUE FI;\cr
\nline RETURN FALSE\cr
\nline END\cr
%
\medskip
\comm{Recursive $\Z_p$-solubility subroutine}
\smallskip
%
\+SUBROUTINE Zp\_soluble(a,b,c,d,e,x\_k,p,k)\cr
\+INPUT:   &&&a, b, c, d, e &&&(integer coefficients of a quartic g(x))\cr %
\+         &&&p &&&(a prime)\cr
\+         &&&x\_k &&&(an integer)\cr
\+         &&&k &&&(a non-negative integer)\cr
\+OUTPUT:  &&&TRUE/FALSE &&&(solubility of y\2=g(x) in $\Z_p$, with x$\equiv$x\_k (mod p$^k$))\cr
%\+         &&&&&&&\cr 
\smallskip
\lineno=0
\nline BEGIN\cr
\nline IF p=2 \cr
\nline THEN code = lemma7(a,b,c,d,e,x\_k,k) \cr
\nline ELSE code = lemma6(a,b,c,d,e,x\_k,p,k) \cr
\nline FI;\cr
\nline IF code=+1 THEN RETURN TRUE FI;\cr
\nline IF code=-1 THEN RETURN FALSE FI;\cr
\nline FOR t = 0 TO p-1 DO\cr
\nline BEGIN \cr
\nline &IF Zp\_soluble(a,b,c,d,e,x\_k+t*p$^k$,p,k+1) THEN RETURN TRUE FI\cr
\nline END;\cr
\nline RETURN FALSE\cr
\nline END\cr
%
\medskip
\comm{$\Z_p$ lifting subroutine: odd $p$}
\smallskip
\lineno=0
%
\+SUBROUTINE lemma6(a,b,c,d,e,x,p,n)\cr
\nline BEGIN\cr
\nline gx = a*x\4+b*x\3+c*x\2+d*x+e;\cr
\nline IF p\_adic\_square(gx,p) THEN RETURN +1 FI;\cr
\nline gdx = 4*a*x\3+3*b*x\2+2*c*x+d;\cr
\nline l = ord(p,gx); m = ord(p,gdx);\cr
\nline IF (l$\ge$m+n) AND (n>m) THEN RETURN +1 FI;\cr
\nline IF (l$\ge$2*n) AND (m$\ge$n) THEN RETURN 0 FI;\cr
\nline RETURN -1 \cr
\nline END \cr
%
\medskip
\comm{$\Z_2$ lifting subroutine}
\smallskip
\lineno=0
\+SUBROUTINE lemma7(a,b,c,d,e,x,n)\cr
\nline BEGIN\cr
\nline gx = a*x\4+b*x\3+c*x\2+d*x+e;\cr
\nline IF p\_adic\_square(gx,2) THEN RETURN +1 FI;\cr
\nline gdx = 4*a*x\3+3*b*x\2+2*c*x+d;\cr
\nline l = ord(p,gx); m = ord(p,gdx);\cr
\nline gxodd = gx; WHILE even(gxodd) DO gxodd = gxodd/2; \cr
\nline gxodd = gxodd (mod 4);\cr
\nline IF (l$\ge$m+n) AND (n>m) THEN RETURN +1 FI;\cr
\nline IF (n>m) AND (l=m+n-1) AND even(l) THEN RETURN +1 FI;\cr
\nline IF (n>m) AND (l=m+n-2) AND (gxodd=1) AND even(l) THEN RETURN +1 FI;\cr
\nline IF (m$\ge$n) AND (l$\ge$2*n) THEN RETURN 0 FI;\cr
\nline IF (m$\ge$n) AND (l=2*n-2) AND (gxodd=1) THEN RETURN 0 FI;\cr
\nline RETURN -1\cr
\nline END\cr
\endalg

A few further remarks on these algorithms: firstly, only trivial
changes need to be made to the algorithms {\tt Qp\_soluble} and {\tt
Zp\_soluble} to make them apply to more general equations of the form
$y^2=g(x)$ where $g(x)$ is a non-constant squarefree integer
polynomial.  This is relevant for work on curves of higher genus, and
was observed by S.~Siksek.  Secondly, extensions to more general
$p$-adic fields are also useful in studying curves over number fields,
and again the extensions of Lemma~6 and Lemma~7 in
\cite\BirchSD\ are not difficult.  See the theses \cite\SSthesis\ and
\cite\PSthesis\ for details of such extensions.

Lastly, D.~Simon observed that in our application of the algorithms
{\tt lemma6} and {\tt lemma7}, we only care whether there is a
solution, not necessarily that there is a solution congruent to the
given $x\pmod{p^k}$; hence line~6 of subroutine {\tt lemma6} and
line~8 of subroutine {\tt lemma7} can both be replaced by:

{\tt
\+&IF l>2*m THEN RETURN +1 FI.\cr
}


\subhead Checking global solubility \endsubhead

To determine whether an equation (\phseqn) has a rational point is
much harder than the corresponding local question.  All we can do at
present is search (efficiently) for a point up to a certain height,
after checking that there is no local obstruction.  The only
satisfactory way known at present to decide on the existence of
rational points on these homogeneous spaces is to carry out so-called
higher descents; as mentioned above, this is the subject of current
work (see \cite\Nigeletal, for example), and we will not consider it
further here.

Our strategy is to look first for a small rational point, using a very
simple procedure with low overheads; if this fails, we check for local
solubility; if this passes, we start a much more thorough search for a
global point, using a quadratic sieving procedure rather similar to
the one described in the previous section for finding points on the
elliptic curve itself.  (In fact, such a sieve-assisted search may be
used to find rational points on any curve given by an equation of the
form $y^2=g(x)$ where $g(x)$ is a polynomial in~$x$.) The philosophy
here is that there is no point in looking hard for rational points
unless one is sure of local solubility, but also that there is no
point in checking local solubility when there is an obvious global
point.

%\def\legendre#1#2{\left(\dfrac{#1}{#2}\right)}
\def\legendre#1#2{(\frac{#1}{#2})}
To carry out the sieve-assisted search, for each possible denominator
of~$x$ one precomputes, for each of several sieving moduli $m$, the
residues to which the numerator of $x$ must belong if the right-hand
side of the equation is to be a square modulo~$m$.  In addition, it is
easy to see that for every odd prime~$p$ dividing the denominator of
the $x$-coordinate of a rational point, we must have
$\legendre{a}{p}=+1$; so provided that the leading coefficient~$a$ is
not a square (in which case the points at infinity are rational
anyway), we precompute a list of primes~$p$ for which
$\legendre{a}{p}=-1$, and discard possible denominators divisible by
any of these primes.  For $p=2$ a similar condition holds.
\footnote{I am grateful to J. Gebel for this idea, which saves
considerable time in practice.}  One also obviously restricts the
search to ranges of~$x$ for which $g(x)$ is positive; depending on the
number of real roots of~$g$ and the sign of~$a$, this splits the
search into up to three intervals.  Finally, in the case of
two-descent via $2$-isogeny, where the quartics are polynomials in
$x^2$ and thus even, we may restrict to positive~$x$.

For reasons of space, we will only give here the code for a simple
point search with no sieving.  

\beginalg{Algorithm for searching for a rational point on a quartic:
simple version}
\+INPUT:   &&&a, b, c, d, e &&&(integer coefficients of a quartic g(x))\cr
\+         &&&k1, k2 &&&(lower and upper bounds)\cr
\+OUTPUT:  &&&TRUE/FALSE &&&(solubility of y\2=g(x) in $\Q$ with x=u/w\cr
\+         &&&           &&& and k1${}\le{}$|u|+w${}\le{}$k2)\cr 
\smallskip\lineno=0
%
\nline BEGIN\cr
\nline FOR n = k1 TO k2 DO\cr
\nline BEGIN\cr
\nline &IF n=1 THEN\cr
\nline &&IF square(a) RETURN TRUE FI;\cr
\nline &&IF square(e) RETURN TRUE FI\cr
\nline &ELSE\cr
\nline &&FOR u = 1 TO n-1 DO\cr
\nline &&BEGIN\cr
\nline &&&IF gcd(u,n)=1 \cr
\nline &&&THEN\cr
\nline &&&&w = n-u;\cr
\nline &&&&IF square(a*u\4+b*u\3w+c*u\2w\2+d*uw\3+e*w\4) RETURN TRUE FI;\cr
\nline &&&&IF square(a*u\4-b*u\3w+c*u\2w\2-d*uw\3+e*w\4) RETURN TRUE FI\cr
\nline &&&FI\cr
\nline &&END\cr
\nline &FI\cr
\nline END; \cr
\nline RETURN FALSE\cr
\nline END\cr
%
\endalg
\bigskip

We will now describe the two main two-descent algorithms: two-descent
via $2$-isogeny for use when $E$ has a rational point of order~2, and
general two-descent in the general case.  We only use general
two-descent when there is no point of order~2, so that the first method
does not apply.  The situation is not appreciably simpler when $E$ has
all three of its points of order two rational than when there is just
one rational point of order two, and so we will not bother to consider
this case separately.

\subhead Method 1: descent using $2$-isogeny \endsubhead

Suppose that $E$ has a rational point $P$ of order~2.  By a change of
coordinates we may assume that $E$ has equation
$$
 E: \quad y^2=x(x^2+cx+d)
$$
where $P=(0,0)$, and $c,d\in\Z$.  Explicitly, in terms of a \W\
equation, let $x_0$ be a root of the cubic $x^3+b_2x^2+8b_4x+16b_6$,
and set $c=3x_0+b_2$, $d=(c+b_2)x_0+8b_4$.  If $a_1=a_3=0$, then we
can avoid a scaling factor of~2 by letting $x_0$ be a root of
$x^3+a_2x^2+a_4x+a_6$, and setting $c=3x_0+a_2$, $d=(c+a_2)x_0+a_4$.
The $2$-isogenous curve $E' = E/\left<P\right>$ has equation
$$
   E': \quad y^2=x(x^2+c'x+d')
$$
where 
$$
   c'=-2c  \qquad\text{and}\qquad d'=c^2-4d.
$$
The nonsingularity condition on $E$ is equivalent to $dd'\not=0$.  The
$2$-isogeny $\phi\colon E\to E'$ has kernel $\{0,P\}$ and in general
maps $(x,y)$ to $\left(\frac{y^2}{x^2},\frac{y(x^2-d)}{x^2}\right)$.
The dual isogeny $\phi'\colon E'\to E$ maps $(x,y)$ to
$\left(\frac{y^2}{4x^2},\frac{y(x^2-d')}{8x^2}\right)$.

For each factorization $d=d_1d_2$, with $d_1$ square-free, we consider
the homogeneous space
$$
   H(d_1,c,d_2): \quad  v^2=d_1u^4+cu^2+d_2.
$$
Let $n_1=n_1(c,d)$ be the number of factorizations of~$d$ for which
the quartic $H(d_1,c,d_2)$ has a rational point, and $n_2=n_2(c,d)$
the number for which the quartic has a point everywhere locally.
Define $n_1'=n_1(c',d')$ and $n_2'=n_2(c',d')$ similarly.  Then it is
not hard to show by rather explicit calculation (see below and the
references given) that $E(\Q)/{\phi'}(E'(\Q))$ is isomorphic to the
subgroup of $\Q^*/(\Q^*)^2$ generated by the factors
$d_1$ for which $H(d_1,c,d_2)$ has a rational point.  Thus
$$
     \left|E(\Q)/{\phi'}(E'(\Q))\right| = n_1,
$$
which must therefore be a power of 2, say $n_1=2^{e_1}$; similarly,
$$
     \left|E'(\Q)/\phi(E(\Q))\right| = n_1' = 2^{e_1'}.
$$
It then follows (see below) that\neweq{\rankeqn}
$$
   \rk(E(\Q))=\rk(E'(\Q))= e_1 + e_1' - 2. \tag\rankeqn
$$

With luck one will find rational points on all the quartics which have
them everywhere locally; then $n_1=n_2$, and there is no ambiguity in
the result.  However there will be cases in which the number
$\tilde{n}_1$ of quartics on which we can find a rational point is
strictly less than~$n_2$.  In such cases, we will only have upper and
lower bounds for $n_1$, and similarly for $n'_1$, leading to upper and
lower bounds for the rank.  This can happen for two reasons: either
there is a rational point on some quartic, but our search bound was
too small to find it; or the quartic has points everywhere locally but
no global rational point.

The quartics $H$ which have points everywhere locally but not globally
come from elements of order~2 in the \TS\ groups $\Sha(E/\Q)$ and
$\Sha(E'/\Q)$.  There is an exact sequence
$$
   0 \to E(\Q)/\phi'(E'(\Q)) \to S^{(\phi')}(E'/\Q) \to \Sha(E'/\Q)[\phi']\to 0
$$
coming from Galois cohomology; here $S^{(\phi')}(E'/\Q)$ is the Selmer
group of order $n_2$ whose elements are represented by the homogeneous
spaces $H(d_1,c,d_2)$ which are everywhere locally soluble, and
$\Sha(E'/\Q)$ is the \TS\ group of $E'$. The injective map
$E(\Q)/\phi'(E'(\Q)) \to S^{(\phi')}(E'/\Q)$ is induced by taking a
point $(x,y)$ in $E(\Q)$ with $x\not=0$ to the space $H(d_1,c,d_2)$
where $d_1=x$ modulo squares:  if $x=d_1u^2$ and $v=uy/x$ then
$(u,v)$ is a rational point on $H(d_1,c,d_2)$.  The point $P=(0,0)$
maps to $d$ modulo squares.  Conversely, if $(u,v)$ is a rational
point on $H(d_1,c,d_2)$ then $(x,y)=(d_1u^2,d_1uv)$ is a rational
point on $E$.  (In proving these statements, one has to check that two
rational points on $E$ have the same $x$-coordinate modulo squares if
and only if their difference is in $\phi'(E'(\Q))$; for example, the
image of~$P$ is~$d$, which is a square if and only if
$P\in\phi'(E'(\Q))$.)  It follows that $n_1$ is the order of
$E(\Q)/\phi'(E'(\Q))$, as stated above, and hence that
$$
   \left|\Sha(E'/\Q)[\phi']\right| = n_2/n_1.
$$
Similarly, from the exact sequence
$$
   0 \to E'(\Q)/\phi(E(\Q)) \to S^{(\phi)}(E/\Q) \to \Sha(E/\Q)[\phi]\to 0
$$
with similarly defined maps, we obtain
$$
   \left|\Sha(E/\Q)[\phi]\right| = n'_2/n'_1.
$$

Thus the result is only genuinely ambiguous when either
$\Sha(E/\Q)[\phi]$ or $\Sha(E'/\Q)[\phi']$ is non-trivial, so that not
all elements of the Selmer groups are obtained from rational points on
the elliptic curves.  This is rare for the curves in the tables, but
obviously must be taken into account in general.  A typical situation
is to have $n_2n_2'=16$ and $n_1n_1'\ge4$, when one suspects that
$r=0$ with $\left|\Sha(E/\Q)[2]\right|=4$ or
$\left|\Sha(E'/\Q)[2]\right|=4$, but where it is possible instead that
$r=2$ and $\left|\Sha(E/\Q)[2]\right|=\left|\Sha(E'/\Q)[2]\right|=1$.
Curve 960D1 in the tables is an example of this, although in this case
since the curve is modular and we know that $L(E,1)\not=0$, it must
have rank~0 by the result of Kolyvagin mentioned earlier.  We can also
deduce this by working with the $2$-isogenous curves 960D3 and 960D2,
where there is no ambiguity: here $n_1=n_2=n'_1=n'_2=2$, showing that
the rank is certainly~0.  (Note that isogenous curves have the same
rank, but not necessarily the same order of $\Sha$, which can work to
our advantage in cases like this.)  Returning to the pair 960D1-960D2
where we compute $n_1=n_2=1$, $n'_2=16$ and $n'_1\ge4$, now we know
that the rank is in fact zero we can conclude that $n'_1=4$, and that
$\left|\Sha(E/\Q)[\phi]\right|=4$.  The nontriviality of $\Sha(E/\Q)$
in this case is confirmed by the \BSD\ conjecture, which for this
curve predicts that $\Sha$ has order~4 (see Table~4).

Local solubility of $H(d_1,c,d_2)$ is automatic for all primes $p$
which do not divide $2dd'$; for those $p$ which do divide $2dd'$ we
may apply the general criteria of Birch and Swinnerton-Dyer. Local
solubility in $\R$ is easy to determine here: if $d'<0$ then we
require $d_1>0$, while if $d'>0$ then either $d_1>0$ or
$c+\sqrt{d'}>0$ is necessary.  Thus if either $d'<0$, or $d'>0$ and
$c+\sqrt{d'}<0$, then we only consider positive divisors~$d_1$ of~$d$,
and need not apply the general test for solubility in~$\R$.

Each rational point $(u,v)$ on $H(d_1,c,d_2)$ maps, as observed above,
to the point $(d_1u^2,d_1uv)$ on $E$; modulo $\phi'(E'(\Q))$, this is
independent of the rational point $(u,v)$, and only depends on $d_1$
modulo squares.  Similarly, a rational point $(u,v)$ on
$H(d_1',c',d_2')$ maps to a point on $E'$, and hence via the dual
isogeny $\phi'$ to the point
$$
  \left(\frac{v^2}{4u^2} , \frac{v(d_1'u^4-d_2')}{8u^3}\right)
$$
in $E(\Q)$.  The set of $n_1n_1'$ points in $E(\Q)$ thus determined
(by adding the points constructed in this way) cover the cosets of
$E(\Q)/2E(\Q)$, either once each, when $|E(\Q)[2]|=4$, which is when
$d'$ is a square, or twice, when $d'$ is not a square and
$|E(\Q)[2]|=2$.  Thus, when $|E(\Q)[2]|=2$ we have
$$ 
  \frac{n_1n_1'}{2} = |E(\Q)/2E(\Q)| = 2^{r+1},
$$
while if $|E(\Q)[2]|=4$ we have
$$
   n_1n_1' = |E(\Q)/2E(\Q)| = 2^{r+2};
$$
hence $2^r=n_1n_1'/4$ in both cases, proving (\rankeqn).

When counting $n_1$ and~$n_2$ (and similarly, $n'_1$ and~$n'_2$), it
is very useful to use the fact that each is a power of~2, being the
order of an elementary abelian $2$-group.  This is particularly
important when $d$ (or $d'$) has many distinct prime factors.  Let
$A_0$ be the group of all divisors of $d$ modulo squares, of order
$n_0$ (say).  Then $A_0$ is generated by $-1$ and the primes dividing
$d$, so that $n_0=2^{e_0}$ where $e_0$ is the number of distinct
prime factors of~$d$, plus~1.  Within $A_0$ we must determine the
subgroups $A_1$ and $A_2$ of orders $n_1$ and $n_2$, consisting of
those divisors $d_1$ of~$d$ for which the corresponding homogeneous
space is everywhere locally or globally soluble, respectively.

We can effectively reduce the size of the set $A_0$ of divisors to be
searched by a factor up to~$8$ as follows: as observed above, if
either $d'<0$, or $d'>0$ and $c+\sqrt{d'}<0$, then we need only
consider positive divisors~$d_1$ of~$d$, cutting in two the number of
elements of $A_0$ which may lie in $A_1$.  Secondly, we may take
advantage of the fact that we know the rational point $(0,0)$ on $E$;
thus we know that $d$ is in $A_2$ (though possibly just the identity
if $d$ is a square); similarly, if $d'$ is a square then $x^2+cx+d$
factorizes, say as $(x-x_2)(x-x_3)$, and we know that $x_2$ and~$x_3$
also lie in $A_2$.

More generally, whenever we find in the course of our systematic
search through the elements of $A_0$ that the element $d_1$ lies in
$A_2$, we can effectively factor out $d_1$ and reduce the number of
remaining values to check by a factor of~2.  Of course, this requires
careful book-keeping in the implementation; for simplicity, we omit
these refinements from the pseudocode below, where we simply loop over
all square-free divisors of $d$ and~$d'$.

As an example of the saving that can be made, consider the curve 
%$$   [0,36861504658225, 0, 1807580157674409809510400, 0]$$ 
of rank~13 constructed by Fermigier in \cite\Ferm; this is of the form
$y^2=x(x^2+cx+d)$ with
$$
\align
c &= 36861504658225 \qquad\text{and}\\
d &= 1807580157674409809510400 = 2^{15}\cdot3^4\cdot5^2\cdot7^2\cdot17\cdot23\cdot29\cdot41\cdot103\cdot113\cdot127\cdot809,\\
\endalign
$$
so that $d$ has~12 distinct prime factors and $2^{13}=8192$
square-free divisors.  Since $d$ is non-square we can cut the set in
half, say by excluding all $d_1$ divisible by the largest prime
factor~$809$, leaving 4096 values to test.  In our implementation, the
results of the test are as follows:
\item{$\bullet$}7 non-trivial values of $d_1$ give rational points after
searching, as well as $d_1=1$ which gives the trivial point;
\item{$\bullet$}120 further values are in the subgroup $A_2$ generated by these 7
values and need not be tested;
\item{$\bullet$}122 further values were tested and found to be not everywhere locally
soluble, hence not in~$A_1$;
\item{$\bullet$}3846 further values were discarded as being a product of an
element of $A_2$ and an element not in $A_1$, and hence not in~$A_1$.

\noindent
Thus in this case we find that $n_1=n_2=256$, after only having to
search for points on seven homogeneous spaces.  Working with the
isogenous curve, we obtain $n'_1=n'_2=128$ after only searching six
homogeneous spaces for points.  Thus $e_1=8$, $e'_1=7$ and the rank
is~13. Note that in the course of computing this value, we have
searched precisely 13 homogeneous spaces, and the points we thereby
construct give 13 generators of $E(\Q)/2E(\Q)$ modulo torsion.
Adding $P=(0,0)$ to this list gives 14 points which generate
$E(\Q)/2E(\Q)$ (which has order $2^{14}$), and which therefore
generate a subgroup of finite odd index in the full Mordell-Weil group
$E(\Q)$.

The situation is not always this simple, however, even for curves
where $\Sha[2]$ is trivial, since there may be homogeneous spaces with
rational points which are hard to find.  For example, consider
Fermigier's curve
%$$ [0,2429469980725060,0,275130703388172136833647756388,0]$$ 
of rank~14 from \cite\Ferm, with $c=2429469980725060$ and
$d=275130703388172136833647756388$ (which has 14 prime factors).  When
we run our program using a (logarithmic) bound of~10 in the search for
rational points on the quartics, we find $n_1\ge64$, $n'_1\ge128$,
while $n_2=n'_2=256$.  Here the correct values are $n_1=n'_1=256$,
giving $r=14$, but we only find $11\le r\le14$; and in the process, we
have had to search many more homogeneous spaces for rational points.

Here is the pseudo-code which implements the algorithm just described.
The main routine aborts if either the input curve is singular (this is
useful if one wants to apply the algorithm systematically to a range
of inputs) or if there is no point of order two.  The latter is
detected in lines~6--7, where an integer root to a monic cubic with
integer coefficients is found (if it exists).  Most of the work is
done in the subroutine {\tt count(c,d,p\_list)} which determines
$n_2(c,d)$ and, as far as possible, $n_1(c,d)$.  Here {\tt p\_list} is
the set of `bad' primes dividing $2dd'$ where local solubility needs
to be checked, which we only compute once.  There are two
calls to the subroutine {\tt rational\_point(a,b,c,d,e,k1,k2)}, which
seeks a rational $u/w$ with $k_1\le |u|+w \le k_2$ such that $g(u/w)$
is a rational square, where $g(x)=ax^4+bx^3+cx^2+dx+e$.  (Here $w>0$
and $\gcd(u,w)=1$.)  In the first call we carry out a quick check for
`small' points; then we look further, having first checked for
everywhere local solubility.  The particular parameters {\tt lim1},
{\tt lim2} for the search will probably be decided at run time. The
subroutines {\tt Qp\_soluble} and {\tt rational\_point} are
implementations of the algorithms given earlier (though in practice we
would use a more efficient algorithm for the second call to {\tt
rational\_point}, as explained above).

\beginalg{Algorithm for computing rank: rational $2$-torsion case}
\+INPUT:   &&&a1, a2, a3, a4, a6 &&&&(coefficients of $E$)\cr %
\+OUTPUT:  &&&r\_min, r\_max &&&&(bounds for rank of $E$)\cr
\+         &&&S,S' &&&&(upper bounds for $\#\Sha(E)[\phi]$ and $\#\Sha(E')[\phi']$)\cr
\smallskip \lineno=0
%
\nline BEGIN\cr
\nline IF a1=a3=0 \cr
\nline THEN s2 = a2; s4 = a4; s6 = a6\cr
\nline ELSE s2 = a1*a1+4*a2; s4 = 8*(a1*a3+2*a4); s6 = 16*(a3*a3+4*a6)\cr
\nline FI; \cr
\nline x\_list = integer\_roots(x\3+s2*x\2+s4*x+s6=0); \cr
\nline IF length(x\_list)=0 THEN abort ELSE x0 = x\_list[1] FI; \cr
\nline c = 3*x0+s2; d = (c+s2)*x0 + s4; \cr
\nline c' = -2*c; d' = c\2-4*d; \cr
\nline IF d*d'=0 THEN abort FI; \cr
\nline p\_list = prime\_divisors(2*d*d'); \cr
\nline (n1,n2) = count(c,d,p\_list); \cr
\nline (n1',n2') = count(c',d',p\_list); \cr
\nline e1 = log\_2(n1); e2 = log\_2(n2); \cr
\nline e1' = log\_2(n1'); e2' = log\_2(n2'); \cr
\nline r\_min = e1+e1'-2; r\_max = e2+e2'-2; \cr
\nline S = n2'/n1'; S' = n2/n1; \cr
\nline RETURN r\_min, r\_max, S, S' \cr
\nline END\cr
%
\medskip
\comm{Main counting subroutine}
\smallskip\lineno=0
%
\+ SUBROUTINE count(c,d,p\_list)\cr
\nline BEGIN\cr
\nline n1 = n2 = 1; d' = c\2-4*d; \cr
\nline d1\_list = squarefree\_divisors(d);\cr
\nline FOR d1 IN d1\_list DO\cr
\nline BEGIN \cr
\nline &IF rational\_point(d1,0,c,0,d/d1,1,lim1) \cr
\nline &THEN n1 = n1+1; n2 = n2+1 \cr
\nline &ELSE\cr
\nline &&IF everywhere\_locally\_soluble(c,d,d',d1,p\_list) \cr
\nline &&THEN \cr
\nline &&&n2 = n2+1; \cr
\nline &&&IF rational\_point(d1,0,c,0,d/d1,lim1+1,lim2)\cr
\nline &&&THEN n1 = n1+1 \cr
\nline &&&FI\cr
\nline &&FI \cr
\nline &FI\cr
\nline END;\cr
\nline RETURN (n1, n2)  \cr
\nline END\cr
%
\medskip
\comm{Subroutine to check for everywhere local solubility}
\smallskip\lineno=0
%
\nline SUBROUTINE everywhere\_locally\_soluble(c,d,d',d1,p\_list)\cr
\nline BEGIN\cr
\nline IF d'<0 AND d1<0 THEN RETURN FALSE FI;\cr
\nline IF d'>0 AND d1<0 AND (c+sqrt(d'))<0 THEN RETURN FALSE FI;\cr
\nline FOR p IN p\_list DO\cr
\nline BEGIN\cr
\nline &IF NOT Qp\_soluble(d1,0,c,0,d/d1,p) THEN RETURN FALSE FI\cr
\nline END; \cr
\nline RETURN TRUE\cr
\nline END\cr
\endalg

\subhead Method 2: general two-descent \endsubhead

We now turn to the general two-descent, which applies whether or not
$E$ has a rational point of order~2.  Again, the basic idea is to
associate to $E$ a collection of $2$-covering quartic curves (or
homogeneous spaces) $H$.  These have equations of the form
$$
   H: \qquad   y^2 = g(x) = ax^4+bx^3+cx^2+dx+e \tag\phseqn 
$$
with $a,b,c,d,e\in\Q$, such that the \und{invariants}
$$
  I =  12ae-3bd+c^2 \qquad\text{and}\qquad J= 72ace+9bcd-27ad^2-27eb^2-2c^3
$$
are related to the $c_4$ and $c_6$ invariants of $E$ via
$$
   I=\lambda^4c_4 \qquad\text{and}\qquad J=2\lambda^6c_6
$$
for some $\lambda\in\Q^*$.  Two such quartics $g_1(x)$, $g_2(x)$ are
\und{equivalent} if
$$
   g_2(x) = \mu^2 (\gamma x+\delta)^4 g_1\kern-3pt\left(\frac{\alpha
                                 x+\beta}{\gamma x+\delta}\right) 
$$
for some $\alpha$, $\beta$, $\gamma$, $\delta$ and $\mu\in\Q$, with $\mu$  and
$\alpha\delta-\beta\gamma$ non-zero.    The invariants of $g_1(x)$ and $g_2(x)$
are then related by the scaling factor 
$\lambda=\mu(\alpha\delta-\beta\gamma)$: 
$$\align
    I(g_2)&=\mu^4(\alpha\delta-\beta\gamma)^4 I(g_1), \cr
    J(g_2)&=\mu^6(\alpha\delta-\beta\gamma)^6 J(g_1). \cr
  \endalign
$$ 
We set $\Delta=4I^3-J^2=27\disc(g)$, and call $\Delta$ the
discriminant.

In particular, by scaling up the coefficients, we may assume that the
invariants $I$ and $J$ are integral. The number of equivalence classes
of quartics with given invariants (up to a scaling factor
$\lambda$) which are everywhere locally soluble is finite.  One of our
tasks will be to determine, for a given integral quartic, an
equivalent integral one with minimal invariants.  This process is
closely analogous to the one considered earlier in this chapter, using
Kraus's conditions or Tate's algorithm to determine minimal models for
elliptic curves.  Indeed, we will see below that if $c_4$ and~$c_6$
are invariants of a minimal model for the elliptic curve~$E$, then
$I=c_4$ and~$J=2c_6$ are also minimal, except possibly at the
prime~$2$.  (We may lose minimality at~$2$ because the equations
(\phseqn) we use for homogeneous spaces are not completely general,
not having terms in $y$, $xy$ or~$x^2y$; to remove these by completing
the square involves a scaling by a factor of~$2$.)

\def\EIJ{E_{I,J}}
\def\EIJQ{E_{I,J}(\Q)}
We now explain the relationship between equivalence classes of soluble
quartics with invariants $I$ and~$J$ and rational points on the
elliptic curve.  More details of this relationship, including proofs,
may be found in \cite\JCinvariants.  For convenience, we again start
by making a coordinate transformation: if $c_4$ and~$c_6$ are the
integral invariants of our curve~$E$, we set $I=c_4$ and $J=2c_6$, and
replace $E$ by the isomorphic curve
\neweq\EIJdef
$$
   \EIJ:\quad Y^2 = F(X) = X^3 -27IX-27J. \tag\EIJdef
$$
This is the model on which the rational points we construct will
naturally lie; it is then a simple matter to transfer them back to the
original model for~$E$.  For simplicity, we will still continue to
refer to the curve simply as $E$ when this will not cause confusion.

Associated to each quartic $g$ there are two so-called
\und{covariants}, which we denote $g_4$ and~$g_6$:\neweq\gfoursixdef
$$\aligned
  g_4(X,Y) &= (3b^2 - 8ac) X^4 
+ 4 (bc - 6ad)             X^3Y 
+ 2 (2c^2 - 24ae - 3bd)    X^2Y^2 \\
&\quad+ 4 (cd - 6be)            X  Y^3 
+   (3d^2-8ce)                Y^4,\\
%
  g_6(X,Y) &= (b^3 + 8a^2d - 4abc)   X^6 
+ 2 (16a^2e + 2abd - 4ac^2 + b^2c)  X^5 Y \\
&\quad+ 5 (8abe + b^2d - 4acd)            X^4 Y^2
+ 20(b^2e - ad^2)                   X^3 Y^3 \\
&\quad- 5 (8ade + bd^2 - 4bce)            X^2 Y^4 
- 2 (16ae^2 + 2bde - 4c^2e + cd^2 ) X   Y^5 \\
&\quad-   (d^3 + 8be^2 - 4cde)                Y^6.
\endaligned\tag\gfoursixdef
$$
Wherever convenient, we will also denote by $g$ the homogenized
polynomial $g(X,Y)=aX^4+bX^3Y+cX^2Y^2+dXY^3+eY^4$.  These three
homogeneous polynomials satisfy an algebraic identity, or
syzygy:\neweq\covarsyz
$$
   27g_6^2 = g_4^3 - 48Ig^2g_4 - 64Jg^3. \tag\covarsyz
$$
Later we will also need a simpler form of this syzygy; set\neweq\semidef
$$
   p=g_4(1,0)=3b^2-8ac \qquad\text{and}\qquad
   r=g_6(1,0)=b^3 + 8a^2d - 4abc;               \tag\semidef
$$
these quantities are called \und{seminvariants} of~$g$.  Substituting
$(X,Y)=(1,0)$ in the covariant syzygy (\covarsyz) gives an identity
(the seminvariant syzygy) between these seminvariants:\neweq\semisyz
$$
  27r^2 = p^3 - 48Ia^2p - 64Ja^3. \tag\semisyz
$$
We will make use of this equation in our search for quartics with
given invariants, where it will allow us to set up a quadratic sieve.

It follows from the covariant syzygy (\covarsyz), by simple
substitution, that the map\neweq\ximap
$$
  \xi:\quad   \left({x},{y}\right)
      \mapsto \left(\frac{3g_4(x,1)}{(2y)^2},\frac{27g_6(x,1)}{(2y)^3}\right)
\tag\ximap
$$
maps rational points $(x,y)$ on $H$ (satisfying $y^2=g(x,1)$) to
rational points on~$\EIJ$, thus defining a rational map~$\xi$, of
degree~4, from $H(\Q)$ to~$\EIJQ$.  We are using affine coordinates
here; the points at infinity on $H$ map to
$\left(\dfrac{3p}{4a},\dfrac{\pm27r}{(4a)^{3/2}}\right)$, which are
rational if and only if $a$ is a square.

We now have the following facts (see \cite\JCinvariants\ for details):

$\bullet$ If $R\in H(\Q)$ with $P=\xi(R)\in\EIJQ$, then the coset of
$P$ modulo $2\EIJQ$ is independent of~$R$, and of the particular
quartic~$g$ up to equivalence; in fact, equivalences between quartics
induce rational maps between the associated homogeneous spaces, and
the covariant property of $g_4$ and $g_6$ ensures that corresponding
rational points on the homogeneous spaces have the same image in
$\EIJQ$.

$\bullet$ Each rational point $P=(x,y)\in\EIJQ$ arises as the image of a
rational point on some quartic~$g$ with invariants $I$ and~$J$:
explicitly, one can take the rational point at infinity on the quartic
with coefficients $(a,b,c,d,e)=(1,0,-x/6,y/27,I/12-x^2/432)$;  the
equivalence class of~$g$ depends only on the coset of $P$
modulo~$2\EIJQ$. 

$\bullet$ The equivalence classes of everywhere locally soluble
quartics with invariants $I$ and~$J$ form a finite elementary abelian
$2$-group, isomorphic to the $2$-Selmer group $S^{(2)}(E/\Q)$.

$\bullet$ The equivalence classes of soluble quartics with invariants
$I$ and~$J$ form a finite elementary abelian $2$-group isomorphic to
$E(\Q)/2E(\Q)$; the identity is the \und{trivial} class, consisting of
quartics with a rational root.

$\bullet$ More generally, when $E$ has no $2$-torsion, for any
extension field~$K$ of~$\Q$ there is a bijection between the roots of
$g(x)$ in $K$ and the solutions $Q\in\EIJ(K)$ to the equation $2Q=P$
(where $P=\xi(R)$ for $R\in H(\Q)$ as above).  In particular,
non-trivial quartics are irreducible in this case.  We will use this
fact with $K=\R$ later.

We therefore classify the set of equivalence classes of quartics with
invariants $I$ and~$J$ as follows:
\roster
\item[0] the trivial class consists of those quartics $g(x)$ which
have a rational root. These are elliptic curves isomorphic to $E$ over
$\Q$.
\item those which have a rational point: these are also elliptic curves, 
isomorphic to~$E$ over~$\Q$.
\item those which have points everywhere locally.
\item those which fail to have points everywhere locally.
\endroster
Let the number of inequivalent quartics in the first three sets be
$n_0=1$, $n_1$ and $n_2$.  (Those in the last set will not be used.)
Because of the group structure, each of these numbers is a power
of~$2$.  We write $n_i=2^{e_i}$ for $i=1,2$.

As in the case of descent via $2$-isogeny, Galois cohomology gives an
exact sequence
$$
   0 \to E(\Q)/2E(\Q) \to S^{(2)}(E/\Q) \to \Sha(E/\Q)[2] \to 0.
$$
Thus the quotient of $S^{(2)}(E/\Q)$ by the image of $E(\Q)$ is
isomorphic to $\Sha(E/\Q)[2]$, the $2$-torsion subgroup of the \TS\
group $\Sha(E/\Q)$.  So it is the points of order~2 in $\Sha(E/\Q)$,
if any, which account for the possible existence of homogeneous spaces
which have points everywhere locally but not globally, and we have
$$
   \left|\Sha(E/\Q)[2]\right| = n_2/n_1.
$$
As before, the potential practical difficulty lies in determining
whether each homogeneous space~$H$ has a rational point, as there is
no known algorithm to do this in general.  Again, for the vast
majority of the curves in the tables, we found a rational
point easily on each space which was everywhere locally soluble, which
not only determined the rank of $E$, but also implied that the \TS\
group had no $2$-torsion.  The only example with $n_1<n_2$ in the
tables (for a curve with no $2$-torsion) is curve 571A1, where $n_1=1$
and $n_2=4$; here the rank is~$0$, and $\left|\Sha(E/\Q)[2]\right|=4$;
the \BSD\ conjecture predicts $\left|\Sha(E/\Q)\right|=4$.

The steps of the algorithm are as follows: first we determine the pair
or pairs of integral invariants $(I,J)$ such that every quartic
associated with our curve $E$ is equivalent to one with integer
coefficients and these invariants.  There will be either one or two
such pairs.  For each pair $(I,J)$, we find a finite set of quartics
with invariants $(I,J)$ such that every non-trivial, everywhere
locally soluble quartic with these invariants is equivalent to one in
the list.  This is the most time-consuming step, as the search region
can be very large when $I$ and $J$ are large.  Now we must test the
quartics in our list pairwise for equivalence, discarding those
equivalent to earlier ones; look for rational points; and test
everywhere local solubility.  Again, there may be quartics where we do
not find rational points despite their having points everywhere
locally, so that although we can always (given enough time) determine
$n_2$, we may in some cases only find bounds on $n_1$.  Since
$n_1=\left|E(\Q)/2E(\Q)\right|$, we can then compute the rank $r$, or
bounds on the rank.  Usually, $E$ will have no rational $2$-torsion,
or we would probably be using descent via $2$-isogeny, and then
simply $2^r=n_1$.

We now consider each of these steps in more detail.

\subhead Step 1: Determining the invariants $(I,J)$ \endsubhead

Given an integral quartic~$g$ with invariants $I$ and~$J$, we must
consider the question of whether there exists an equivalent integral
quartic with smaller invariants.  The smaller invariants will have the
form $\lambda^{-4}I$, $\lambda^{-6}J$ with $\lambda\in\Q^*$.  In
\cite{\BirchSD, Lemmas 3--5}, conditions are stated under which $g$ is
equivalent to an integral quartic with invariants $p^{-4}I$, $p^{-6}J$
for a prime $p$; we call such a quartic \und{$p$-reducible}, otherwise
\und{$p$-minimal}.  Clearly a necessary condition for reducibility is
that $p^4\div I$ and $p^6\div J$.  We say that the pair $(I,J)$ is
\und{$p$-reducible} if every integral quartic with these invariants
\und{which is $p$-adically soluble} is equivalent to an integral quartic
with invariants $p^{-4}I$ and $p^{-6}J$.

The question of $p$-reducibility is almost completely settled by the
following proposition.  The result is simplest for primes greater
than~$3$, but even for these it is important to note that the
assumption of $p$-adic solubility is necessary for reduction to be
possible when the divisibility conditions are satisfied.

\newprop\IJreduction
\proclaim{Proposition \IJreduction}Let $I$ and $J$ be integers such that
$\Delta=4I^3-J^2\not=0$. 
\roster
\item If $p$ is a prime and $p\ge5$, then $(I,J)$ is $p$-reducible if and
only if $p^4\div I$ and $p^6\div J$.
\item $(I,J)$ is $3$-reducible if and only if {\it
either\/} $3^5\div I$ and $3^9\div J$, {\it or\/} $3^4\div\div I$,
$3^6\div\div J$ and $3^{15}\div\Delta$.
\item $(I,J)$ is $2$-reducible if $2^6\div I$, $2^9\div J$ and
$2^{10}\div 8I+J$.
\endroster
\endproclaim

This proposition is stated in \cite\BirchSD\ as Lemmas~3--5, but only
the proof of Lemma~3 (covering the case $p\ge5$) is given there.
Complete proofs in all cases (which are elementary though somewhat
lengthy) can be found in \cite\PSthesis.

Note that for $p=2$ we only have sufficient conditions for
reducibility.  Because of this, we will sometimes have to consider two
pairs of invariants, a smaller pair $(I_0,J_0)$ and a larger pair
$(16I_0,64J_0)$.  However, when searching for integral quartics with
the larger invariants, we may assume that the quartic cannot be
$2$-reduced, and this provides us with useful congruence conditions on
the coefficients of such a quartic.  We state these here.

\newprop\twoconditions
\proclaim{Proposition \twoconditions}
Let $g$ be an integral $2$-adically soluble quartic whose invariants
satisfy $2^4\div I$ and $2^6\div J$, such that
\roster
\item $g$ is not equivalent to an integral quartic with invariants
$2^{-4}I$ and~$2^{-6}J$;
\item $g$ is not equivalent to an integral quartic with the same 
invariants $I$ and~$J$ and smaller leading coefficient~$a$.
%\item $g$ is $2$-adically soluble.  
\endroster
Then the coefficients of $g$ satisfy
\roster
\item"(a)" $2\ndiv a$, $2^2\div b$, $2\div c$, $2^4\ndiv e$ and $2^4\ndiv
a+b+c+d+e$; \quad or
\item"(b)" $2\div\div a$, $2^2\div b$, $2^2\div c$, $2^3\ndiv e$ and $2^3\ndiv a+b+c+d+e$.
\endroster
Moreover, if $2^6\div I$ and $2^7\div J$, then we must have
\roster
\item"(a')" $2\ndiv a$, $2^2\div b$, $2^2\div\div c$, $2^3\div d$, and
$2^2\div\div e$; \quad or
\item"(b')" $2\ndiv a$, $2^2\div b$, $2^2\div\div c-2a+3b$, $2^3\div
d-b$ and $2^2\div\div a+c+e$.
%%%\item"(a')" $2\ndiv a$, $2^2\div b$, $c\equiv4\pmod8$, $d\equiv0\pmod8$, and
%%%$e\equiv4\pmod8$; \quad or
%%%\item"(b')" $2\ndiv a$, $2^2\div b$, $c\equiv 2a-3b+4\pmod8$, $d\equiv
%%%b\pmod8$ and $a+c+e\equiv4\pmod8$.
\endroster
\endproclaim

The first set of conditions stated here were given in \cite\BirchSD;
the second set are from \cite\PSthesis, which contains complete proofs
in both cases.  

Using this proposition, we may ensure that few of the quartics we find
when searching the larger pair of invariants are equivalent to one
with smaller invariants.  More significantly in terms of running time,
we have extra congruence conditions to apply when searching for the
larger invariants, which speeds up this search.

%in \cite\PSthesis\ it is estimated that these congruence conditions
%save about 60\% of the CPU time needed for the larger $(I,J)$ pair,
%which in turn takes about 90\% of the time.

It would appear that rational points in $E(\Q)$ whose quartics have
the larger pair of invariants lie in certain components of the
$2$-adic locus $E(\Q_2)$.  Further study of this would be very useful,
since if the search for quartics with the larger pair of invariants could
be eliminated or curtailed, it could result in a major saving of time
in the algorithm.

In practice, suppose that our original curve $E$ is given by a minimal
equation, with invariants $c_4$ and $c_6$.  We set $I=c_4$ and
$J=2c_6$.  Clearly the pair $(I,J)$ is $p$-minimal for $p\ge5$: for if
$p^4\div I$ and $p^6\div J$ then $p^{-4}c_4$ and $p^{-6}c_6$ would be
integral invariants of an elliptic curve, contradicting minimality
of~$E$, and similarly the pair $(p^4I,p^6J)$ is certainly
$p$-reducible by Proposition \IJreduction \therosteritem1.  Less
obvious is that $(I,J)$ is also $3$-minimal; using Kraus's conditions,
it is easy to check first that $(3^4I,3^6J)$ is certainly
$3$-reducible (one needs here that $\ord_3(c_6)\not=2$), and then that
$(I,J)$ itself is not $3$-reducible, using Proposition \IJreduction
\therosteritem2.

For $p=2$, the best we can do is the following.  First set $I=c_4$ and
$J=2c_6$.  Replace $(I,J)$ by $(2^{-4}I,2^{-6}J)$ if $2^4\div I$ and
$2^6\div J$; the resulting pair $(I,J)$ (which will not be further
divisible by~$2$) will be the basic pair of invariants.  Then we also
use the pair $(16I,64J)$ unless $4\div I$, $8\div J$ and
$16\div(2I+J)$.

The result of this step is then to produce either one or two pairs of
invariants $(I,J)$.  In the latter case, the following steps must be
carried out with both pairs separately.  
%
%Provided that we impose the extra congruence conditions on the quartic
%coefficients when using the larger pair, we will not need to test
%equivalence between quartics with different invariants.
%

\subhead Step 2: Finding the quartics with given $I$ and $J$ \endsubhead

We now have a fixed pair of invariants $(I,J)$ with
$\Delta=4I^3-J^2\not=0$, and we wish to find all integral quartics
with these invariants, up to equivalence.  We classify the quartics
$g(x)$ into types, according as $g(x)$ has no real roots (type~1),
four real roots (type~2) or two real roots (type~3).  When $\Delta<0$
only type~3 is possible, while if $\Delta>0$, only types 1 and~2 are
possible.  For each relevant type, we now determine a finite list of
quartics of that type with the given invariants such that every
soluble quartic with these invariants is equivalent to at least one on
the list.  We can ignore quartics which are negative definite (type~1
with $a<0$), since they will not be soluble over $\R$.  For each type,
we will determine a finite region of $(a,b,c)$-space such that every
quartic with invariants $I$ and~$J$ is equivalent to at least one in
this region.

As observed above, the number of real roots of $g(x)$ is equal to the
number of points $Q\in E(\R)$ satisfying $2Q=P$, where $P\in E(\R)$ is
the image under the map~$\xi$ of any real point on the homogeneous
space~$H$ with equation $y^2=g(x)$.  When $\Delta<0$, the real locus
is in one component, and $E(\R)$ is isomorphic to the circle group,
which is $2$-divisible with two $2$-torsion points, so in this case
the equation $2Q=P$ has exactly two solutions for all $P\in E(\R)$.
This agrees with the observation just made, that quartics with
negative discriminant~$\Delta$ will all have exactly two real roots.

Consider further the case $\Delta>0$.  Now $E(\R)$ has two components,
the connected component of the identity $E^0(\R)$ and a second
component which we call the `egg'.  There are four $2$-torsion
points, and $2E(\R) = E^0(\R)$.  There are therefore two possibilities
for a point $P\in E(\R)$ and its associated real quartic: if $P\in
E^0(\R)$, then there are four solutions~$Q$ to $2Q=P$, and $P$ will be
associated to a quartic of type~2 with four real roots.  On the other
hand, if $P\notin E^0(\R)$, then there are no solutions and the
quartic associated to~$P$ will be of type~1, with no real roots.

The image of $E(\Q)$ in $E(\R)/2E(\R)$ has order~$2$ or~$1$, depending
on whether or not there are any rational points on the egg.  Thus
there are two sub-cases to the case $\Delta>0$: if $E(\Q)\subset
E^0(\R)$, then there are no rational points on the egg, the index
is~$1$, and there will be {\it no\/} soluble quartics of type~1; on
the other hand, if $E(\Q)\not\subset E^0(\R)$, then there are rational
points on the egg, the index is~$2$, and there are equal numbers of
(equivalence classes of) soluble quartics of types 1 and~2.  Those of
type~2 will lead to rational points on $E(\Q)\cap E^0(\R)$, while
those of type~1 will lead to rational points on the egg.

To take advantage of this in practice, when $\Delta>0$ we will first
look for quartics of type~2; let the number of these be $n_1^+$, where
$n_1/n_1^+$ is either 1 or~2.  At this stage we will already know the
rank to within one, since if we set $r^+=\log_2(n_1^+)$ then (assuming
no rational $2$-torsion) we have either $r=r^+$ or $r=r^++1$.  Then we
start to look for quartics of type~1; as soon as we find one which is
soluble, then we may abort the search for type~1 quartics at that
point, and assert that $r=r^++1$.  On the other hand, if we complete
the search for quartics of type~1 without finding any soluble ones,
then we will know that $r=r^+$, and we will have proved that there are
no rational points on the egg.  An example of the second possibility
happens with the curve $E=[0,0,1,-529,-3042]$ (which is the
$-23$-twist of the curve $[0,0,1,-1,0]$ with conductor~37 and rank~1),
which has rank~1 with generator $(46,264)$ on the identity component,
and no rational points on the egg.%
\footnote{Thanks to Nelson Stephens for this example.}

If we happened to know in advance that there were rational points on
the egg (perhaps by a short preliminary search for such points with
small height), then we would already know that $r=r^++1$, and we would
not need to search for type~1 quartics at all.

In order to find all integral quartics of a given type (up to
equivalence) we proceed as follows.  First, following \cite\BirchSD,
we determine bounds on the coefficients $a$, $b$ and~$c$.  We also set
up a sieve based on the seminvariant syzygy (\semisyz) to speed up our
search through this region of $(a,b,c)$-space.  For triples $(a,b,c)$
in the region which pass the sieve, we solve for $d$ and $e$ and
ensure that they are integral.  Finally, we check that the quartic we
have constructed satisfies any further congruence conditions we
require (for example, when we are using the larger pair of
invariants).

The method for bounding the coefficients which is developed in
\cite\BirchSD\ involves using the auxiliary (resolvent) cubic equation
\neweq\auxcubic
$$
     \phi^3-3I\phi+J=0 \tag\auxcubic
$$
which will have one real root (type 3) or three real roots (types 1
and 2), since its discriminant is $27\Delta$.  Indeed, $\phi$ is a
root of (\auxcubic) if and only if $(-3\phi,0)$ is a point of
order~$2$ on the curve $\EIJ$.  

In each case, the bound for $b$ arises simply from the fact that the
quartics $g(x)$ and $g(x+k)$ are equivalent, and the coefficients
of the latter are $(a,b+4ak,\ldots)$, so that we may assume that
$b$ is reduced modulo~$4a$.  Also, note that the bounds on $c$ are
effectively bounds on the seminvariant $8ac-3b^2=-p$, which is how they
arise in \cite\BirchSD.

\subsubhead Bounds for $(a,b,c)$: Type 1 \endsubsubhead
%
Here we may assume $a>0$ for real solubility.  Order the three real
roots of (\auxcubic) as $\phi_1>\phi_2>\phi_3$, and set $K=(4I-\phi_1^2)/3$. 
Then the bounds on $a$, $b$, $c$ are
$$ \align
   0 < {}&a \le \frac{K+K^{\frac12}\phi_1}{3K^{\frac12}+\phi_1+2\phi_2}; \cr
   -2a< {}&b\le 2a; \cr
  \frac{4a\phi_2+3b^2}{8a} \le {}&c \le \frac{4a\phi_1+3b^2}{8a}.\cr
 \endalign
$$

\subsubhead Bounds for $(a,b,c)$: Type 2 \endsubsubhead
%
This subdivides into subtypes according as $a>0$ or $a<0$.  For
$a>0$ we take $\phi_1>\phi_2>\phi_3$ and search the region
$$ \align
   0 < {}&a \le \frac{I-\phi_2^2}{3(\phi_2-\phi_3)}; \cr
   -2a< {}&b\le 2a; \cr
  \frac{4a\phi_2-\frac43(I-\phi_2^2)+3b^2}{8a} \le{}&c \le
\frac{4a\phi_3+3b^2}{8a}.\cr
 \endalign
$$
Then for $a<0$ we take $\phi_1<\phi_2<\phi_3$ and search over 
$$ \align
   0 <{} &-a \le \frac{I-\phi_2^2}{3(\phi_3-\phi_2)}; \cr
   -2|a|< {}&b\le 2|a|; \cr
  \frac{4a\phi_2-\frac43(I-\phi_2^2)+3b^2}{8a} \ge{} &c \ge
\frac{4a\phi_3+3b^2}{8a}.\cr
 \endalign
$$

\subsubhead Bounds for $(a,b,c)$: Type 3 \endsubsubhead
%
Here we let $\phi$ be the unique real root of (\auxcubic), and
search
$$ \align
   \frac13\phi-\sqrt{\frac{4}{27}(\phi^2-I)} \le {}&a \le
\frac13\phi+\sqrt{\frac{4}{27}(\phi^2-I)}; \cr
   -2|a|< {}&b\le 2|a|; \cr
  \frac{9a^2-2a\phi+\frac13(4I-\phi^2)+3b^2}{8|a|} \le {}&c.\sgn(a) \le
\frac{4a\phi+3b^2}{8|a|}.\cr
 \endalign
$$

\subsubhead The syzygy sieve \endsubsubhead
%
Recall the seminvariant syzygy
$$
  27r^2 = p^3 - 48Ia^2p - 64Ja^3 = s(a,p), \tag\semisyz
$$
say, where $p=3b^2-8ac$ and $r=b^3+8a^2d-4abc$.  For fixed $I$, $J$
the expression $s(a,p)$ is a polynomial in $a$, $b$ and~$c$, which we
require to be 27~times an integer square.  We can set up a quadratic
sieve as follows: for each of several sieving moduli $m$ we create and
initialize an $m\times m$ array indicating whether $s(a,p)$ is
27~times a square modulo~$m$, for each pair $(a,p)$ modulo~${m}$.  We
take one of the moduli to be~$9$ and use it to force the right-hand
side of (\semisyz) to be divisible by~$27$; it will certainly be
positive, as this is ensured by the bounds on ~$c$.

For each $(a,b,c)$ in the region searched, we check that it passes the
sieving test; it is then quite likely that $s(a,p)$ will be 27~times a
square, since it is so modulo a large modulus and is positive.  We
then test whether this is the case, discarding $(a,b,c)$ if not, and
if so we then find $r$.  We can take $r>0$, since the quartics with
coefficients $(a,b,c,d,e)$ and $(a,-b,c,-d,e)$ are equivalent, with
opposite signs of their respective $r$-seminvariants.  In fact, we
treat the triples $(a,\pm b,c)$ together in practice.

Implementation note: It is worth pointing out that a large proportion
of the running time of our algorithm is spent testing whether large
integers are squares (given that they are positive and congruent to
squares modulo several carefully chosen moduli), and find their
integer square root if so.  This is needed here, and in our searches
for rational points, both on the elliptic curve directly, and on the
homogeneous spaces.  Hence it is crucial that we have access to
efficient procedures for this in the multiprecision integer package we
use.

\subsubhead Solving for $d$ and $e$ \endsubsubhead
%
Given integers $a$, $b$, $c$, $r$ satisfying (\semisyz) with
$p=3b^2-8ac$, we can solve for $d$ and $e$, setting
$$
  d = (r - b^3 + 4abc) / (8a^2)  \qquad\text{and}\qquad 
  e = (I + 3bd - c^2)  / (12a).
$$
This will certainly give rational values for $d$ and $e$; we must
check that they are integral, discarding the triple $(a,b,c)$ if not.
If they are, we have integral coefficients $(a,b,c,d,e)$ of a quartic
$g(x)$ with invariants $I$ and $J$ in the search region, which we add
to our list for further processing.

\subsubhead Solving for the roots of $g(x)$ \endsubsubhead
%
For later use, when we check for triviality, and again when we search
for rational points on the homogeneous spaces, we will need to know
the real roots of the quartic $g(x)$ we have constructed.  Although
the formulae for finding the roots of quartic are well-known, we give
them here: since we already know the roots of the resolvent cubic,
there is very little work remaining.

For $i=1,2,3$ we set $z_i=(4a\phi_i+p)/3$ where the $\phi_i$ are the
three roots of (\auxcubic).  The product of these quantities is $r^2$
(from (\semisyz) again), and we form their square roots with
product~$r$ by setting $w_1=\sqrt{z_1}$, $w_2=\sqrt{z_2}$, and
$w_3=r/(w_1w_2)$.  Then the roots of $g(x)$ are
$$
  \align
    x_1 &= (\phantom{-}w_1 + w_2 - w_3 - b) / (4a), \\
    x_2 &= (\phantom{-}w_1 - w_2 + w_3 - b) / (4a), \\
    x_3 &= (-w_1 + w_2 + w_3 - b) / (4a), \\
    x_4 &= (-w_1 - w_2 - w_3 - b) / (4a). \\
  \endalign
$$

\bigskip

We will not give here a pseudo-code algorithm for the search for
quartics, as it is straightforward in principle, although in practice
it needs careful book-keeping.  As this is the most time-consuming
part of the whole procedure, particularly when the second, larger,
pair of invariants must be used, it is important to make the
implementation code as efficient as possible.  

At the end of this step we will have a list of quartics with the
desired invariants.  We now discard any which are equivalent to
earlier ones, or are not locally soluble at some prime $p$, and try to
find rational roots on the remainder.  In practice we may choose to
apply these tests in a different order, such as not bothering to check
equivalences between quartics which are not locally soluble.

\subhead Step 3: Testing triviality \endsubhead

For each quartic $g(x)$ in the list, we already know its roots $x$ to
reasonable precision.  If $x$ is rational, then $ax$ is integral,
which we can test.  If we suspect that $ax$ is equal to an integer $n$
to within some working tolerance, we can check whether $n/a$ is a root
of $g(x)$ using exact arithmetic.

\subhead Step 3: Testing equivalence of quartics \endsubhead

With each quartic we find with the right invariants, we store its
coefficients, type, roots and seminvariants $p$ and $r$.  We also
compute and store the number of roots of the quartic (including roots
at infinity) modulo each of several primes not dividing its
discriminant, as these numbers are clearly invariant under
equivalence.\footnote{This was suggested to us by S. Siksek.}

When testing equivalence of two quartics, we first check that their
invariants and type are the same, as well as their numbers of roots
modulo these primes.  If this is the case, we use a general test for
equivalence (valid over any field) from \cite\JCinvariants, which we
state here.\footnote{The algorithm presented here only applies to
quartics.  In the First Edition we presented a different algorithm,
described in \cite\BirchSD, which is messier to implement, but which
generalizes more readily to more general situations, such as testing
the equivalence of binary forms of higher degree.}
\newprop\zequivalg

\proclaim{Proposition \zequivalg} Let $g_1$ and~$g_2$ be quartics over the
field~$K$, both having the same invariants $I$ and~$J$, and with
leading coefficients~$a_i$ and seminvariants $p_i$ and~$r_i$ for
$i=1,2$.  Then $g_1$ is equivalent to~$g_2$ over~$K$ if and only if
the quartic $u^4-2pu^2-8ru+s$ has a root in~$K$, where
$$\align
p &= (32a_1a_2I + p_1p_2)/3, \\
r &= r_1r_2, \qquad\text{and}\\
s &= ( 64I(a_1^2p_2^2 + a_2^2p_1^2 + a_1a_2p_1p_2)
   - 256a_1a_2J(a_1p_2+a_2p_1) - p_1^2p_2^2) /27.
\endalign
$$
\endproclaim

The quantities $p$, $r$ and~$s$ in this proposition will be integers
when $g_1$ and $g_2$ are integral.  Converting the proposition into an
algorithm is straightforward.

\subhead Step 5: Testing local and global solubility \endsubhead

This is carried out using the procedures and strategy described
earlier.

\subhead Step 6: Final computation of the rank \endsubhead

\def\tn1{\tilde{n}_1}
\def\te1{\tilde{e}_1}

The number of quartics found (up to equivalence) which are everywhere
locally soluble is $n_2$, the order of the $2$-Selmer group.  This
must be a power of~$2$, say $n_2=2^{e_2}$, which serves as a check on
our procedures.  The number $n_1$ with a rational point is also a
power of~2, say $n_1=2^{e_1}$, equal to the order of $E(\Q)/2E(\Q)$.
If we have found rational points on all $n_2$ locally soluble
quartics, then certainly $n_1=n_2$, so that $\Sha(E/\Q)[2]$ is
trivial, and the rank of $E(\Q)$ is $e_1-e_0$ where
$\left|E(\Q)[2]\right|=2^{e_0}$ with $e_0=0$, $1$ or~$2$.  The rank is
equal to the Selmer rank $e_2-e_0$ in this case.  (Usually $e_0=0$
when we are using this method.)

As before, we may not have found global points on all the locally
soluble quartics; if the number on which we have points is $\tn1$
with $\tn1<n_2$ then we only know that $\tn1 \le n_1 \le n_2$.  If
$\tn1$ is not a power of~2, we will know that $n_1>\tn1$, so that at
least some of our locally soluble quartics must have rational points
which we have not found.  In this case, we replace $\tn1$ by the next
highest power of~2, say $\tn1=2^{\te1}$.  Then we have bounds on
the rank, namely
$$
    \te1-e_0 \le e_1-e_0 = \rk(E(\Q)) \le e_2-e_0,
$$
and on the order of~$\Sha(E/\Q)[2]$:
$$
    \left|\Sha(E/\Q)[2]\right| \le n_2/\tn1.
$$

One final point: from the Selmer conjecture, we expect the Selmer rank
$e_2-e_0$ to differ from the actual rank $e_1-e_0$ by an even number,
so that $e_2\equiv e_1\pmod2$.  This would also follow from
the conjecture that $\Sha(E/\Q)$ is finite, since then its order is
known to be a perfect square, so that $n_2/n_1$ must be a square.  So
if we find that $e_2\not\equiv\te1\pmod2$, then we suspect that the
rank is at least one more than our lower bound, and can output a
comment to this effect, though of course we will not have proved that the
rank is greater than our lower bound.  In some cases, such as for a
modular curve where we know the sign of the functional equation, we
may have other conjectural evidence for the parity of the rank.

\subhead Step 7: Recovering points on $E$ \endsubhead

Each quartic $g(x)$ for which the homogeneous space $y^2=g(x)$ has a
rational point $R$ leads to a rational point $P=\xi(R)$ on the model
$\EIJ$ of our curve~$E$, via the formula (\ximap) given above.  If we
apply this formula to all the inequivalent quartics with rational
points which we found in computing the rank of $E$, we will have a
complete set of coset representatives for $2E(\Q)$ in $E(\Q)$,
provided that $\tn1=n_1$.  In cases where we have rounded up $\tn1$ to
the nearest power of~2, we will still have generators for
$E(\Q)/2E(\Q)$, and can fill in the missing coset representatives if
we wish.

\bigskip

This completes our description of algorithms for determining the
Mordell-Weil group $E(\Q)$.

%
% CHAPTER 3 SECTION 7
%
\beginsection{\Periods}
\head\Periods\ The period lattice \endhead

In this section we show how to compute the complex periods for an 
elliptic curve defined over the complex numbers.  We used this in our 
investigation of modular curves to check that the exact integral 
equations we found (after rounding the approximate computed values of 
$c_4$ and $c_6$) did have the correct periods; and also in our method for 
computing isogenous curves, which we describe in the following section.

Let $E$ be an elliptic curve defined over the complex numbers $\C$,
given by a \W\ equation.  We wish to compute periods $\lambda_1$ and
$\lambda_2$ which are a $\Z$-basis for the period lattice $\Lambda$ of
$E$.  We do this using Gauss's arithmetic--geometric mean (\agm)
algorithm.  Write the equation for $E$ in the form \neweq{\agmperiods}
$$
 \left(y+\frac{a_1x+a_3}{2}\right)^2 = x^3 + \frac{b_2}{4}x^2 + \frac{b_4}{2}x +
\frac{b_6}{4} = (x-e_1)(x-e_2)(x-e_3),  
$$
where the roots $e_i$ are found as complex floating-point approximations 
(using Cardano's formula, say).  Then the periods are given by 
$$
  \aligned
    \lambda_1&=\frac{\pi  }{\agm(\sqrt{e_3-e_1},\sqrt{e_3-e_2})},\\
    \lambda_2&=\frac{\pi i}{\agm(\sqrt{e_3-e_1},\sqrt{e_2-e_1})}.\\
  \endaligned \tag\agmperiods
$$
Notice that in general this involves the \agm\ of pairs of complex
numbers.  This is a multi-valued function: at each stage of the \agm\
algorithm we replace the pair $(z,w)$ by $(\sqrt{zw},\frac12(z+w))$,
and must make a choice of complex square root.  It follows from work
of Cox (see \cite\Cox) that while a different set of choices does lead
to a different value for the \agm, the periods we obtain this way will
nevertheless always be a $\Z$-basis for the full period lattice
$\Lambda$.  We have found this to be the case in practice, where we
always choose a square root with positive real part, or with positive
imaginary part when the real part is zero.  The computation of
$\lambda_1$ and $\lambda_2$ by this method is very fast, as the \agm\
algorithm converges extremely quickly, even in its complex form.  As a
check on the values obtained, in each case we recomputed the
invariants $c_4$ and $c_6$ of each curve from these computed periods
$\lambda_1$ and $\lambda_2$, using the standard formulae given in
Chapter II; in every case we obtained the correct values (known
exactly from the coefficients of the minimal \W\ equation) to within
computational accuracy.

If the curve is defined over $\R$, we can avoid the use of the complex
\agm, and also arrange that $\lambda_1$ is a positive real period, as
follows.  First suppose that all three roots $e_i$ are real; order the
roots so that $e_3 > e_2 > e_1$, and take the positive square root in
the above formulae.  Then we may use the usual \agm\ of positive reals
in (\agmperiods), and thus obtain a positive real value for
$\lambda_1$ and a pure imaginary value for $\lambda_2$.  This is the
case where the discriminant $\Delta>0$ and the period lattice is
rectangular.  When $\Delta<0$ there is one real root, say $e_3$, and
$e_2=\overline{e_1}$.  If $\sqrt{e_3-e_1}=z=s+it$ with $s>0$ then
$\sqrt{e_3-e_2}=\overline{z}=s-it$, so that
$\lambda_1=\pi/\agm(z,\overline{z}) = \pi/\agm(|z|,s)$ which is also
real and positive.

%
% CHAPTER 3 SECTION 8
%

\beginsection{\Isogenies}
\head\Isogenies\ Finding isogenous curves\endhead

Given an elliptic curve $E$ defined over $\Q$, we now wish to find all
curves $E'$ isogenous to $E$ over $\Q$.  The set of all such curves is
finite (up to isomorphism), and any two curves in the isogeny class
are linked by a chain of isogenies of prime degree $l$.  Thus it
suffices to be able to compute $l$-isogenies for prime $l$, if we can
determine those $l$ for which rational $l$-isogenies exist.  The
latter question can be rather delicate in general, and we have to have
a completely automatic algorithmic procedure if we are to apply it to
several thousand curves, such as we had to when preparing the tables.

When the conductor $N$ of $E$ is square-free, so that $E$ has good or
multiplicative reduction at all primes, $E$ is called semi-stable.  In
this case, a result of Serre (see \cite{\Serre}) says that either $E$
or the isogenous curve $E'$ has a rational point of order $l$, and so
by Mazur's result already mentioned, $l$ can only be $2$, $3$, $5$ or
$7$.  Moreover, if a curve $E$ possesses a rational point of order
$l$, then the congruence $1+p-a_p\equiv0\pmod{l}$ holds for all primes
$p$ not dividing $Nl$, so the presence of such a point is easy to
determine, even if it is not $E$ itself but the isogenous curve $E'$
which possesses the rational $l$-torsion, since the trace of Frobenius
$a_p$ is isogeny-invariant.

If $E$ is not semi-stable we argue as follows.  The existence of a
rational $l$-isogeny is purely a function of the $j$-invariant $j$ of
$E$: in fact, pairs $(E,E')$ of $l$-isogenous curves parametrize the
modular curve $X_0(l)$ whose non-cuspidal points are given by the
pairs $(j(E),j(E'))$.  For $l=2$, $3$, $5$, $7$ or $13$ the genus of
$X_0(l)$ is zero, and infinitely many rational $j$ occur.  The only
other values of $l$ for which rational $l$-isogenies occur are
$l=11,17,19,37,43,67$, and $163$, and these occur for only a small
finite number of $j$-invariants (see below).  The fact that no other
$l$ occur is a theorem of Mazur (see \cite{\MazurTa} and
\cite{\MazurTb}), related to the theorem limiting the rational torsion
which we quoted earlier in Section \Torsion\ of this chapter.  These
extra values occur only for curves with CM (\CM, see the next
section), apart from $l=17$ (where $X_0(l)$ has genus~1) and the
exotic case $l=37$ studied by Mazur and Swinnerton--Dyer in
\cite{\MazurSD} (where $X_0(l)$ has genus~2).

For isogenies of non-prime degree $m$, the degrees which occur are:
$m\le10$, and $m=12$, 16 18, and 25 (where $X_0(l)$ has genus~0,
infinitely many cases); and finally $m=14$, 15, 21, and 27.  The
latter occur first for conductors $N=49$ (with CM), $N=50$, $N=162$
and $N=27$ (with CM) respectively.  See \cite{\Antwerp, pages 78--80}\
for more details.

Thus our procedure is:

\item{$\bullet$} If $N$ is square-free, try $l=2,3,5,7$ only;

\item{$\bullet$} else try $l=2,3,5,7$ and $13$ in all cases; and

\item{$\bullet$} if $j(E)=-2^{15}$, $-11^2$, or $-11\cdot131^3$, try also $l=11$;

\item{$\bullet$} if $j(E)=-17^2\cdot101^3/2$ or $-17\cdot373^3/2^{17}$, try also $l=17$;

\item{$\bullet$} if $j(E)=-96^3$, try also $l=19$;

\item{$\bullet$} if $j(E)=-7\cdot11^3$ or $-7\cdot137^3\cdot2083^3$, try also $l=37$;

\item{$\bullet$} if $j(E)=-960^3$, try also $l=43$;

\item{$\bullet$} if $j(E)=-5280^3$, try also $l=67$;

\item{$\bullet$} if $j(E)=-640320^3$, try also $l=163$.

\smallskip
Now we turn to the question of finding all curves (if any) which are 
$l$-isogenous to our given curve $E$ for a specific prime $l$.  The 
kernel of the isogeny is a subgroup $A$ of $E(\overline{\Q})$ which is 
defined over $\Q$, but the points of $A$ may not be individually rational 
points. If we have the coordinates of the points of a subgroup of $E$ of 
order $l$ defined over $K$, we may use V\'elu's formulae in \cite\Velu\ to 
find the corresponding $l$-isogenous curve.    Finding such coordinates 
by algebraic means is troublesome, except when the subgroup is point-wise 
defined over $K$, and instead we resort to a floating-point method.

The case $l=2$ is simpler to describe separately. Obviously in this
case the subgroup of order~2 defined over $\Q$ must consist of a
single rational point $P$ of order~2 together with the identity. We
have already found such points, if any, in computing the torsion.
There will be 0, 1 or 3 of them according to the number of rational
roots of the cubic $4x^3 + b_2x^2 + 2b_4x + b_6$. If $x_1$ is such a
root, then $P=(x_1,y_1)$ has order~2, where $y_1=-(a_1x_1+a_3)/2$.  As a
special case of V\'elu's formulae we find that the isogenous curve
$E'$ has coefficients $[a_1',a_2',a_3',a_4',a_6'] =
[a_1,a_2,a_3,a_4-5t,a_6-b_2t-7w]$ where
$$
     t=(6x_1^2+b_2x_1+b_4)/2  \qquad\text{and}\qquad  w=x_1t.
$$
Note that the point $(x_1,y_1)$ need not be integral even when $E$ has
integral coefficients $a_i$, but that $4x_1$ and $8y_1$ are certainly
integral, by the formulae given; thus the model just given for the
isogenous curve may need scaling by a factor of~2 to make it integral.

The simpler formula for a curve in the form $y^2=x^3+cx^2+dx$ and the
point $P=(0,0)$ was given in the previous section: the formulae just
given take the curve $[0,c,0,d,0]$ to $[0,c,0,-4d,-4cd]$, which
transforms to $[0,-2c,0,c^2-4d,0]$ after replacing $x$ by $x-c$.  The
relation between the two formulae is given by $c=12x_1+b_2$ and
$d=16t$.

For reference we give here similar algebraic formulae for
$l$-isogenies for $l=3$ and $l=5$, from Laska's book \cite\Laskabook.
In each case we assume that the curve $E$ is given by an equation of
the form $y^2=x^3+ax+b$, and the isogenous curve~$E'$ by
$y^2=x^3+Ax+B$.  Each subgroup of~$E$ of order~$l$ is determined by a
rational factor of degree~$(l-1)/2$ of the $l$-division polynomial of
degree~$(l^2-1)/2$, whose roots are the $x$-coordinates of the points
in the subgroup. The simplest case is $l=3$, where there is just one
$x$-coordinate, which must be rational.

\subsubhead $l=3$\endsubsubhead  Let $\xi$ be a root of the 3-division
polynomial $3x^4+6ax^2+12bx-a^2$. Then the 3-isogenous curve $E'$ is
given by
$$
  \align
              A &= -3(3a+10\xi^2)\cr
              B &= -(70\xi^3+42a\xi+27b).\cr
  \endalign
$$

\subsubhead $l=5$\endsubsubhead Let $x^2+h_1x+h_2$ be a rational factor of the 5-division polynomial
$5x^{12} + 62ax^{10} + 380bx^9 -105a^2x^8 + 240abx^7 -
(300a^3+240b^2)x^6 - 696a^2bx^5 - (125a^4+1920ab^2)x^4 -
(1600b^3+80a^3b)x^3 - (50a^5+240a^2b^2)x^2 - (100a^4b+640ab^3)x +
(a^6-32a^3b^2-256b^4)$. Then the 5-isogenous curve $E'$ is given by
$$
  \align
              A &= -19a - 30(h_1^2-2h_2) \cr
              B &= -55b-14(15h_1h_2-5h_1^3-3ah_1).\cr
  \endalign
$$

A similar formula is given in \cite\Laskabook\ for $l=7$, where $A$
and $B$ are given in terms of $a$, $b$ and the coefficients of a
factor $x^3+h_1x^2+h_2x+h_3$ of the 7-division polynomial.  Rather
than take up space by giving the latter here, we refer the reader to
\cite{\Laskabook, page 72}.

Now we turn to V\'elu's formulae in the case of an odd prime $l$.  Let
$P=(x_1,y_1)$ be a point of order $l$ in $E(\overline{\Q})$, and set
$kP=(x_k,y_k)$ for $1\le k\le (l-1)/2$. Define
$$
   t_k = 6x_k^2+b_2x_k+b_4  \qquad\text{and}\qquad
   u_k = 4x_k^3+b_2x_k^2+2b_4x_k+b_6,
$$
and then set
$$
   t = \sum_{k=1}^{(l-1)/2}t_k \qquad\text{and}\qquad
   w = \sum_{k=1}^{(l-1)/2}\left(u_k+x_k t_k\right).
$$
Then the isogenous curve $E'$ has coefficients 
$[a_1,a_2,a_3,a_4-5t,a_6-b_2t-7w]$ as before.    Again, these may not be 
integral, even when the original coefficients were; but since the $x_k$ 
are the roots of a polynomial of degree $(l-1)/2$ with integral 
coefficients and leading coefficient $l^2$ (the so-called $l$-division 
equation), we must have $l^2x_k$ integral.  Thus a scaling factor of $l$ 
will certainly produce an integral equation.

We make these remarks on integrality as our method is to find the 
coordinates $x_k$ and $y_k$ as real floating-point approximations, and 
thus to determine the coefficients of any curves $l$-isogenous to $E$ 
over $\R$; there will always be exactly two such curves over $\R$, but of 
course they will not necessarily be defined over $\Q$. As we will only 
know the coefficients $a_i'$ of the isogenous curves approximately, we 
wish to ensure that if they are rational then they will in fact be 
integral, so that we will be able to recognize them as such.

First we find the period lattice $\Lambda$ of $E$, as described in the
previous section.  The $\Z$-basis $\left[\lambda_1,\lambda_2\right]$
of $\Lambda$ is normalized as follows: there are two cases to
consider, according as $\Delta>0$ (first or `harmonic' case) or
$\Delta<0$ (second or `anharmonic' case).  In both cases $\lambda_1$
is real (the least positive real period); in the first case,
$\lambda_2$ is pure imaginary, while in the second case,
$2\lambda_1-\lambda_2$ is pure imaginary.  We can also ensure that
$\tau=\lambda_2/\lambda_1$ is in the upper half-plane; however we can
not simultaneously arrange that $\tau$ is in the usual fundamental
region for $\SL(2,\Z)$, and this needs to be remembered when evaluating
the \W\ functions below.

Of the $l+1$ subgroups of $\C/\Lambda$ of order $l$, the two defined
over $\R$ are the one generated by $z=\lambda_1/l$ (in both cases),
and in the first case, the one generated by $z=\lambda_2/l$, or in the
second case, the one generated by $z=(\lambda_1-2\lambda_2)/l$.  Thus
$z/\lambda_1$ is either $1/l$, $\tau/l$, or $(1-2\tau)/l$.  Let
$\wp(z;\tau)$ denote the
\W\ $\wp$-function relative to the lattice $[1,\tau]$.  Then we have
$$
x_k = \wp(kz\lambda_1^{-1};\tau)\lambda_1^{-2} - \frac{1}{12}b_4 
\qquad\text{and}\qquad
y_k = \frac12\left(\wp'(kz\lambda_1^{-1};\tau)\lambda_1^{-3} - a_1x_k-a_3\right).
$$
Here we have had to take account of the lattice scaling 
$[\lambda_1,\lambda_2] = \lambda_1[1,\tau]$, and also of the fact that 
$(\wp(z),\wp'(z))$ is a point on the model of $E$ of the form 
$y^2=4x^3-g_2x-g_3 = 4x^3-(c_4/12)x-(c_6/216)$ rather than a standard 
model where the coefficient of $x^3$ is~1.

We evaluate these points of order $l$ numerically for
$k=1,2,\ldots,(l-1)/2$, for each of the two values of $z$ (depending
on whether we are in case~1 or case~2).  Substituting into V\'elu's
formulae, we obtain in each case the real coefficients $a_i'$ of a
curve which is $l$-isogenous to $E$ over $\R$.  If these coefficients
are close to integers we round them and check that the resulting curve
over $\Q$ has the same conductor $N$ as the original curve $E$.  If
not, we also test the curve with coefficients $l^ia_i'$.

The resulting program finds $l$-isogenous curves very quickly for any
given prime $l$.  We run it for all primes $l$ in the set determined
previously, applying it recursively to each new curve found until we
have a set of curves closed under $l$-isogeny for these values of $l$.
Since the set of primes~$l$ for which a rational $l$-isogeny exists is
itself an isogeny invariant, once we have finished processing the
first curve in the class, we will already know which primes~$l$ to use
for all the remaining curves.

Some care needs to be taken with a method of computation such as this,
where we use floating-point arithmetic to find integers.  The series
we use to compute the periods and the \W\ function and its derivative
all converge very quickly, so that we can compute the $a_i'$ to
whatever precision is available, though of course in practice some
rounding error is bound to arise.  When we test whether a
floating-point number is `approximately an integer' in the program, we
must make a judgement on how close is close enough.  With too relaxed
a test, we will find too many curves are `approximately integral';
usually these will fail the next hurdle, where we test the conductor,
but this takes time to check (using Tate's algorithm).  On the other
hand, too strict a test might mean that we miss some rational
isogenies altogether, which is far more serious. In compiling the
tables, there was only one case which caused trouble after the program
had been finely tuned.  The resulting error resulted in a curve
(916B1) being erroneously listed as 3-isogenous to itself in the first
(preprint) edition of the tables; this is possible only when a curve
has complex multiplication, which is not the case here, though it does
not often occur even in the complex multiplication case (see the
remarks in the next section). Unfortunately the error was not noticed
in the automatic generation of the typeset tables, and I am grateful
to Elkies for spotting it.\footnote{This error also somehow survived
into the first edition of this book, despite these comments in the
text.}  The curve $E=[0,0,0,-1013692,392832257]$ has three real points
of order~2, two of which are equal to seven significant figures; the
period ratio is approximately $7i$.  One of the curves 2-isogenous to
$E$ over $\R$ has coefficients
$[0,0,0,-1013691.999999999992,392832257.000000006]$, which are
extremely close to those of $E$ itself.  Thus this new curve, which is
not defined over $\Q$, passed both our original tests (the
coefficients are extremely close to integers, and the rounded
coefficients are those of a curve of the right conductor, namely $E$
itself).  After this example was discovered, we inserted an extra line
in the program, to print a warning whenever a supposedly isogenous
curve was the original curve itself, and reran the program on all 2463
isogeny classes (which only takes a few minutes of machine time).  The
result was that expected, namely that 916B1 is the only curve for
which this phenomenon occurs within the range of the
tables\footnote{Another example of the same type occurs for curve
1342C3, where the period ratio is approximately $9.5i$.}.  There is no
example of a curve actually $l$-isogenous to itself with conductor
less than 1000.

Our original implementation of this algorithm in Algol68 used a
precision of approximately 30 significant figures for its real and
complex arithmetic, which was sufficient to find all the isogenous
curves up to conductor~1000.  However, our implementation in C++
misses several isogenous curves when using standard double precision,
with approximately 15 digits (though this runs very quickly); we need
to use a multiprecision floating-point package (such as the one
included in {\tt LiDIA}) to obtain a satisfactory working program, though
the resulting code runs very much slower.  In our extended
computations to conductor~5077, we have computed the isogenies
independently using both a C++/{\tt LiDIA} program and a PARI program, and
the results agree.

When we were initially persuaded to extend the tables to include
isogenous curves as well as the modular curves themselves, we were
afraid that the total number of resulting curves would be rather
larger than it turned out to be.  On average, we found that the number
of curves per isogeny class was 5113/2463, or just under 2.08.  We do
not know of any asymptotic analysis, or even a heuristic argument,
which would predict an average number of two curves per class.
However, it is dangerous to generalize from the limited amount of data
which we have available.  In the extended computations to
conductor~$5077$, the ratio slowly diminishes; for all curves up to
this conductor, the ratio is $31570/17583 = 1.795$.

%
% CHAPTER 3 SECTION 9
%
\beginsection{\TwistsEtc}
\head\TwistsEtc\ Twists and complex multiplication \endhead

\subhead Traces of Frobenius \endsubhead

If $E$ is given by a standard minimal \W\ equation over $\Z$, then for all
primes $p$ of good reduction the trace of Frobenius $a_p$ is given by 
$$
  a_p = 1+p-\left|E({\Bbb F}_p)\right|.
$$
If $E$ has bad reduction at $p$, this same formula gives the correct value
for the $p$th Fourier coefficient of the $L$-series of $E$.

Since in our applications we never needed to compute $a_p$ for large
primes $p$, we used a very simple method to count the number of points
on $E$ modulo $p$.  First, for all primes $p$ in the desired range
(say $3\le p\le1000$; $p=2$ would be dealt with separately), we
precompute the number $n(t,p)$ of solutions to the congruence
$s^2\equiv t\pmod{p}$.  Then we simply compute 
$$
 a_p = p -
\sum_{x=0}^{p-1} n(4x^3 + b_2x^2 + 2b_4x + b_6,p).  
$$ 
This was sufficient for us to compute $a_p$ for all $p<1000$ for all
the curves in the table, which we did to compare with the
corresponding Hecke eigenvalues.  For large $p$, there are far more
efficient methods, such as the baby-step giant-step method or Schoof's
algorithm (see \cite{\Schoof}).  Details of these may be found in
\cite\Cohenbook.  More recently, even better algorithms have been
developed, by Atkin, Elkies, Morain, M\"uller and others.  For
example, Morain and Lercier in 1995 successfully computed the number
of points on the curve $[0,0,0,4589,91228]$ over ${\Bbb F}_p$ for
$p=10^{499}+153$, a prime with 500 decimal digits.  This took 4200 hours
of computer time.

\subhead Twists \endsubhead

A twist of a curve $E$ over $\Q$ is an elliptic curve defined 
over $\Q$ and isomorphic to $E$ over $\overline{\Q}$ but not necessarily 
over $\Q$ itself.  Thus the set of all twists of $E$ is the set of all 
curves with the same $j$-invariant as $E$.  These can be simply 
described, as follows.

First suppose that $c_4\not=0$ and $c_6\not=0$; equivalently,
$j\not=1728$ and $j\not=0$ (respectively).  Then the twists of $E$ are
all quadratic, in that they become isomorphic to $E$ over a quadratic
extension of $\Q$.  For each integer $d$ (square-free, not 0 or 1),
there is a twisted curve $E*d$ with invariants $d^2c_4$ and $d^3c_6$,
which is isomorphic to $E$ over $\Q(\sqrt{d})$.  If $E$ has a model of
the form $y^2=f(x)$ with $f(x)$ cubic, then $E*d$ has equation
$dy^2=f(x)$.  A minimal model for $E*d$ may be found easily by the
Laska--Kraus--Connell algorithm.  The conductor of $E*d$ is only
divisible by primes dividing $ND$, where $D$ is the discriminant of
$\Q(\sqrt{d})$.  The simplest case is when $\gcd(D,N)=1$; then $E*d$
has conductor $ND^2$.  More generally, if $D^2\ndiv N$ then $E*d$ has
conductor $\lcm(N,D^2)$, but if $D^2\div N$ then the conductor may be
smaller; for example, $(E*d)*d$ is isomorphic to $E$, so has conductor
$N$ again.

Twisting commutes with isogenies, in the sense that if two curves $E$,
$F$ are $l$-isogenous then so are their twists $E*d$, $F*d$.  If $E$
has no complex multiplication (see below), then the structure of the
isogeny class of $E$ is a function of $j(E)$ alone.

The trace of Frobenius of $E*d$ at a prime $p$ not dividing $N$ is
$\chi(p)a_p$, where $\chi$ is the quadratic character associated to
$\Q(\sqrt{d})$ and $a_p$ is the trace of Frobenius of $E$.  Thus if
$E$ is modular, attached to the newform $f$, then $E*d$ is also
modular and attached to the twisted form $f\otimes\chi$, in the
notation of Chapter 2.

When $j=0$ (or equivalently $c_4=0$), $E$ has an equation of the form
$y^2=x^3+k$ with $k\in\Z$ non-zero and free of sixth powers.  Such
curves have complex multiplication by $\Z[(1+\sqrt{-3})/2]$.  Two such
curves with parameters $k$, $k'$ are isomorphic over
$\Q(\root6\of{k/k'})$.

Similarly, when $j=1728$ (or equivalently $c_6=0$), $E$ has an
equation of the form $y^2=x^3+kx$ with $k\in\Z$ non-zero and free of
fourth powers.  Such curves have complex multiplication by
$\Z[\sqrt{-1}]$.  Two such curves with parameters $k$, $k'$ are
isomorphic over $\Q(\root4\of{k/k'})$.

\subhead Complex multiplication \endsubhead

Each of the 13 imaginary quadratic orders $\OO$ of class number~1 has
a rational value of $j(\OO)=j(\omega_1/\omega_2)$, where
$\OO=\Z\omega_1+\Z\omega_2$. Elliptic curves $E$ with $j(E)=j(\OO)$
have complex multiplication: their ring of endomorphisms defined over
$\C$ is isomorphic to $\OO$.  In all other cases the endomorphism ring
of an elliptic curve defined over $\Q$ is isomorphic to $\Z$, since an
elliptic curve with complex multiplication by an order of class number
$h>1$ has a $j$-invariant which is not rational, but algebraic of
degree $h$ over $\Q$.

We give here a table of triples $(D,j,N)$ where $j=j(\OO)$ for an order 
of discriminant $D$, and $N$ is the smallest conductor of an elliptic 
curve defined over $\Q$ with this $j$-invariant.  All but the last three 
values ($D=-43$, $-67$, $-163$) have $N<1000$ and so occur in the tables.

\bigskip
\vbox{\offinterlineskip
\halign to \hsize{\tabskip 0pt
\strut#&\vrule#\tabskip 0pt plus 1fil&\hfil$#$\hfil&\vrule#&%
\hfil$#$\hfil&\vrule#&\hfil$#$\hfil&\vrule#&\hfil$#$\hfil&\vrule#&\hfil$#$\hfil&\vrule#&\hfil$#$\hfil&\vrule#&\hfil$#$\hfil&\vrule#&\hfil$#$\hfil&\vrule#&%
\hfil$#$\hfil&\vrule#&\hfil$#$\hfil&\vrule#&\hfil$#$\hfil&\vrule#&\hfil$#$\hfil&\vrule#&\hfil$#$\hfil&\vrule#&\hfil$#$\hfil&%
\tabskip0pt\vrule#\cr
\noalign{\hrule}
\omit&height2pt&&&&&&&&&&&&&&&&&&&&&&&&&&&&\cr
&&D&&-4&&-16&&-8&&-3&&-12&&-27&&-7&&-28&&-11&&-19&&-43&&-67&&-163&\cr
\omit&height2pt&&&&&&&&&&&&&&&&&&&&&&&&&&&&\cr
\noalign{\hrule}
\omit&height2pt&&&&&&&&&&&&&&&&&&&&&&&&&&&&\cr
&&j&&12^3&&66^3&&20^3&&0&&2\cdot30^3&&-3\cdot160^3&&-15^3&&255^3&&-32^3&&-96^3&&-960^3&&-5280^3&&-640320^3&\cr
\omit&height2pt&&&&&&&&&&&&&&&&&&&&&&&&&&&&\cr
\noalign{\hrule}
\omit&height2pt&&&&&&&&&&&&&&&&&&&&&&&&&&&&\cr
&&N&&32&&32&&256&&27&&36&&27&&49&&49&&121&&361&&43^2&&67^2&&163^2&\cr
\omit&height2pt&&&&&&&&&&&&&&&&&&&&&&&&&&&&\cr
\noalign{\hrule}
}
}
\bigskip

If $E$ has complex multiplication by the order $\OO$ of discriminant $D$, 
then the twist $E*D$ is isogenous to $E$, though not usually isomorphic 
to $E$ (over $\Q$).  Indeed, the only cases where $E$ is isomorphic to 
$E*D$ are $D=-4$ and $D=-16$ with $j(E)=1728$: the curves $y^2=x^3+16kx$ 
and $y^2+256kx$ are twists of, and isomorphic to, $y^2=x^3+kx$.  Since 
curves are isogenous if and only if they have the same $L$-series 
by Faltings's Theorem (see \cite{\Faltings}),  this implies that $E$ has 
complex multiplication if and only if $a_p=\chi(p)a_p$ for all primes 
$p$, where $\chi$ is the quadratic character as above.  Thus $a_p=0$ for 
half the primes $p$, namely those for which $\chi(p)=-1$.  This gives an 
alternative way of recognizing a curve with complex multiplication, from 
its traces of Frobenius. This is particularly convenient in the case of 
modular curves, where we compute the $a_p$ first, and will always know 
when a newform $f$, and hence the associated curve $E_f$, has complex 
multiplication.  For, in such a case, we must have $D^2\div N$ and 
$f=f\otimes\chi$, which we may easily check from the tables.


\enddocument
