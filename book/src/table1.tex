%
%  TABLE 1
%
\input book.def
\advance\pageno by\chaponepages
\advance\pageno by\chaptwopages
\advance\pageno by\appendixpages
\advance\pageno by\chapthreepages
\advance\pageno by\chapfourpages
%
\topmatter
\title\chapter\nofrills{Table 1} Elliptic curves \endtitle
\endtopmatter

\document
%\openup2pt

The table is arranged in blocks by conductor.  Each conductor is given in
factorized form at the top of its block (repeated, if necessary, on 
continuation pages), together with the number of isogeny classes of 
curves with that conductor.   Each block is subdivided into isogeny 
classes by a row of dashes.

The columns of the table give the following data for each curve $E$:
\roster

\item an identifying letter (A, B, C, \dots) for each isogeny class of curves 
      with the same conductor, choosing consecutive letters for the
      curves in the order in which they were computed.  Within each
      isogeny class we also number the curves in that class, with
      curve~1 being the ``strong Weil curve''.\footnote{For class 990H
      the ``strong'' curve is 990H3 and not 990H1.}  For ease of
      reference, when $N\le200$ we also give the identifying letter of
      each curve as given in Table~1 of \cite{\Antwerp}.

\item The integer coefficients $a_1$,~$a_2$,~$a_3$,~$a_4$ and~$a_6$ of a 
minimal equation for $E$.

\item The rank $r$ of $E(\Q)$. 

\item The order $|T|$ of the torsion subgroup $T$ of $E(\Q)$. 

\item The sign of the discriminant $\Delta$  of $E$, and its 
factorization. 

\item The prime factorization of the denominator of $j(E)$. 

\item The local indices $c_p$ for the primes of bad reduction. 

\item The Kodaira symbols for $E$ at each prime of bad reduction. 

\item The curves isogenous to $E$ via an isogeny of prime degree, with  the
degree $l$ in bold face.  For example, the entry ``{\bf2}: 3; {\bf3}:  2, 6''
for curve 448C4 indicates it is 2-isogenous to 448C3 and  3-isogenous to both
448C2 and 448C6.   From these entries it is easy to  draw isogeny diagrams for
each isogeny class in the manner of the Antwerp tables  \cite{\Antwerp}. We
regret that we could not persuade Birch to draw  little diagrams for us in this
column, as he did for  \cite\Antwerp.

\endroster

For convenience, we give the factorization of $N$ at the head of each 
section of the table.  This order of the `bad' prime factors 
$p_1,\ldots,p_k$ of $N$ is used within the table itself.  We give the 
discriminant $\Delta=\pm p_1^{e_1}\ldots p_k^{e_k}$ in factorized form as 
$\pm,e_1,\ldots,e_k$ in the columns headed $s$, $\hbox{ord}(\Delta)$.  
The column headed $\hbox{ord}_-(j)$ contains the exponents of these same 
primes in the denominator of the $j$-invariant, as in \cite\Antwerp.  
Finally the local factors $c_p$, and then the Kodaira symbols, are given 
for each of these primes in order.

\vfill
\eject
\input table1.def
\input table1.dat

\enddocument
