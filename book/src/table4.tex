%
%  TABLE 4
%
\input book.def
\advance\pageno by\chaponepages
\advance\pageno by\chaptwopages
\advance\pageno by\appendixpages
\advance\pageno by\chapthreepages
\advance\pageno by\chapfourpages
\advance\pageno by\tabonepages
\advance\pageno by\tabtwopages
\advance\pageno by\tabthreepages
%
\topmatter
\title\chapter\nofrills{Table 4} Birch--Swinnerton-Dyer data \endtitle
\endtopmatter
\document
%\openup2pt
%\raggedbottom

In Table~4 we give the numbers relating to the \BSD\ conjecture, for
each ``strong Weil'' curve $E_f$.  Each curve is identified as before
with a single letter X after the conductor, and is
\footnote{As remarked earlier, for class 990H the ``strong'' curve is
990H3 and not 990H1, and so the data here is for the curve 990H3.}  
the curve $N$X1 of Table~1.

For each curve we first give the rank $r$ , and then list (to 10
decimal places): the real period $\RP(f)$, the value of
$L^{(r)}(f,1)/r!$, the regulator $R$ of $E_f$, and the ratio
$L^{(r)}(f,1)/r!R\RP$.  For curves of rank~0 we obviously have $R=1$
exactly; also the ratio $L(f,1)/\RP(f)$ is known exactly in these
cases, so we give it as an exact rational rather than as a decimal.
Finally we give the value of
$$
   S = \frac {L^{(r)}(f,1)}  {r!\;\RP(f)} \left/ 
       \frac {(\prod c_p)\; R} {\left|T\right|^2}  \right. ,
$$ 
which according to the \BSD\ conjecture should equal the order of the
\TS\ group $\Sha(E_f/\Q)$.  The value $S$ is known exactly in case $r=0$ and
approximately in case $r>0$; in each case it is a positive integer to
within 10 decimal places, and is (to this precision) equal to 1 in all
but 4 cases.  The exceptions are
$$ \align
    S=4        &\qquad\text{for 571A, 960D, and 960N;}\\
    S=9        &\qquad\text{for 681B.}
   \endalign
$$
These are all curves of rank~0, so that the value of $S$ is exact.

We have also computed the corresponding data for all the curves in
each isogeny class, but this extra data is not included here for
reasons of space.  For a summary of the results obtained, see
\cite\JCsha.  As well as $S=1$, 4 and~9, the values $S=16$, 25
and~49 were also obtained, all for curves of rank~0.

\vfill\eject
\catcode`\@=11
\def\rightheadline{\hfill TABLE 4: BIRCH--SWINNERTON-DYER DATA
\firstc--\lastc\hfill \llap{\folio}} 
\def\leftheadline{\rlap{\folio} \hfill TABLE 4: BIRCH--SWINNERTON-DYER DATA \firstc--\lastc
\hfill}
\headline={\def\chapter#1{}%\chapterno@. }%
  \def\\{\unskip\space\ignorespaces}\headlinefont@
  \ifodd\pageno \rightheadline \else \leftheadline\fi}
\catcode`\@=\active
\vsize=10.7truein
%
% MACRO DEFINITIONS
%
\def\gap#1{\omit&height #1pt &&&&&&&&&&&&&&&&\cr}
\def\fivegap{\gap3} % gap between groups of  lines
\def\topline{\noalign{\hrule}\gap3}
\def\hline{\gap3\noalign{\hrule}\gap3}
\def\botline{\gap3\noalign{\hrule}}
%
\def\headings{&&\multispan3Curve&&\hfill r&&\hfill \RP&&%
              \hfill L^{(r)}(1)/r!&&\hfill R&&\hfill L^{(r)}(1)/r!\RP R&&%
              \hfill S&\cr}
\def\begintable#1\endtable#2#3{
% \line{}\vfill
  \vbox{\offinterlineskip
  \halign to \hsize{
  ##\mathstrut&\tabskip0pt\vrule##\tabskip 5pt plus 1fil &\hfill##\tabskip1pt&%
  \hfill##\hfill&\hfill##\hfill\tabskip 5pt plus 1fil &%
  \vrule##&$##$\hfill&\vrule##&$##$\hfill&\vrule##&$##$\hfill&%
  \vrule##&$##$\hfill&\vrule##&$##$\hfill&\vrule##&$##$\hfill&%
  \tabskip0pt\vrule##\cr
  \topline
  \headings
  \hline
  #1
  \botline
  }}
  \vfill\line{}
  \def\firstc{#2}\def\lastc{#3}
  \eject
}


\input table4.dat

\enddocument
