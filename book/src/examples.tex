% ALGORITHMS FOR MODULAR ELLIPTIC CURVES       Last change: 10/96
%
% EXAMPLES SECTION (BETWEEN CHAPTERS 2 AND 3)
%
% Use AmSTeX 2.0
%
\input book.def
\advance\pageno by\chaponepages
\advance\pageno by\chaptwopages
%
% Equation numbers from Chapter 2: need to be checked by hand!
\def\mtwoterm{2.2.6}
\def\mthreeterm{2.2.7}
\def\dotseriesoneb{2.10.5}
\def\dotseriestwob{2.10.8}
%
%
\topmatter
%
\title\chapter\nofrills{Appendix to Chapter II} Examples \endtitle
%
\endtopmatter
%
%The following is to get round a bug in the AmSTeX running head macros
%sections not called "Chapter":
%
\leftheadtext{Appendix to Chapter II: Examples}  
% Rightheadtext will pick up section title

%
\document
%\openup 2pt 
%\raggedbottom % for drafts only

\def\i{\infty}

We give here some worked examples of the methods described in the preceding
chapter, to illustrate and clarify the different situations which arise.
The first example is $N=11$, which is the first non-trivial level; here
we give most detail.  Then we consider $N=33$, where we encounter oldforms
and more complicated M-symbols, and $N=37$, where there are two newforms,
one of which has $L(f,1)=0$, necessitating a different method of computing
Hecke eigenvalues.  Finally we look at a square level, $N=49$, to illustrate
the direct method of computing periods.

\head Example 1: $N=11$ \endhead

For simplicity we will only work in $H(11)$, rather than the smaller
quotient space $H^+(11)$.  The M-symbols for $N=11$ are $(c:1)$ for
$c$ modulo~11 and $(1:0)$, which we abbreviate as $(c)$ and $(\i)$
respectively, with $|c|\le5$.  (Similarly with other prime levels).
The 2-term and 3-term relations (\mtwoterm) and (\mthreeterm) are as
follows.
$$
   \eqalign {(0)+(\i)&=0\cr
          (1)+(-1)&=0\cr
          (2)+(5)&=0\cr
          (-2)+(-5)&=0\cr
          (3)+(-4)&=0\cr
          (-3)+(4)&=0\cr } 
\qquad\qquad
  \eqalign {(0)+(\i)+(-1)&=0\cr
          (1)+(-2)+(5)&=0\cr
          (2)+(4)+(-4)&=0\cr
          (3)+(-5)+(-3)&=0\cr }
$$
Solving these equations we can express all 12 symbols in terms of
$A=(2)$, $B=(3)$ and $C=(0)$:
$$
  \eqalign{
          (0)&=C\cr
          (\i)&=-C\cr
          (1)=(-1)&=0\cr
          (2)=(-2)&=A\cr
          (5)=(-5)&=-A\cr
  }
  \qquad
  \eqalign{ 
          (3)&=B\cr
          (-3)&=A-B\cr
          (4)&=B-A\cr  
          (-4)&=-B\cr 
  }
$$
There are two classes of cusps, $[0]$ and $[\i]$, with $[a/b]=[0]$ 
if $11\ndiv b$ and $[a/b]=[\i]$ if $11\div b$.  
Hence $\delta((c))=\delta(\left\{0,1/c\right\})=[1/c]-[0]=0$ 
for $c\not\equiv0$.  It follows that
$$
  H(11) = \ker(\delta) = \left<A,B\right>,
$$
with $2g=\dim H(11)=2$, so that the genus is 1.  There is therefore one 
newform $f$. This makes the rest of the calculation simpler, as 
we do not have to find and split off eigenspaces.

The conjugation $*$ involution maps $(c)\mapsto(-c)$, so $A^*=A$ and
$B^*=A-B$.  This has matrix $\mat(1,1;0,-1)$ with respect to the basis
$A$, $B$.  The $+1$-{} and $-1$-eigenspaces are generated by $A$ and
$A-2B$ respectively, and we have left eigenvectors $v^+=(2,1)$ and
$v^-=(0,1)$.  Thus the period lattice is Type~1 (non-rectangular), and
$\RP(f)=\rp(f)=\<A,f>$.

If we had worked in $H^+(11)$, viewed as the quotient 
$H(11)/H^-(11)$, by including relations $(c)=(-c)$, the effect 
would be to identify $(c)$ and $(-c)$.  This gives a 
1-dimensional space generated by $\overline{B}$ with 
$\overline{A}=2\overline{B}$, where the bars denote the 
projections to the quotient.  Notice that although $\overline{B}$ 
is a generator here, the integral of $f$ over $B$ is not a real 
period; its real part is half the real period.  However we do 
still have $\RP(f)=\<B+B^*,f>=2\hbox{\rm Re}\<B,f>$, so we could 
compute $\RP(f)$ in this context without actually knowing whether 
it was 1 or 2 times the smallest real period.

To compute Hecke eigenvalues we may work in the subspace
$\left<A\right>$; since this subspace is conjugation invariant (being
the $+1$-eigenspace) we will have $T_p(A)=a_pA$ for all $p\not=11$.
We first compute $T_2$ explicitly.  The first method, converting the
M-symbol $A=(2:1)$ to the modular symbol $\left\{0,1/2\right\}$,
gives:
$$\align
T_2(A)=T_2\left(\left\{0,\frac12\right\}\right)
     &=\left\{0,1\right\}+\left\{0,\frac14\right\}
                         +\left\{\frac12,\frac34\right\}\\
     &=\left\{0,1\right\}+\left\{0,\frac14\right\}+\left\{\frac12,1\right\}
                                                  +\left\{1,\frac34\right\}\\
     &=(1:1) + (4:1) + (1:2) + (-4:1)\\
     &=(1) + (4) + (-5) + (-4)\\
     &=0 + (B-A) + (-A) + (-B) = -2A,\\
\endalign
$$
so that $a_2=-2$.  Alternatively, using the Heilbronn matrices from
Section~2.4, we compute:
$$\align
T_2(A) = T_2((2:1)) &= (2:1)R_2 \\
&=(2:1)\mat(1,0;0,2)+(2:1)\mat(2,0;0,1)+(2:1)\mat(2,1;0,1)+(2:1)\mat(1,0;1,2)\\
&=(2:2)+(4:1)+(4:3)+(3:2)\\
&=(1)+(4)+(5)+(-4)\\
&=0+(B-A)+(-A)+(-B)\\
&=-2A.\\
\endalign
$$
Now
$(1+2-a_2)L(f,1)=\left<\left\{0,1/2\right\},f\right>=\left<A,f\right>=\RP(f)$,
giving
$$
  \frac{L(f,1)}{\RP(f)} = \frac15.
$$

For all primes $p\not=11$ we will evaluate 
$\mu_p=\sum_{a=0}^{p-1}\left\{0,a/p\right\}=n_pA$ for a certain integer 
$n_p$, since then also $1/5=n_p/(1+p-a_p)$, giving
$$
   a_p = 1+p-5n_p.
$$
At this stage we already know that the corresponding elliptic 
curve has rank 0, and that $1+p-a_p\equiv0\pmod5$ for all 
$p\not=11$, so that it will possess a rational 5-isogeny.

To save time, we can use the fact that $\left\{0,a/p\right\}^* = 
\left\{0,-a/p\right\}$;
thus for odd $p$ we need only evaluate half the sum, say
$$
  \mu_p'=\sum_{a=1}^{(p-1)/2}\left\{0,\frac{a}{p}\right\},
$$
and then set $\mu_p=\mu_p'+(\mu_p')^*$.

For $p=3$, we have $\mu_3'=\left\{0,1/3\right\}=(3:1)=(3)=B$, 
so $\mu_3=B+B^*=A$, giving $n_3=1$ and $a_3=1+3-5n_3=-1$.

For $p=5$ we compute:
$$\align
  \left\{0,\frac15\right\}&=(5:1)=(5)=-A;\\
  \left\{0,\frac25\right\}&=\left\{0,\frac12\right\}+\left\{\frac12,\frac25\right\} \\
               &=(2:1)+(-5:2)=(2)+(3)=A+B;\\
 \mu_5'&=(-A)+(A+B)=B;\\
 \mu_5&=B+B^*=A, \qquad\text{so that $n_5=1$;}\\
 a_5&=1+5-5n_5=1.\\
\endalign
$$

Similarly, with $p=7$ we have $n_7=2$, so that $a_7=1+7-5n_7=-2$, and 
with $p=13$ we have $n_{13}=2$ so that $a_{13}=4$.

These computations can also be carried out using Heilbronn matrices,
by applying the Hecke operators directly to $C=(0:1)=\{0,\infty\}$,
as follows.  Once we know that $a_2=-2$, we have
$$
 -2C=T_2(C)=T_2((0:1))=(0:1)R_2=(0:2)+(0:1)+(0:1)+(1:2)=3C-A,
$$
giving $C=\frac15A$ in agreement with the ratio $L(f,1)/\RP(f)$ found
earlier.  Similarly, using the Heilbronn matrices $R_3$ listed in
Section~2.4, we find
$$\align
  a_3C&=T_3(C)=(0:1)R_3\\
&=(0:3)+(0:1)+(1:3)+(0:1)+(0:1)+(1:-3)\\
&=C+C+(B-A)+C+C+(-B)\\
&=4C-A=-C,\\
\endalign
$$
giving $a_3=-1$ again.

For the prime $q=11$ we compute the involution $W_{11}$ induced by the
action of the matrix $\mat(0,-1;11,0)$:
$$\align
 W_{11}(A)=\mat(0,-1;11,0)\left\{0,\frac12\right\} &=\left\{\i,\frac{-2}{11}\right\}\\
          &=\left\{\i,0\right\}+\left\{0,\frac{-1}{5}\right\}+\left\{\frac{-1}{5},\frac{-2}{11}\right\}\\
          &=(1:0)+(-5:1)+(11:5)\\
          &=(\i)+(-5)+(0)\\
          &=-A,\\
\endalign
$$
so that the eigenvalue $\eps_{11}$ of $W_{11}$ is $-1$.  In fact, 
this was implicit earlier, since $L(f,1)\not=0$ implies that the 
sign of the functional equation is $+1$, which is minus the 
eigenvalue of the Fricke involution $W_{11}$.

The Fourier coefficients $a(n)=a(n,f)$ for $1\le n\le16$ are now given
by the following table.

\medskip
\def\gap{\omit&height2pt&&&&&&&&&&&&&&&&&&&\cr}
\def\m{\phantom{-}}
{\offinterlineskip
\centerline{
\vbox{
\halign to \hsize{
\strut#&\vrule#\tabskip5pt plus1fil&          % left border
\hfil$#$&\hfil#\vrule&                          % left  col
\hfil$#$&\hfil$#$&\hfil$#$&\hfil$#$&                        % 
\hfil$#$&\hfil$#$&\hfil$#$&\hfil$#$&                        % 
\hfil$#$&\hfil$#$&\hfil$#$&\hfil$#$&                        % 
\hfil$#$&\hfil$#$&\hfil$#$&\hfil$#$&                        % 
\vrule\tabskip0pt#\cr           % 
\noalign{\hrule}
\gap
&&n&&1&2&3&4&5&6&7&8&9&10&11&12&13&14&15&16&\cr
\gap
\noalign{\hrule}
\gap
&&a(n)&&\m1&-2&-1&\m2&\m1&\m2&-2&\m0&-2&-2&\m1&-2&\m4&\m4&-1&-4&\cr
\gap
\noalign{\hrule}
}}}}
\smallskip
\noindent
Here we have used multiplicativity, and:
$$\align
a(11)&=-\eps_{11}=+1;\\
a(4)&=a(2)^2-2a(1)=2;\\
a(8)&=a(2)a(4)-2a(2)=0;\\
a(16)&=a(2)a(8)-2a(4)=-4;\\
a(9)&=a(3)^2-3a(1)=-2.\\
\endalign
$$

We know that the period lattice $\Lambda_f$ has a $\Z$-basis of the
form $[\omega_1,\omega_2]=[2x,x+iy]$, where $\omega_1=P_f(A)$ and
$\omega_2=P_f(B)$.  We can compute the real period
$\omega_1=\RP(f)=5L(f,1)$ by computing $L(f,1)$:
$$
   L(f,1)=2\sum_{n=1}^{\infty}\frac{a(n)}{n}t^n
$$
where $t=\exp(-2\pi/\sqrt{11})=0.15\ldots$.  Using the first 16 terms
which we have, already gives this to 13 decimal places:
$$
  L(f,1) = 0.2538418608559\ldots;
$$
thus 
$$
  \omega_1 = \RP(f) = 1.269209304279\ldots.
$$

For the imaginary period $y$ we twist with a prime $l\equiv3\pmod4$.  Here
$l=3$ will do, since
$$
  \gamma_3 = \left\{0,\frac13\right\}-\left\{0,\frac{-1}{3}\right\} 
           = (3)-(-3) = -A+2B \not=0.
$$
To project onto the minus eigenspace we take the dot product of this
cycle (expressed as a row vector $(-1,2)$) with $v^-=(0,1)$ to get
$m^-(3)=2$.  Hence
$$
  y=\frac{1}{2i}P(3,f)=\frac{\sqrt{3}}{2}L(f\otimes3,1).
$$
Summing the series for $L(f\otimes3,1)$ to 16 terms gives only 4 decimals:
$$
  L(f\otimes3,1) = 1.6845\ldots.
$$
This is less accurate than $L(f,1)$ since this series is a power series in 
$\exp(-2\pi/3\sqrt{11})=0.53\ldots$, compared with $0.15\ldots$.
Hence $y=1.4588\ldots$, so that
$$
  \omega_2 = 0.634604652139\ldots+1.4588\ldots i.
$$

So far we have only used the Hecke eigenvalues $a_p$ for $p\le13$, and only 
16 terms of each series.   If we use these approximate values for the period 
lattice generators $\omega_1$ and $\omega_2$ we already find the approximate 
values $c_4=495.99$ and $c_6=20008.09$ which round to the integer values
$c_4=496$ and $c_6=20008$.  Taking the first 25 $a_p$ and the first 100
terms of the series gives
$$
  c_4 = 495.9999999999954\ldots \qquad\text{and}\qquad
  c_6 = 20008.0000000085.
$$
The exact values $c_4=496$ and $c_6=20008$ are the invariants of an elliptic
curve of conductor~11, which is in fact the modular curve $E_f$:
$$
  y^2+y=x^3-x^2-10x-20.
$$
This is the first curve in the tables, with code 11A1 (or Antwerp code 11B).
The value ${L(f,1)}/{\RP(f)} = 1/5$  agrees with the value predicted by the
\BSD\ conjecture for ${L(E_f,1)}/{\RP(E_f)}$ , provided that $E_f$ has trivial 
\TS\ group.

\medskip

We now illustrate the shortcut method presented in Section~2.11, where
we guess the imaginary period and lattice type without computing
$H(11)$.  Having computed $P(3,f)=2.9176\ldots i$ which is certainly
non-zero, we consider the lattices $\<x,yi>$ and $\<2x,x+yi>$, where
$2x=1.2692\ldots$ (from above) and $yi=P(3,f)/m^-$, for
$m^-=1,2,3,\ldots$.  With $m^-=1$ we do not find integral invariants,
but for $m^-=2$ and lattice type~1 we find the curve
$E_f=[0,-1,1,-10,-20]$ given above\footnote{Here,
$[a_1,a_2,a_3,a_4,a_6]$ denote the Weierstrass coefficients of the
curve; see Chapter 3.}.

Using the first variant of the method, where we do not even know $x$,
we can take $l^+=5$ since $P(5,f)=6.346\ldots\not=0$.  The correct
value of $m^+$ here is 10; if we do not know this, but try
$m^+=1,2,3\ldots$ in a double loop with $m^-$, the first valid lattice
we come across is with $(m^+,m^-)=(2,2)$ and type~1, which leads to
the curve $E'=[0,-1,1,0,0]$, also of conductor~11; this is 5-isogenous
to the ``correct'' curve $E_f$, which comes from $(m^+,m^-)=(10,2)$
and type~1.

We may also consider the ratios $P(l,f)/P(3,f)$ for other primes
$l\equiv3\pmod4$; we restrict to those $l$ satisfying
$\left(\frac{-11}{l}\right)=\left(\frac{l}{11}\right)=+1$, since
otherwise $P(l,f)$ is trivially 0 (since the sign of the functional
equation for the corresponding $L(\fchi,s)$ is then $-1$).  We find
the following table of values (rounded: they are only computed
approximately):
\medskip
\def\gap{\omit&height2pt&&&&&&&&&&&&&&&&&\cr}
\def\m{\phantom{-}}
{\offinterlineskip
\centerline{
\vbox{
\halign to\hsize{
\strut#&\vrule#\tabskip5pt plus1fil&          % left border
\hfil$#$\hfil&#\vrule&                          % left  col
\hfil$#$&\hfil$#$&\hfil$#$&\hfil$#$&                        % 
\hfil$#$&\hfil$#$&\hfil$#$&\hfil$#$&                        % 
\hfil$#$&\hfil$#$&\hfil$#$&\hfil$#$&                        % 
\hfil$#$&\hfil$#$&                     % 
\vrule\tabskip0pt#\cr           % 
\noalign{\hrule}
\gap
&& l     & &3&23&31&47&59&67&71&103&163&179&191&199&223&251&\cr
\gap
\noalign{\hrule}
\gap
&&\dfrac{P(l,f)}{P(3,f)} & &1&1&1&0&1&9&1&0&4&25&1&4&1&1&\cr
\gap
\noalign{\hrule}
}}}}

\smallskip
\noindent
The zero values for $l=47$ and $l=103$ indicate that the corresponding
twists of the newform $f$ have positive even analytic rank (one can
check that the corresponding twists of the curve $E_f$ do indeed have
rank~2).  As all these values are integral here ({\it a priori\/} they
are only known to be rational) we do not find any nontrivial divisor
of $m^-$ (which we know in fact equals~2).  The fact that all the
integers are perfect squares is an amusing observation, but has a
simple explanation in terms of the numbers appearing in the \BSD\
conjecture for the twists of $E_f$.

There is one other curve $E''$ isogenous to $E_f$ in addition
to $E'$ (found above).  If the period lattice of
$E_f=[0,-1,1,-10,-20]$ is $\left<2x,yi\right>$ with $x=0.6346...$ and
$y=1.4588...$, then $E'=[0,-1,1,0,0]$ has period lattice
$\left<10x,5x+yi\right>$, and $E''=[0,-1,1,-7820,263580]$ has lattice
$\left<x/5,2x/5+yi\right>$.  These curves are linked by 5-isogenies
$E_f\leftrightarrow E'$ and $E_f\leftrightarrow E''$.

\medskip

Finally, we compute the degree of the modular parametrization
$\varphi\colon X_0(11)\to E_f$.  Of course, this is obviously~1, since
the modular curve $X_0(11)$ has genus~1, so that $\varphi$ is the
identity map in this case; but this example will serve to illustrate
the general method.

The twelve M-symbols form 4 triangles which we choose as follows:
$$
\align
  (1,0), (-1,1), (0,1); &\qquad (1,1), (-2,1), (-1,2);\cr
  (1,2), (-3,1), (-2,3); &\qquad (1,3), (-4,1), (-3,4).\cr
\endalign
$$
There are two $\t$-orbits, corresponding to the two cusps at $\infty$
(of width~1) and at 0 (of width~11).  The first contributes nothing.  The
second is as follows:
$$\multline
  (1,0) \mapsto (1,1) \mapsto (1,2) \mapsto (1,3) \mapsto (1,4) \equiv
(-2,3) \mapsto (-2,1) \mapsto (-2,-1) \\ \equiv (-3,4) \mapsto (-3,1)
\mapsto (-3,-2) \equiv (-4,1) \mapsto (-4,-3) \equiv (-1,2) \mapsto
(-1,1) \mapsto (1,0).  
\endmultline
$$ 
There are four jump matrices coming from the above sequence. From
$(1,4)\equiv(-2,3)$ we obtain 
$$
 \delta_1 = \mat(0,-1;1,4)\mat(1,-1;-2,3)^{-1} = \mat(-2,-1;11,5); 
$$  
the others are $\delta_2=\mat(4,1;11,3)$, $\delta_3=\mat(-5,-1;11,2)$
and $\delta_4=\mat(-3,1;11,-4)$.  Using modular symbols, we can
compute the coefficients of $P_f(\delta_i)$ with respect to the period
basis $\w_1$, $\w_2$, to obtain $P_f(\delta_1)=-\w_1$,
$P_f(\delta_2)=\w_2$, $P_f(\delta_3)= \w_1$,
and~$P_f(\delta_4)=-\w_2$.  Hence
$$ \align
   \deg(\varphi) &= \frac{1}{2} \left( \detmat01{-1}0 + \detmat00{-1}1 +
\detmat0{-1}{-1}0 + \detmat1001 + \detmat1{-1}00 + \detmat0{-1}10
\right)\cr
              &= \frac12(1+0-1+1+0+1) = 1,\cr
\endalign
$$
as expected.

\head Example 2: $N=33$ \endhead

Since $33=3\cdot11$, the number of M-symbols is $48=4\cdot12$, consisting of
33 symbols $(c)=(c:1)$, 13 symbols $(1:d)$ with $\gcd(d,33)>1$, and the symbols
$(3:11)$ and $(11:3)$.  (In fact, whenever $N$ is a product  $pq$ of 2 
distinct primes, the M-symbols have this form, with exactly two symbols,
$(p:q)$ and $(q:p)$ not of the form $(c:1)$ or $(1:d)$).

There are four cusp classes represented by $0$, $1/3$, $1/11$ and $\i$, 
with the class of a cusp $a/b$ being determined by $\gcd(b,33)$.  
(Similarly, whenever $N$ is square-free, the cusp classes are in one-one 
correspondence with the divisors of $N$).

Using the two-term and three-term relations, and including the relations
$(c:d)=(-c:d)$, we can express all the M-symbols in terms of six of them,
and $\ker(\delta^+)=\left<(7),(2),(15)-(9)\right>$.  Hence
$H^+(33)$ is three-dimensional.  We know there will be a two-dimensional
oldclass coming from the newform at level $11$, so there will also be a 
single newform $f$ at this level.

If we compute the images of the basis modular symbols $\{0,1/7\}$,
$\{0,1/2\}$ and $\{1/9,1/15\}$ under $T_2$ and $W_{33}$, we find that 
they have matrices
$$
  T_2=\pmatrix-2&0&0\\0&1&2\\0&0&-2\endpmatrix \qquad\text{and}\qquad
  W_{33}=\pmatrix1&0&0\\0&-1&0\\1&0&-1\endpmatrix.
$$
$T_2$ has a double eigenvalue of $-2$, coming from the oldforms, which we 
ignore, and also the new eigenvalue $a_2=1$ with left eigenvector $v=(0,1,0)$.
The corresponding eigenvalue for $W_{33}$ is $\eps_{33}=-1$.  Hence the
sign of the functional equation is $+$, and the analytic rank is even.
Moreover since the eigencycle for $a_2$ is the second basis element,
which is $\{0,1/2\}=\mu_2$, we have $2(1+2-a_2)L(f,1)=\RP(f)$, so
that
$$
  \frac{L(f,1)}{\RP(f)} = \frac14.
$$
In particular, $L(f,1)\not=0$, so that the analytic rank is 0.  Note
that because we have factored out the pure imaginary component, we do
not usually know at this stage whether the least real period $\rp(f)$
is equal to $\RP(f)$ or half this; all we can say is that $\RP(f)/2 =
\hbox{Re}\left<\{0,1/2\},f\right>$ is the least real part of a period
(up to sign).  But in this case, $\{0,1/2\}$ is certainly an integral
cycle, and since $\left<\{0,1/2\},f\right>$ is real, we can in fact
deduce already that the period lattice is of type~2 (rectangular) with
$\RP(f)=2\rp(f)$.

To compute more $a_p$ we express each cycle $\mu_p$ as a linear
combination of the basis and project to the eigenspace by taking the
dot product with the left eigenvector $v$, which just amounts in this
case to taking the second component.  In this way we find $a_5=-2$,
$a_7=4$, $a_{13}=-2$, and so on.  For the involutions $W_3$ and
$W_{11}$ we can either compute their $3\times 3$ matrices or just
apply them directly to the eigencycle $\{0,1/2\}$, and we find that
$\eps_3=+1$ and $\eps_{11}=-1$.  In fact we already knew that the
product of these was $\eps_{33}=-1$, so we need not have computed
$\eps_{11}$ directly, though doing so serves as a check.

Now we go back and compute the full space $H(33)$, which is 
six-dimensional, with basis
$$
  \left\{0,\frac{1}{7}\right\},
  \left\{0,\frac{1}{4}\right\},
  \left\{0,\frac{-1}{4}\right\},
  \left\{\frac{1}{12},\frac{-1}{6}\right\},
  \left\{\frac{1}{12},\frac{-1}{3}\right\},
  \left\{0,\frac{1}{10}\right\}.
$$
By computing the $6\times6$ matrices of conjugation and $T_2$, we may 
pick out the left eigenvectors
$$
  v^+=(0,1,-1,1,2,0) \qquad\text{and}\qquad
  v^-=(-1,0,0,2,1,1).
$$
Since these vectors are independent modulo~2, it follows (as expected)
that the period lattice is type~2, with a $\Z$-basis of the form
$[\omega_1,\omega_2] = [x,yi]$.

Firstly, $x=\rp(f)=\RP(f)/2=2L(f,1)$. Summing the series for $L(f,1)$ we
obtain $L(f,1)=0.74734\ldots$, so that $\omega_1=x=1.49468\ldots$
and $\RP(f)=2x=2.98936\ldots$.  Then we use the twisting prime $l=7$:
the twisting cycle
$$
   \gamma_7 = \sum_{a=1}^{6}\left(\frac{a}{7}\right)\left\{0,\frac{a}{7}\right\}
$$
is evaluated in terms of our basis to be $(2,2,0,-2,0,0)$, whose dot product
with $v^-$ is $-6$.  Hence $y=\sqrt{7}L(f\otimes7,1)/6$.  The value of 
$L(f\otimes7,1)$ is determined by summing the series to be $3.11212\ldots$,
so that $y=1.37232\ldots$ and $\omega_2=1.37232\ldots i$.  If we evaluate
these from the first 100 terms of the series, using $a_p$ for $p<100$,
we find the approximate values $c_4=552.99999\ldots$ and 
$c_6=-4084.99947\ldots$.  These round to $c_4=553$ and $c_6=-4085$, which 
are the invariants of the curve 33A1: $y^2+xy=x^3+x^2-11x$.  Notice that
this curve has four rational points, which we could have predicted since
the ratio $L(f,1)/\RP(f)=1/4$ implies that $1+p-a_p\equiv0\pmod4$ for all
$p\not=2,3,11$.



\head Example 3: $N=37$ \endhead

Since 37 is prime the M-symbols are simple here, as for $N=11$.  We
find that $H^+(37)$ is two-dimensional, generated by $A=(8)$ and $B=(13)$.
With this basis the matrices of $T_2$ and $W_{37}$ are
$$
  T_2=\pmatrix-2&0\\0&0\endpmatrix \qquad\text{and}\qquad
  W_{37}=\pmatrix1&0\\0&-1\endpmatrix.
$$
Thus we have two one-dimensional eigenspaces, generated by $A$ and $B$
respectively, with eigenvalues $(a_2=-2, \eps_{37}=+1)$ for $A$ and
$(a_2=0, \eps_{37}=-1)$ for $B$.  The left eigenvectors are simply
$v_1=(1,0)$ and $v_2=(0,1)$.
Let us denote the corresponding newforms by $f$ and $g$ respectively.
Now $\{0,1/2\}=2B$, so
$$
  \frac{L(f,1)}{\RP(f)} = 0 \qquad\text{and}\qquad 
  \frac{L(g,1)}{\RP(g)} = \frac13.
$$
The fact that $\eps_{37}(f)=+1$ implies that $f$ has odd analytic
rank, while the previous line shows that $g$ has analytic rank~$0$..

To compute Hecke eigenvalues, the method we used previously would only work
for $g$, so instead we use the variation discussed in Section~2.9.
The cycle $\{1/5,\i\}$ projects non-trivially onto both eigenspaces.
In fact $(1+2-T_2)\{1/5,\i\}=-5A-B$, so the components in the two eigenspaces
are $(-5)/(1+2-(-2)) = -1$ and $(-1)/(1+2-0)=-1/3$.  Hence by computing
$(1+p-T_p)\{1/5,\i\}=n_1(p)A+n_2(p)B$ for other primes $p\not=37$, we
may deduce that
$$
  a(p,f)=1+p+n_1(p) \qquad\text{and}\qquad a(p,g)=1+p+3n_2(p).
$$
In this way we find that the first few Hecke eigenvalues are as follows:
\medskip
\def\gap{\omit&height2pt&&&&&&&&&&&&\cr}
\def\m{\phantom{-}}
{\offinterlineskip
\centerline{
\vbox{
\halign {
\strut#&\vrule#\tabskip15pt plus1fil&           % left border
\hfil$#$&\hfil#\vrule&                          % left  col
\hfil$#$&\hfil$#$&\hfil$#$&\hfil$#$&                        % 
\hfil$#$&\hfil$#$&\hfil$#$&\hfil$#$&\hfil$#$&                        % 
\vrule\tabskip0pt#\cr           % 
\noalign{\hrule}
\gap
&&p&&2&3&5&7&11&13&17&19&\ldots&\cr
\gap
\noalign{\hrule}
\gap
&&a(p,f)&&-2&-3&-2&-1&-5&-2&\m0&\m0&\ldots&\cr
\gap
&&a(p,g)&&\m0&\m1&\m0&-1&\m3&-4&\m6&\m2&\ldots&\cr
\gap
\noalign{\hrule}
}}}}
\smallskip
\noindent
%$$
%  \matrix   p   &:\quad&2&3&5&7&11&13&17&19&\ldots  \\
%          a(p,f)&:\quad&-2&-3&-2&-1&-5&-2&0&0&\ldots  \\
%          a(p,g)&:\quad&0&1&0&-1&3&-4&6&2&\ldots 
%  \endmatrix
%$$
Two things can be noticed here: the preponderance of negative values amongst
the first few $a(p,f)$ means that the curve $E_f$ has many points modulo $p$
for small $p$, which we might expect heuristically since we know that its
analytic rank is odd, and hence positive.  Secondly, since 
$1+p-a(p,g)\equiv0\pmod3$ for all $p\not=37$, we know that $E_g$ will have
a rational 3-isogeny.

Turning to the full space $H(37)$, we find that it has basis
$\left<(8),(16),(20),(28)\right>$.   Conjugation and $W_{37}$ have
matrices
$$
\pmatrix 0&-1&0&0\\-1&0&0&0\\0&1&0&1\\1&0&1&0 \endpmatrix
\qquad\text{and}\qquad
\pmatrix 0&-1&0&0\\-1&0&0&0\\1&0&0&-1\\0&-1&-1&0 \endpmatrix
$$
respectively.  

For the $A$ eigenspace corresponding to $f$ we find left
eigenvectors $v_1^+=(-1,1,0,0)$ and $v_1^-=(-1,0,-1,1)$.  These are independent
modulo~2, so the period lattice is rectangular, say $[x,yi]$.  To find $x$
we must twist by a real quadratic character, using a prime $l\equiv1\pmod4$.
Here $l=5$ will do: the twisting cycle is $\{0,1/5\} - \{0,2/5\} - 
\{0,3/5\} + \{0,4/5\} = (2,-2,0,2)$, whose dot product with $v_1^+$ is $-4$,
so that $x=\sqrt{5}L(f\otimes5,1)/4$.  For the imaginary period we use
$l=3$ with twisting cycle $(0,0,-1,1)$ and a dot product of 2 with $v_1^-$,
so that $y=\sqrt{3}L(f\otimes{3},1)/2$. Evaluating numerically,
using 100 terms of the series and $a_p$ for $p<100$, we find the values
$$\align
  L(f\otimes5,1) = 5.35486\ldots, \qquad&\text{so that}\qquad
  x = 2.99346\ldots;         \\
  L(f\otimes3,1) = 2.83062\ldots, \qquad&\text{so that}\qquad
  y = 2.45139\ldots;         \\
\endalign
$$
and finally,
$$\align
  c_4 &= 47.9999999996\ldots,  \\
  c_6 &= -216.000000004\ldots.  \\
\endalign
$$
The rounded values $c_4=48$ and $c_6=-216$ are those of the curve 37A1,
with equation $y^2+y=x^3-x$.  This curve does have rank~1.  We may also
check that the analytic rank is~1 by computing $L'(f,1)$ by summing the
series given in Section~2.13: we find that $L'(f,1)=0.306\ldots$,
which is certainly non-zero.

The $B$ eigenspace is handled similarly to the example at level~33.
We find $v_2^+=(0,1,1,1)$ and $v_2^-=(1,1,0,0)$.  The period lattice
is $[x,iy]$ with $x=3L(g,1)/2$ and $y=\sqrt{19}L(g\otimes{19},1)/4$.
The latter needs more terms to compute to sufficient accuracy, as 19 is
larger than the twisting primes we have previously used.  Using $p<100$
as before we find $c_4=1119.878\ldots$, which rounds to the correct (with
hindsight) value 1120, but for $c_6$ we get $36304.495$, and neither
36304 nor 36305 is correct.  Going back to compute $a_p$ for $100<p<200$
we reevaluate the series to 200 terms, and find
$$\align
  L(g,1) = 0.72568\ldots, \qquad&\text{so that}\qquad
  x = 1.08852\ldots;         \\
  L(g\otimes19,1) = 1.62207\ldots, \qquad&\text{so that}\qquad
  y = 1.76761\ldots;         \\
\intertext{and hence}
  c_4 = 1120.000008\ldots,&\qquad\text{and}\qquad  
  c_6 = 36295.99943\ldots.  \\
\endalign
$$
Now the rounded values $c_4=1120$ and $c_6=36296$ are the invariants of
the curve 37B1 with equation $y^2+y=x^3+x^2-23x-50$.  As expected, this curve 
does admit a rational 3-isogeny.

\head Example 4: $N=49$ \endhead

$H(49)$ is two-dimensional, with a basis consisting of the M-symbols 
$(11)$, $(2)$.  Hence there is a unique newform $f$ at this level, which 
must be its own $-7$-twist, or in other words have complex multiplication by $-7$.  The 
conjugation matrix with respect to this basis is $\mat(-1,0;-1,1)$, so we 
take $v^+=(1,-2)$ and $v^-=(1,0)$.  Hence the period lattice has the form
$[2x,x+yi]$ with $2x=\rp(f)=\RP(f)$.  Also $a_2=1$, so we have
$L(f,1)/\RP(f)=1/2$.  Hence we may compute the real period via $L(f,1)$
as before, and find $L(f,1)=0.96666\ldots$, so that $\RP(f)=1.9333\ldots$.
But the method we have used in the earlier examples to find the imaginary
period will not work here, since for every prime $l\equiv3\pmod4$, 
$l\not=7$, we have $L(f\otimes l,1)=0$, since $\chi(-49)=\chi(-1)=-1$ 
where $\chi$ is the associated quadratic character modulo $l$.

Instead, we compute periods directly, as in Section~2.10.  The
cycle $(5)=\{0,1/5\}$ is equal to $(11)+(2)$, from which it follows
that $\left<(5),f\right>=-x+yi$; the coefficients are the dot products
of the vector $(1,1)$ with $v^{\pm}$.  Now $\{0,1/5\}=\{0,M(0)\}$ with
$M=\mat(10,1;49,5)$.  Hence the simpler formula (\dotseriesoneb) gives
$$
P_f(M) =   \left<\left\{0,\frac15\right\},f\right> = -x+yi = 
\sum_{n=1}^{\i}\frac{a(n)}{n}e^{-2\pi n/49}\left(e^{2\pi inx_2}-e^{2\pi 
inx_1}\right)
$$
where $x_1=-5/49$ and $x_2=10/49$.  Summing the first 100 terms as 
before, we find the values
$$
  x = 0.96666\ldots \qquad\text{and}\qquad y = 2.557536\ldots.
$$
Of course, the value of $x$ merely confirms the value we had previously 
obtained a different way.  These values give, in turn,
$$
  c_4 = 104.99992\ldots \qquad\text{and}\qquad c_6=1322.9994\ldots,
$$
which round to the exact invariants $c_4=105$ and $c_6=1323$ of the curve
49A1, which has equation $y^2+xy=x^3-x^2-2x-1$.

Using the improved formula (\dotseriestwob) with better convergence,
gives (also using 100 terms)
$$
  x = 0.96665585\ldots \qquad\text{and}\qquad y = 2.55753099\ldots,
$$
which lead to the better values
$$
  c_4 = 104.9999992\ldots \qquad\text{and}\qquad c_6=1322.99998\ldots.
$$

In this computation, we have not exploited the presence of complex
multiplication.  Notice that, in fact, $y/x=\sqrt{7}$.  Obviously if
we had known this it would have given us an easier way of computing
$y$ from $x$, and hence from $L(f,1)$. However not all newforms at
square levels have complex multiplication.  Some are twists of forms
at lower levels (for example, 100A is the 5-twist of 20A, and 144B is
the $-3$-twist of 48A), which means that we could find the associated
curves more easily by twisting the earlier curve.  Others first appear
at the square level in pairs which are twists of each other (for
example, 121A and 121C are $-11$-twists of each other, and 196B is the
$-7$-twist of 196A).  One could probably find both periods of all such
forms by looking at suitable twists to moduli not coprime to the
level, but we have not done this systematically, as the more direct
method was adequate in all the cases we came across in compiling the
tables.

\bigskip
\goodbreak

In practice we always computed the Hecke eigenvalues for $p<1000$ at 
least, with a larger bound for higher levels.  In some cases, 
particularly when the target values of $c_4$ or (more usually) $c_6$ were 
large, and especially when a large twisting prime was needed, we needed 
to sum the series to several thousand terms before obtaining the vales of 
$c_4$ and $c_6$ to sufficient accuracy.

These four examples exhibit essentially all the variations which can
occur.  The only problem with the larger levels is one of scale, as
the number of symbols and the dimensions of the spaces grow.  A large
proportion of the computation time, in practice, is taken up with
Gaussian elimination.  This is why we have tried wherever possible to
reduce the size of the matrices which occur: first by carefully using
the 2-term symbol relations to identify symbols in pairs as early as
possible, and secondly by working in $H^+(N)$ during the stage where
we are searching for Hecke eigenvalues.  The symbol relation matrices
are very sparse (with at most three entries per row); sparse matrix
techniques, which we use in our implementation, help greatly here.
For finding eigenvectors of the Hecke algebra, however, we use a
completely general purpose exact Gaussian elimination procedure.

The second time-consuming stage is when we are computing a large number 
of Hecke eigenvalues, where we call a very large number of times the 
procedures to convert rational numbers (cusps) to M-symbols and look 
these up in tables to find their coordinates with respect to the symbol 
basis.  It is vital that these procedures are written efficiently; during 
the preparation of the tables, many great improvements in the efficiency 
of the program were achieved over a period of several months.


\enddocument
