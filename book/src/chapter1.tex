% ALGORITHMS FOR MODULAR ELLIPTIC CURVES       Last change: 10/96
%
% CHAPTER 1  INTRODUCTION
%
% Use AmSTeX 2.0
%
\input book.def
%
\topmatter
%
\title\chapter{1} Introduction \endtitle
%
\endtopmatter
%
\document
%\openup 2pt 
%\raggedbottom % for drafts only

\head Introduction to the First (1992) Edition \endhead

This book is in three sections.   First, we describe in detail an 
algorithm based on modular symbols for computing modular elliptic curves: 
that is, one-dimensional factors of the Jacobian of the modular curve 
$X_0(N)$, which are attached to certain cusp forms for the congruence 
subgroup $\Gamma_0(N)$.   In the second section, various algorithms for 
studying the arithmetic of elliptic curves (defined over the rationals) 
are described.  These are for the most part not new, but they have not 
all appeared in book form, and it seemed appropriate to include them 
here. Lastly, we report on the results obtained when the modular symbols 
algorithm was carried out for all $N\le1000$.  In a comprehensive set of 
tables we give details of the curves found,  together with all isogenous 
curves (5113 curves in all, in 2463 isogeny classes
\footnote{In the first edition, these numbers were given as~5089 and~2447 respectively, as the curves of conductor~702 were inadvertently
omitted.}).  Specifically, we give for each curve the rank and
generators for the points of infinite order, the number of torsion
points, the regulator, the traces of Frobenius for primes less than
100, and the leading coefficient of the $L$-series at $s=1$; we also
give the reduction data (Kodaira symbols, and local constants) for all
primes of bad reduction, and information about isogenies.

For $N\le200$ these curves can be found in the well-known tables usually 
referred to as ``Antwerp IV'' \cite{\Antwerp}, as computed by Tingley 
\cite{\Tingley}, who in turn extended earlier tables of curves found by 
systematic search; our calculations agree with that list in all 281 
cases.   For values of $N$ in the range $200<N\le320$ Tingley computed 
the modular curves attached to newforms for $\Gamma_0(N)$ only when 
there was no known curve of conductor $N$ corresponding to the newform: 
these appear in his thesis \cite{\Tingley} but are unpublished. As in 
\cite{\Antwerp}, the curves $E$ we list for each $N$ have the following 
properties. \roster \item They have conductor $N$, as determined by 
Tate's algorithm \cite{\Tate}. \item The coefficients given are those of 
a global minimal model for $E$, and these coefficients (or, more 
precisely, the $c_4$ and $c_6$ invariants) agree with the numerical 
values obtained from the modular calculation to several decimal places: 
in most cases, depending on the accuracy obtained---see 
below---differing by no more than $10^{-30}$. \item Their traces of 
Frobenius agree with those of the modular curves for all primes 
$p<1000$. \endroster

We have also investigated, for each curve, certain numbers related to
the \BSD\ conjecture.  Let $f(z)$ be a newform for $\Gamma_0(N)$ with
rational Fourier coefficients, and $E$ the elliptic curve defined over
$\Q$ attached to $f$.  The value of $L(f,1)$ is a rational multiple of
a period of $f$, and may be computed easily using modular symbols (see
\cite{\Manin} and Section~2.8 below). We have computed this rational number in each case, and find that it is always consistent with the \BSD\ conjecture
for $E$.  More specifically, let $\rp(f)$ be the least positive real
period of $F$ and $\RP(f)=2\rp(f)$ or $\rp(f)$ according as the period
lattice of $f$ is or is not rectangular. Then we find that
$L(f,1)/\RP(f)=0$ if (and only if) the Mordell-Weil group $E(\Q)$ has
positive rank, and when $E(\Q)$ is finite we find in each case that
$$
   L(f,1)/\RP(f) = \prod_{p|N}c_p\cdot|E(\Q)|^{-2}\cdot S
$$
with $S\in\N$ (in fact $S=1$ in all but four cases:  $S=4$ in three 
cases and $S=9$ in one case). This is consistent with the \BSD\ 
conjecture if the \TS\ group $\Sha$ is finite of order $S$.  (Here $c_p$ 
is the local index $[E(\Q_p) : E_0(\Q_p)]$; see \cite{\Silvera, p.362}.)   
When $L(f,1)=0$, we compute the sign $w$ of the functional equation for 
$L(f,s)$, and verify that $w=+1$ if and only if the curve has even rank. 
More precisely, we also compute the value of $L^{(r)}(f,1)$, where $r$ 
is the rank, the regulator $R$, and the quotient 
$$
   S = \frac {L^{(r)}(f,1)}  {r!\;\RP(f)} \left/ 
       \frac {(\prod c_p)\; R} {\left|E(\Q)_{\tor}\right|^2}  \right. . 
$$ 
In all but the four cases mentioned above we find that $S=1$ to within the 
accuracy of the computation.

Our algorithm uses modular symbols to compute the 1-hom\-ology of 
$\Gamma_0(N)\backslash\H^*$  where $\H^*$ is the extended upper 
half-plane $\{z\in\C : \hbox{Im}(z)>0\}\cup\{\infty\}\cup\Q$.   While 
similar in some respects to Tingley's original algorithm described in 
\cite{\Tingley}, it also uses ideas from \cite{\Manin} together with 
some new ideas which will be described in detail below.   One important 
advantage of our method, compared with Tingley's, is that we do not need 
to consider explicitly the exact geometric shape of a fundamental region 
for the action of $\Gamma_0(N)$ on $\H^*$: this means that highly 
composite $N$ can be dealt with in exactly the same way as, say, prime 
$N$.   Of course, for prime $N$ there are other methods, such as that of 
Mestre \cite{\Mestre}, which are probably faster in that case, though 
not apparently yielding the values of the ``\BSD\ numbers''
$L(f,1)/\RP(f)$.   There is also a strong similarity between the 
algorithms described here and those developed by the author in his 
investigation of cusp forms of weight two over imaginary quadratic 
fields \cite{\JECa}, \cite{\JECb}, \cite{\bsdiqf}.   A variant of this 
algorithm has also been used successfully to study modular forms for 
$\Gamma_0(N)$ with quadratic character, thus answering some questions 
raised by Pinch (see \cite{\Pincha} or \cite{\Pinchb}) concerning elliptic 
curves of everywhere good reduction over real quadratic fields.   See 
\cite\gammaone\ for details of this, and for a generalization to 
$\Gamma_1(N)$: one could find cusp forms of weight two with arbitrary 
character using this extension of the modular symbol method, though at 
present it  has only been implemented for quadratic characters, as
described in \cite\gammaone.

It is not our intention in this book to discuss the theory of modular 
forms in any detail, though we will summarize the facts that we need,
and give references to suitable texts.  The theoretical construction and 
properties of the modular elliptic curves will also be excluded, except 
for a brief summary.  Likewise, we will assume that the reader has some 
knowledge of the theory of elliptic curves, such as can be obtained from 
one of the growing number of excellent books on the subject.  Instead we
will be concentrating on computational aspects, and hope thus to 
complement other, more theoretical, treatments.

In Chapter~2 we describe the various steps in the modular symbol algorithm 
in detail. At each step we give the theoretical foundations of the method 
used, with proofs or references to the literature.  Included here are some
remarks on our implementation of the algorithms, which might be useful 
to those wishing to write their own programs.   At the end of this stage
we have equations for the curves, together with certain other data for
the associated cusp form: Hecke eigenvalues, sign of the functional 
equation, and the ratio $L(f,1)/\RP(f)$.

Following Chapter~2, we give some worked examples to illustrate the
various methods.

In Chapter~3 we describe the algorithms we used to study the elliptic
curves we found using modular symbols, including the finding of all
curves isogenous to those in the original list.  These algorithms are
more generally applicable to arbitrary elliptic curves over $\Q$,
although we do not consider questions which might arise with curves
having bad reduction at very large primes.  (For example, we do not
consider how to factorize the discriminant in order to find the bad
primes, as in all cases in the tables this is trivially achieved by
trial division). Here we compute minimal equations, local reduction
types, rank and torsion, generators for the Mordell--Weil group, the
regulator, and traces of Frobenius.  This includes all the information
published in the earlier Antwerp IV tables. The final calculations,
relating to the \BSD\ conjecture, are also described here; these
combine values obtained from the cusp forms (specifically, the leading
coefficient of the expansion of the $L$-series at $s=1$, and the real
period) with the regulator and local factors obtained directly from
the curves.  Thus we can compute in each case the conjectural value
$S$ of the order of $\Sha$, the \TS\ group.

Finally, in Chapter~4 we discuss the results of the computations for 
$N\le1000$, and introduce the tables which follow.

All the computer programs used were written in Algol68 (amounting to
over 10000 lines of code in all) and run on the ICL3980 computer at
the South West Universities Regional Computing Centre at Bath, U.K..
The author would like to express his thanks to the staff of SWURCC for
their friendly help and cooperation, and also to Richard Pinch for the
use of his Algol68 multiple-length arithmetic package.  At present,
our programs are not easily portable, mainly because of the choice of
Algol68 as programming language, which is not very generally
available.  However we are currently working on a new version of the
programs, written in a standard version of the object-oriented
language C++, which would be easily portable.  The elliptic curve
algorithms themselves are currently (1991) available more readily, in
a number of computer packages.\footnote{See the end of the
introduction for more on obtaining these packages.}  In particular,
the package {\tt apecs}, written in Maple and available free via
anonymous file transfer from Ian Connell of McGill University, will
compute all the data we have included for each curve. (A slightly
limited version of {\tt apecs}, known as {\tt upecs}, runs under
UBASIC on MS-DOS machines).  There are also elliptic curve functions
available for Mathematica (Silverman's Elliptic Curve Calculator) and
in the PARI/GP package.  These packages are all in the process of
rapid development.

An earlier version of Chapter~2 of this book, with the tables, has
been fairly widely circulated, and several people have pointed out
errors which somehow crept in to the original tables.  We have made
every effort to eliminate typographical errors in the tables, which
were typeset directly from data files produced by the programs which
did the calculations.  Where possible, the data for each curve has
been checked independently using other programs.  Amongst those who
have spotted earlier errors or have helped with checking, I would like
to mention Richard Pinch, Harvey Rose, Ian Connell, Noam Elkies, and
Wah Keung Chan; obviously there may still be some incorrect entries,
but these remain solely my responsibility.

\head Introduction to the Second (1996) Edition \endhead

Since the first edition of this book appeared in 1992, some
significant advances have been made in the algorithms described and in
their implementation.  The second edition contains an account of
these advances, as well as correcting many errors and omissions in the
original text and tables.  We give here a summary of the more
substantial changes to the text and tables.

Of course, the most significant theoretical advance of the last four
years is the proof by Wiles, Taylor--Wiles and others of most cases of
the Shimura--Taniyama--Weil conjecture, which almost makes the word
``modular'' in the title of this book redundant.  However, the only
effect the new results have on this work are to guarantee that every
elliptic curve defined over the rationals and of conductor less
than~1000 is isomorphic to one of those in our Table~1.

\subhead Chapter 2 \endsubhead  % NO BLANK LINE HERE
%
Section~2.1 has been completely rewritten and expanded to give a much
more coherent, self-contained, and (we hope) correct account of the
theoretical background to the modular symbol method.  The text here is
based closely on some unpublished lecture notes of the author for a
series of lectures he gave in Bordeaux in 1995 at the meeting ``\'Etat
de la Recherche en Algorithmique Arithm\'etique'' organized by the
Soci\'et\'e Math\'ematique de France.

In Section~2.4, we give a self-contained treatment of the method of
Heilbronn matrices for computing Hecke operators, similar to the
treatment by Merel in \cite\Merel, as this now forms part of our
implementation.

In Section~2.10, we give a new method of computing periods of cusp
forms, as described in \cite\JCperiods, which is as efficient as the
``indirect'' method; this largely makes the indirect method redundant,
but we still include it in Section~2.11.  Also in Section~2.11, we
include some tricks and shortcuts which we have developed as we pushed
the computations to higher levels, which can greatly reduce the
computation time needed to find equations for the curves of
conductor~$N$, at the expense of not necessarily knowing which is the
so-called ``strong Weil'' curve in its isogeny class.

Section~2.14 has been rewritten to take into account the results of
Edixhoven on the Manin constant (see \cite\Edixhoven), which imply
that the values of $c_4$ and $c_6$ which we compute for each curve are
known {\it a priori\/} to be integral.  This means that the values we
compute are guaranteed to be correct, and eliminates the
uncertainty previously existing as to whether the curves we obtain by
rounding the computed values are the modular elliptic curves
they are supposed to be.

Section~2.15 is entirely new: we show how to compute the degree of the
modular parametrization map $\varphi\colon X_0(N)\to E$ for a modular
elliptic curve of conductor~$N$, using our version (see
\cite\JCdegphi) of a method of Zagier \cite\Zagier.  This method is
easy to implement within the modular symbol framework, and we have
added it to our programs, so that we now compute the degree
automatically for each curve we find.

The appendix to Chapter~2, containing worked examples, now includes
the Heilbronn matrix method, and also illustrates some of the tricks
mentioned in Section~2.11.

\subhead Implementation changes \endsubhead  % NO BLANK LINE HERE
%
The implementations of all the algorithms described here have been
completely rewritten in C++, to be easily portable.  We use the GNU
compiler {\tt gcc} for this.  For multiprecision arithmetic we use
either the GNU package {\tt libg++} or the package {\tt LiDIA}.  For
solving the systems of linear equations giving the relations between
M-symbols, we use sparse matrix routines which not only reduce memory
requirements, but also speed up that part of the computations
considerably.  These routines were written by L.~Figueiredo
specifically for his work on imaginary quadratic fields (see
\cite\Luiz) which in turn built on the author's work in \cite\JECa\
and \cite\JECb.

\subhead Chapter 3 \endsubhead  % NO BLANK LINE HERE
%
In Section~3.1 we give simpler formulae for recovering the \W\
coefficients of a curve from the invariants $c_4$ and~$c_6$; this
enables us to simplify the Kraus--Laska--Connell algorithm slightly.
In Section~3.4 we give a slightly improved formula for the global
canonical height, and include this as a separate algorithm.
Section~3.5 now contains references to other bounds between the naive
and canonical heights, and other methods for the infinite descent
step, but without details.

The main changes in this chapter are to Section~3.6 on two-descent
algorithms.  On the one hand, we give a better explanation of the
theoretical basis for these algorithms, making the account more
self-contained (though we do not include all proofs).  We have also
moved the discussion on testing homogeneous spaces for local and
global solubility forward, as this is common to the two main
algorithms (general two-descent and two-descent via $2$-isogeny).  On
the other hand, several parts of the algorithm have been subject to
major improvements over the last few years, thanks to collaboration
with P.~Serf, S.~Siksek and~N.~P.~Smart, and these are now included.
Notable here are the syzygy sieve in the search for quartics, the
systematic use of group structure in the $2$-isogeny case, and the use
of quadratic sieving in searching for rational points on homogeneous
spaces.  We also simplify the test for equivalence of quartics and the
process of recovering rational points on the curve from points on the
homogeneous spaces.  Many of these improvements are from the author's
paper \cite\JCinvariants, which contains some proofs omitted here.

\subhead Implementation changes \endsubhead  % NO BLANK LINE HERE
%
As with the modular symbol algorithms, we have rewritten all the
elliptic curve algorithms in C++.  In the case of the program to find
isogenies, which is very sensitive to the precision used, we have
written an independent implementation in PARI/GP; using this we have a
check on the isogeny computations which gave the isogenous curves
listed in Table~1.  (The standard precision version of this program,
while much faster, does miss several of the isogenies, for reasons
given in Section~3.8.)

Versions of our algorithms will shortly become generally available in
two forms.  First, the package {\tt LiDIA} (a library of C++ classes
for computational number theory, developed by the LiDIA group at the
Universit\"at des Saarlandes in Germany) will include them in a
coming release.  Secondly, the package {\smc Magma} is also in the
process of implementing the algorithms.

In addition to these packages and those mentioned in the original
Introduction, we should also mention the package {\smc Simath},
developed by H.~G.~Zimmer's research group in Saarbr\"ucken, which
also has a large collection of very efficient elliptic curve
algorithms.

See the end of this Introduction for how to obtain more information on
these packages.

\subhead Chapter 4 and Tables \endsubhead  % NO BLANK LINE HERE
%
The two main changes in the tables are to include all the data for
$N=702$ in Tables~1--4 and include the new Table~5 giving the degree
of the modular parametrization for each strong Weil curve.  The
omission of level $702$ in the first edition is hard to explain; in
our original implementation and file structure, it was not possible to
distinguish between a level which had run successfully, but with no
rational newforms found, and a level which had not yet run.  The
original runs were done as batch jobs on a remote mainframe computer,
with manual record-keeping to keep track of which levels had run
successfully.  Our current implementation is much more robust in this
respect.  We are grateful to Henri Cohen who first discovered this
error on comparing our data with his own tables (of modular forms of
varying weight and level, computed by him together with Skoruppa and
Zagier).  The omission was also noted by Jacques Basmaji of Essen, who
recomputed Table~3 independently.

The new implementation finds the newforms at each level in a
consistent order.  In the original runs, the order in which the
newforms were found changed as the program developed.  Unfortunately,
we did not recompute the earlier levels with the final version of the
program before publishing the first edition of the tables, and the
identifying letter for each newform given in the tables has now become
standard.  Hence our current implementation reorders the newforms
during output to agree with the order as originally published (this is
necessary for 147 levels in all, the largest being 450).

Also concerning the order and naming of the curves: the convention we
normally use is that in each isogeny class the first curve is the
strong Weil curve whose period lattice is exactly that of the
corresponding newform for $\Gamma_0(N)$, such as 11A1 for example.  In
precisely one case, an error caused the first curve listed in class
990H to be not the strong Weil curve but a curve isogenous to it.
The strong Weil curve in this class is in fact 990H3 and not 990H1.
In the notation of Section~2.11, the correct values of $l^+$
and $m^+$ to obtain the strong Weil curve 990H3 are~13 and~8, but for
some reason we had used the value $m^+=24$ which leads to the
3-isogenous curve~990H1.

In Table~1, the other corrections are: $N=160$ has the Antwerp codes
corrected, and 916B1 has a spurious indication of a non-existent
3-isogeny removed. 

In Table~2, we include the generators for the curves of conductor~702
and positive rank, and again correct the Antwerp code for curve~160A1.
We also give the generator of~427C1 correctly as $(-3,1)$ rather than
$(-3,0)$ as previously, and for 990H we give the generator $(-35,97)$
of the strong Weil curve 990H3 rather than a generator of~990H1 as
before.

In Table~3, as well as inserting the data for $N=702$, we correct the
eigenvalues for $N=100$ which had been given incorrectly.

In Table~4, we insert the data for $N=702$ and also for 600E--600I
which had been omitted by mistake.  Moreover, for $N=990$ we give the
data for 990H3 instead of 990H1 as before, as this is the strong Weil
curve (the only difference being that $\Omega$ has been multiplied
by~3 and $R$ divided by~3).

\subhead Extension of the Tables \endsubhead  % NO BLANK LINE HERE
%
Using our new implementation of the algorithms of Chapter~2, we have
extended the computations of all modular elliptic curves up to
conductor~5077 (chosen as the smallest conductor of a curve of
rank~3).  We have also computed in each case the other data tabulated
here for conductors up to~1000.  For reasons of space, we cannot print
extended versions of the tables: as there are 17598 newforms (or
isogeny classes) and a total of 31586 curves up to~5077, this would
have made this book approximately six times as thick as it is at
present!

Instead, the data for curves whose conductors lie in the range
from~1001 to~5077 (and beyond, as they become available) may be
obtained by anonymous file transfer from
the author's web site at
\hbox{\tt
http://www.maths.nott.ac.uk/personal/jec/ftp/data/INDEX.html}
%{\tt ftp://euclid.ex.ac.uk/pub/cremona/data}.

%In this extended data, it is important to note one thing concerning the
%numbering of the curves.  For levels over 4500, in most cases we have
%not carried out in full the second phase of the computation (see the
%description of the algorithm in Chapter~2 for details of this) to
%determine the strong Weil curve.  So in this range, whenever there is
%more than one curve in the isogeny class, we do not know which is the
%strong Weil curve.  For this reason, anyone referring to a specific
%curve in one of these classes should regard its identifying number as
%only provisional.  For the same reason, the extended table of the
%degrees of the modular parametrizations is only complete to
%level~4500.

\medskip

Finally, many thanks to those who have told me of misprints and other
errors in the First Edition, including J.~Basmaji, G.~Bailey,
B.~Brock, F.~Calegari, J.~W.~S.~Cassels, T.~Kagawa, B.~Kaskel,
P.~Serf, S.~Siksek, and N.~Smart.  Apologies to any whose names have
been omitted.  Extra thanks are also due to Nigel Smart, who read a
draft of Chapter~3 of the Second Edition, and made useful suggestions.

\head web and ftp sites \endhead

More information on the packages mentioned above, and in most cases
the packages themselves, can be obtained from the following web and
ftp sites.  Apart from {\smc Magma} they are all free.

\medskip
\settabs 3\columns
\+{\tt apecs} (for Maple):
&{\tt ftp://math.mcgill.ca/pub/apecs}\cr

\+{Elliptic Curve Calculator} (for Mathematica):\cr
\+&{\tt ftp://gauss.math.brown.edu/dist/EllipticCurve}\cr

\+{\tt LiDIA}: 
&{\tt http://www-jb.cs.uni-sb.de/LiDIA}\cr
%%%\+&{\tt http://www.informatik.th-darmstadt.de/TI}\cr
%%%\+&{\tt ftp://ftp.informatik.th-darmstadt.de/pub/TI/systems/LiDIA}\cr
%%%\+&{\tt ftp://unix.hensa.ac.uk/mirrors/LiDIA}\cr

\+{\smc Magma}: 
&{\tt http://www.maths.usyd.edu.au:8000/comp/magma}\cr

\+{\tt mwrank}:
&{\tt ftp://euclid.ex.ac.uk/pub/cremona/progs}\cr

\+PARI/GP: 
&{\tt ftp://megrez.math.u-bordeaux.fr/pub/pari}\cr

\+{\smc Simath}: 
&{\tt http://emmy.math.uni-sb.de/\~{}simath}\cr

\+{\tt upecs} (for UBASIC):
&{\tt ftp://math.mcgill.ca/pub/upecs}\cr

\medskip

Links to all of these can be found at the following place, which will
be updated. 
 
\medskip

%%{\tt http://www.maths.ex.ac.uk/\~{}cremona/packages.html}.
{\tt
http://www.maths.nott.ac.uk/personal/jec/packages.html}


\enddocument
