% ALGORITHMS FOR MODULAR ELLIPTIC CURVES  (2nd edn.)  Last change: 10/96
%
% CHAPTER 4  THE TABLES
%
% Use AmSTeX 2.1 or higher
%
\input book.def
\advance\pageno by\chaponepages
\advance\pageno by\chaptwopages
\advance\pageno by\appendixpages
\advance\pageno by\chapthreepages
%
\topmatter
%
\title\chapter{4} The tables \endtitle
%
\endtopmatter
%
\document
%\openup 2pt 
%\raggedbottom % for drafts only

\head Introduction to the tables \endhead

\subhead Table~1.  Elliptic curves \endsubhead

In Table~1 we give details of each computed elliptic curve $E$ of 
conductor $N$ for $N\le1000$, arranged by conductor and isogeny class.  
The first curve in each class
\footnote{with the single exception of class 990H, where curve H3 is
the strong Weil curve $E_f$, owing to a slip; we have not changed
the numbering, in order to maintain consistency with the first edition
of these tables.}  is the `strong Weil curve' $E_f$ computed from the
periods of the newforms $f$, and it is followed by the isogenous
curves, if any.  There are 2463 isogeny classes, and 5113 curves in
all.

The table contains the coefficients of minimal models of all the
curves.  Each curve has a code of the form $N$X$i$, where $N$ is the
conductor, X is the letter code for the corresponding newform or
isogeny class, and $i$ is the number of the curve in its class.  The
order of the isogeny classes for each $N$ is the order in which the
corresponding newforms were found.  
\footnote{Roughly speaking, this is in lexicographic order of the
vector of Hecke eigenvalues, but we claim no uniformity here; as the
program evolved its strategy changed at least twice: once when we
first started using the $W$ operators, and again when we changed the
order of searching for eigenvalues of $T_p$ from
$\ldots,-2,-1,0,1,2,\ldots$ to $0,1,-1,2,-2,\ldots$, after realizing
that small values of $a_p$ were more likely to occur.  Unfortunately,
this means that the current version of the program recomputes the
newforms in a different order; for consistency, we now have to build
into the program the permutations necessary to produce output in the
order fixed by the first edition of the tables.}

After the first curve in each class,  the other curves in the class (if 
any) are listed in the order in which they were found by the isogeny 
program, as described in Section~3.8.  The isogeny information 
given in the last column was also recorded by that program.  For each 
curve for $N\le200$ we also give in parentheses the Antwerp code of each, 
as in \cite\Antwerp.  Thus curves 11A1, 11A2 and 11A3 are the Antwerp 
curves 11B, 11C and 11A in that order.  We hope that this new system of 
identifying codes will not cause confusion; it was the most natural, 
given the way the curves were found.

The other data in this table is, for each curve: the rank and number of 
torsion points; the factorization of the discriminant and $j$-invariant; 
the local index $c_p$ and the Kodaira symbol at each bad prime $p$; and 
the isogenies of prime degree.

When the number of torsion points is of the form $4k$,
one can tell whether the torsion subgroup is cyclic ($C_{4k}$) or not
($C_{2k}\times C_2$) by seeing whether the number of 2-isogenies is 1
or 3 (respectively), since this number is the same as the number of
points of order~2.

We have not indicated on this table either the presence of complex
multiplication, or when a curve is a twist of one elsewhere in the
table.  These omissions were made to save space.  Complex
multiplication may be determined most easily by referring to the table
in Section~3.9, where a complete list of the rational
$j$-invariants of curves with \CM\ was given.  For example, the
isogeny class 49A1-2-3-4 consists of four \CM\ curves; 49A1,3 have
$j=-3375=-15^3$ and CM by $-7$, while 49A2,4 have
$j=16581375=255^3=3^35^317^3$ and CM by $-28$.  All other curves with
these \CM s are twists of these.  Within the range of the tables here
we find their $-3$-twists at 441D1-2-3-4 and their $-4$-twists at
784H1-2-3-4.  As remarked earlier, the presence of~\CM\ can also be spotted in
Table~3, where half the Hecke eigenvalues will be zero.

In the non-\CM\ case, unless $N$ is divisible by $D^2$ the twist with
discriminant $D$ of a (non-\CM) newform at level $N$ will have level
$\lcm(N,D^2)>N$; thus a necessary condition for a newform to have a
twist earlier in the table is that its level should be divisible by a
square.  It is then usually easy to see which newform at a lower level
is the twist.  Table~3 can help here, since twisted newforms have the
same Hecke eigenvalues $a_p$ up to sign.  Thus we can determine when
two isogeny classes of curves are twists of each other, since each
class corresponds to a newform.  Once two isogeny classes have been
identified as twists of each other, one can determine which curves in
the first class are twists of which in the second by comparing
$j$-invariants.

For example, consider level $N=704=2^6\cdot11$.  At this level three
twists operate, with $D=-4$, $-8$ and $+8$.  The 12 newforms form 6
pairs which are $-4$-twists of each other: A--K, B--C, D--F, E--I,
G--J and H--L.  Their $\pm8$-twists, however, are all at lower levels.
704A and 704K are $\pm8$ twists of 11A and 176B; 704D and 704F of 44A
and 176C; 704E and 704I of 88A and 176A; 704G and 704J of 352A and
352C; 704B and 704C of 352B and 352D; and 704L and 704H of 352E and
352F (respectively).  Thus of the 24 newforms, we only have 6 up to
twists, whose first representatives are 11A, 44A, 88A, 352A, 352B and
352E.  The first of the corresponding sets of isogeny classes consists
of three curves, linked by 5-isogenies: 11A3-1-2; 176B1-2-3;
704A1-2-3; 704K1-2-3 (respectively).  (Note that in order to keep to
our convention that the first curve in each class is the `strong Weil
curve', we were not able to number the curves in the classes in such a
way that the numbers in twisted classes correspond: being the `strong'
curve is not preserved under twisting, as this example shows.)  Also
the classes 44A, 176C, 704D, 704F each consist of a pair of curves
linked by 3-isogeny: in this case the first curves do all correspond
under twisting.  Each of the other isogeny classes consists of a
single curve.

\subhead Table~2.  Mordell--Weil generators \endsubhead

The second table contains generators for the Mordell--Weil group
(modulo torsion) of the first curve\footnote{except for class 990H,
where we give the generator for curve 990H3 as this is the strong Weil
curve} in each isogeny class of curves of positive rank. In the case
of rank~1 curves, this generator $P$ is unique up to replacing $P$ by
$\pm P+Q$ where $Q$ is a torsion point; in rank~2 cases, the given
generators $P_1$, $P_2$ could be replaced by $aP_1+bP_2+Q_1$ and
$cP_1+dP_2+Q_2$, where $ad-bc=\pm1$ and $Q_1$, $Q_2$ are torsion
points.

To save space, we only list generators for one curve in each isogeny
class.  (Generators for the other curves may be obtained from the
author's ftp site given below.)  As in Table~1, for $N\le200$ we give
in brackets the Antwerp code of the curve.

For $N\le200$ there are some discrepancies with Table~2 of
\cite{\Antwerp}: the generators for 143A and 154C are omitted there;
the point $(0,2)$ on 155D has order~5, with $(2,5)$ being a generator
of infinite order; and on 170A, a generator is $P=(0,2)$, and the
point $(2,1)$ given in \cite{\Antwerp} is $-2P$.

\subhead Table~3.  Hecke eigenvalues \endsubhead

In Table~3 we give the Hecke eigenvalues for all the rational newforms at 
all levels up to $N=1000$.  As in \cite\Antwerp, we give the eigenvalue 
$a_p$ for all $p<100$ not dividing $N$, and the $W_q$ eigenvalue $\eps_q$ 
for all primes $q$ dividing $N$.  

Almost all of these numbers could be computed from the modular curves
themselves, as listed in Table~1, by the formulae of Section~2.6.  For
$p\ndiv N$, the eigenvalue $a_p$ is equal to the trace of Frobenius of
the curve, which is easily computed as in Section~3.9.  When
$q\div\div N$, we have $-\eps_q=a_q=1+q-\left|E({\Bbb F}_q)\right|$.
However when $q^2\div N$ we cannot recover $\eps_q$ this way, since
then $a_q=1+q-\left|E({\Bbb F}_q)\right|=0$.  We did in fact check in
each case that the Hecke eigenvalues and traces of Frobenius agreed
for all $p<1000$.

Each newform is identified by its level $N$ and a letter, as in Table~1.  
Thus 50A is the newform corresponding to the curve 50A1 (and by isogeny 
to 50A2,3,4), while 50B corresponds to curves 50B1,2,3,4. As in Table~1, 
for $N\le200$ we give in brackets the Antwerp codes, for ease of 
cross-reference.

\subhead Table~4.  Birch--Swinnerton-Dyer data \endsubhead

In this table we present the data pertaining to the \BSD\ conjectures
for the modular curves $E=E_f$ of conductor $N$ up to 1000 attached to rational
newforms $f$ in $S_2(N)$.  In each case we list the following quantities:

$\bullet$ the rank $r$ 
%(either the analytic rank of $f$ or the Mordell--Weil rank of $E(\Q)$);

$\bullet$ the period $\RP(f)=\RP(E)$;

$\bullet$ the value of $L^{(r)}(E,1)/r!=L^{(r)}(f,1)/r!$;

$\bullet$ the regulator $R$ of $E(\Q)$; 

$\bullet$ the ratio $L^{(r)}(E,1)/r!\RP R$;

$\bullet$ and finally the quantity $S$ defined as
$$
   S = \frac {L^{(r)}(f,1)}  {r!\;\RP(f)} \left/ 
       \frac {(\prod c_p)\; R} {\left|E(\Q)_{\tor}\right|^2}  \right. . 
$$ 

Note that some of these quantities are computed from the newform $f$,
while others are computed from the curve $E$.  Some can be obtained
from either: for example, we know that the analytic rank is in each
case equal to the Mordell--Weil rank.  When the rank is 0, we use the
exact value of the ratio ratio $L(f,1)/\RP(f)$ obtained via modular
symbols; dividing by the rational number $\prod
c_p/\left|E(\Q)_{\tor}\right|^2$, we thus obtain an exact rational
value for $S$.  This was equal to 1 in all but four cases: $S=4$ for
571A1, 960D1 and 960N1, and $S=9$ for 681B1.  These are consistent
with the data concerning the order of $\Sha(E/\Q)$ coming from the
two-descent which we carried out: in the three cases 571A1, 960D1 and
960N1 and in no other cases there are homogeneous spaces (2-coverings
of $E$) which have points everywhere locally, but on which we could
find no rational point.  Since these curves are modular, and we know
that $L(E,1)\not=0$ in each case, we know that the rank is in fact 0
(by Kolyvagin's result), so that $\left|\Sha[2]\right|=4$ in each
case.  For the \BSD\ conjecture to hold we would need to establish
$\left|\Sha\right|=4$. This should be possible using the methods of
Rubin and Kolyvagin.

When the rank is positive the ratio is computed from three approximations
to the values of $L^{(r)}(f,1)$, $R$ and $\RP(E)$.  Thus in these cases 
we only compute an approximation to the value of $S$; but in all cases in 
the tables this value was equal to 1 to within the accuracy of the 
computation.  These values are listed as $1.0$ in the table to emphasize 
the fact that they were obtained as approximations.

\subhead Table~5.  Parametrization degrees \endsubhead

This table shows, for each newform $f$, the degree of the modular
parametrization $X_0(N)\to E_f$, together with the prime
factorization of the degree.

\head Some remarks on the computations \endhead

We have implemented the algorithms described in Chapter~2 and run them
for all $N$ up to 5077. For a summary of the results obtained over
this range, see the end of this section.

In the first phase of the computation for each $N$, we worked in the
smaller space $H^+(N)$ and found rational newforms, storing in a data
file the number of forms, and for each, the rational number
$L(f,1)/\RP(f)$ and a certain number of Hecke eigenvalues including
all the $W_q$ eigenvalues.  We also stored enough modular symbol
information that if we needed more Hecke eigenvalues later we could
resume the calculation with the minimum of repetition.  At this stage
we already had all the data given in Table~3 below, and knew the sign
of the functional equation (and hence the parity of the analytic rank)
and whether or not $L(f,1)=0$.  Thus we had enough to guess the
analytic rank as 0, 1 or 2; and in each case this preliminary estimate
turned out to be correct, since no curves of rank greater than~2 were
found for $N<5077$.

The second phase was to work in $H(N)$, using the eigenvalues already
known, to find the period lattices, the approximate $c_4$ and $c_6$
invariants of the modular curves, and hence their (rounded)
coefficients.  These coefficients were stored.  We also stored
information about the twisting primes $l$ and matrices used to
evaluate the periods.  In some cases the first pass through this phase
did not produce $c_4$ and $c_6$ to sufficient precision; this tended
to happen when their values were large and when the auxiliary primes
$l$ were also large.  In such cases we went back to phase~1 to compute
more eigenvalues, so that we could evaluate more terms of the relevant
series.  The re-evaluation of the periods given more $a_p$ was then
very fast, as we had all the relevant information stored and merely
had to sum the series. In very few cases did we need to use $a_p$ for
$p>5000$.

We then implemented and ran the algorithms of Chapter 3 on the
resulting curves, including finding all curves isogenous to
them. Checking that the $c_4$ and $c_6$ invariants were indeed those
of a minimal model of a curve of conductor $N$ was in fact done by the
program which computed them in the first place, so that we could tell
when sufficient accuracy had been obtained.  Starting from a file
containing the coefficients of the original `strong' curves $E_f$, we
ran the isogeny program to produce a larger file with the complete
list of curves, together with information on the degrees of the
isogenies linking them. This file was then used by a further program
which produced the \TeX\ source code for Table~1, with all the other
data there being recomputed or read from files.

The number of torsion points was computed by the method of Section~3.3.  

The rank was first guessed as the smallest possible value consistent
with the \BSD\ conjecture, given that we knew whether $L(E,1)$ was
zero and the sign of the functional equation; this value $r=0$, $1$
or~$2$ was then confirmed as the analytic rank by the computation of
$L^{(r)}(f,1)$.  When $r=0$ or~1 it then follows that the
Mordell--Weil rank of $E(\Q)$ is also $r$ by Kolyvagin's results.  In
almost all cases, including all of those where $r=2$, we verified this
using our two-descent rank program.  The exceptional cases were for
curves with no rational two-torsion and very large coefficients, where
the two-descent would have taken too long.  In these cases we did know
(independently of Kolyvagin) that the given value of $r$ was a
lower bound, since we always found that number of independent
generators of infinite order.

The case $r=2$ occurs only 18 times in these tables, with the
following curves: 389A, 433A, 446D, 563A, 571B, 643A, 655A, 664A,
681C, 707A, 709A, 718B, 794A, 817A, 916C, 944E, 997B and 997C.  In
each case, there are no isogenous curves.

Apart from the rank, all the other data in Table~1 was computed by our 
implementation of Tate's algorithm.   

The generators given in Table~2 were obtained by the methods of
Section~3.5, where we first searched for the expected number of
independent points of infinite order, and then refined these where
necessary to be sure we had generators of the curves (modulo torsion),
and not of a subgroup of finite index.  In most cases the generators
were found immediately; the hard ones to find were those with most
digits, and particularly those which are not integral.  For the last
few to be found, the modular refinements mentioned in Chapter~3 were
essential.  (For the record, the generator of 873C1 was first found
not by search, but via Heegner points, during the July 1989 Durham
meeting on $L$-series; by day we learned about the latest results of
Rubin and Kolyvagin, and about Heegner points, while by night we
applied some of these ideas in a new program, which eventually came up
with the elusive point.)  We also found generators on the isogenous
curves; see \cite\JCsha\ for details of this.

Table~4 was produced by a program which took as data the Hecke
eigenvalues, sign of functional equation (and hence the analytic rank)
of the newform $f$; the coefficients of the curve $E_f$ attached to
$f$; and the generators.  Then the value $L^{(r)}(f,1)$ and the period
$\RP$ were computed from the eigenvalues as in Section~2.13; and the
regulator from the heights of the generators as in Section~3.4.  We
also recomputed the local factors $c_p$ and number of torsion points,
and from all these could obtain the conjectured value $S$ of the order
of the \TS\ group $\Sha$.

Finally, the data in Table~5 was obtained by implementing the formula
for $\deg(\varphi)$ given in Section~2.15.  The degree was computed
during the second phase of each computation (working in $H(N)$) and
stored on file with the other data.

All the tables were produced as follows, to minimize the risk of transcription
error.   The programs which computed the numbers themselves wrote the results
to data files; separate programs read in this data and added the \TeX\
formatting characters, producing the \TeX\ source files, which were then
processed in the usual way.  Thus none of the numbers in the tables was typed
by human hand at any stage.   Many consistency checks were applied along the
way.  Almost all the errors in the earlier versions of the tables arose as a
result of using old versions of data files by mistake, rather than from errors
in the programs.  We sincerely hope that all such slips have been avoided here.

More details of the specific layout of each of the five sets of tables
is given immediately before each set below.

\head Summary of results obtained up to 5077 \endhead

Although the tables included here only contain results for levels up
to~1000, we have to date carried out the computations described above
for all levels $N$ up to~5077.  For each $N$ we found the rational
newforms, and computed many Hecke eigenvalues for each; for each form
we computed a period lattice, and hence found a corresponding curve of
conductor $N$.  
%
% Next sentence was deleted when computations completed to 5077:
%
%At present, we have found the full period lattice and hence $E_f$ only
%for $N\le4500$; for $4500<N\le5077$ we have only used the method of
%Section~2.11 to find a suitable curve, which we only know to be
%isogenous to $E_f$.
%
For each curve, we verified by 2-descent (in most cases, including all
cases of rank~2) that the rank was equal to the analytic rank, and by
finding the Mordell-Weil group of each curve (again, in most cases) we
were able to compute the value predicted by the Birch Swinnerton-Dyer
conjecture for the order of the
\TS\ group.

We summarize the results obtained in the following table, where for
brevity we only give the numbers of newforms found, subdivided by
rank.  The first examples of each rank are: $N=11$ for rank~0; $N=37$
for rank~1; $N=389$ for rank~2; and $N=5077$ for rank~3.

\medskip
\centerline{\bf Rational newforms for $\Gamma_0(N)$, $N\le5077$}
\bigskip
\def\gap{\omit&height2pt&&&&&&&&&&&&\cr}
{\offinterlineskip
\centerline{
\vbox{
\halign{
\strut#&\vrule#\tabskip5pt plus1fil&          % left border
\hfil#&\vrule#&                          % range
\hfil$#$&\vrule#&                        % total
\hfil$#$&\vrule#&                        % r=0
\hfil$#$&\vrule#&                        % r=1
\hfil$#$&\vrule#&                        % r=2
\hfil$#$&\vrule\tabskip0pt#\cr           % r=3
\noalign{\hrule}
\gap
&&{Range of $N$}&&\hbox{Total}&&r=0&&r=1&&r=2&&r=3&\cr
\gap         
\noalign{\hrule}
\gap   
&&   1--1000&&2463&&1321&&1124&&18&&0&\cr
&&1001--2000&&3391&&1575&&1737&&79&&0&\cr
&&2001--3000&&3661&&1663&&1852&&146&&0&\cr
&&3001--4000&&3837&&1665&&2006&&166&&0&\cr
&&4001--5000&&3962&&1690&&2092&&180&&0&\cr
&&5001--5077&&284&&121&&148&&14&&1&\cr
\gap         
\noalign{\hrule}
\gap   
&&   1--5077&&17598&&8035&&8959&&603&&1&\cr
\gap         
\noalign{\hrule}
}}}}

\medskip
For conductors $N>1000$, tables of the curves and related data may be
obtained online from {\tt
http://www.maths.nott.ac.uk/personal/jec/ftp/data/INDEX.html}.

\enddocument
