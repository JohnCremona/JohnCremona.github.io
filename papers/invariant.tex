%  INVARIANTS AND ELLIPTIC CURVES
%                                                   
%  last edited 24/6/96 and put into departmental preprint series
%
\documentstyle{amsppt}     
\magnification=\magstep 1
\overfullrule 0pt
%
\define\Z{\Bbb Z}
\define\N{\Bbb N}
\define\Q{\Bbb Q}
\define\R{\Bbb R}
\define\C{\Bbb C}
\define\T{\Bbb T}
\define\SL{\text{SL}}
\define\PSL{\text{PSL}}
\define\GL{\text{GL}}
\define\PGL{\text{PGL}}
\define\knl{\operatorname{ker}}
\define\rk{\operatorname{rank}}
\define\tor{\operatorname{tor}}
\define\ord{\operatorname{ord}}
\define\sgn{\operatorname{sign}}
\define\id{\operatorname{id}}
\def\Gal{\operatorname{Gal}}
\def\Tr{\operatorname{Tr}}
\def\lcm{\operatorname{lcm}}
\define\eps{\varepsilon}
\define\w{\omega}
\def\phi{\varphi}
\def\from{\colon}
\redefine\H{\hbox{$\Cal H$}}
\def\G#1(#2){\Gamma_{#1}(#2)}
\define\mat(#1,#2;#3,#4){\pmatrix#1&#2\\#3&#4\endpmatrix}
\redefine\div{\mid}
\define\ndiv{\nmid}
\define\<#1>{\left<#1\right>}
\def\slash#1{\left|#1\right.}
\def\Kbar{\overline{K}}
\def\Ft{\tilde{F}}
\def\EIJ{E_{I,J}}
\def\CC{{\Cal C}}
\def\P{{\Cal P}}
\def\d{\partial}
\def\longnearrow{\lower4pt\hbox{$\diagup$}\kern-1.3pt\raise4pt\hbox{$\nearrow$}}
\def\part#1{\par\noindent{\hbox to \parindent{\rm(#1)\hss}}}  % For parts of Theorems, etc.

\newcount\refno  \refno=0
\newcount\secno  \secno=0
\newcount\eqno   \eqno=0
\newcount\propno \propno=0

\def\beginsection#1{\xdef\sec{#1} \eqno=0 \propno=0}

\def\newref#1{\advance\refno by1 \xdef#1{\the\refno}}
\def\newsec#1{\advance\secno by1 \xdef#1{\the\secno}}
\def\neweq#1{\advance\eqno by1 \xdef#1{\sec.\the\eqno}}
\def\newprop#1{\advance\propno by1 \xdef#1{\sec.\the\propno}}

\newref\BSD        % 1963 Crelle paper
% \newref\Cassels    % Survey article
\newref\JCbook     % My book
\newref\JCPS       % Joint paper with Pascale
\newref\Elliott    % Theory of quantics
\newref\Hilbert    % Invariants book
\newref\PSthesis   % Pascale's thesis
%
\newsec{\Intro}
\newsec{\Invariants}
\newsec{\Galois}
\newsec{\Descent}
\newsec{\GStructure}

%%%%%%%%%%%%%%%%%%%%%%%%%%%%%%%%%%%%%%%%%%%%%%%%%%%%%%%%%%%%%%%%%%%%%%

\topmatter
\title  Classical Invariants and $2$-descent on Elliptic Curves
\endtitle
\rightheadtext{Classical Invariants and Elliptic Curves}
\author J. E. CREMONA
\endauthor
\address{Department of Mathematics,
         University of Exeter,
         North Park Road,
         Exeter EX4~4QE
         U.K.}
\endaddress
\email cremona\@maths.exeter.ac.uk 
\endemail
%%%%%%%%%%%%%%%%%%%%%%%%%%%%%%%%%%%%%%%%%%%%%%%%%%%%%%%%%%%%%%%%%%%%%%
\abstract The classical theory of invariants of binary quartics is
applied to the problem of determining the group of rational points of
an elliptic curve defined over a field $K$ by $2$-descent.  The
results lead to some simplifications to the method first presented by
Birch and Swinnerton-Dyer in \cite\BSD, and can be applied to give a
more efficient algorithm for determining Mordell-Weil groups over
$\Q$, as well as being more readily extended to other number fields.
In this paper we restrict to general theory, valid over arbitrary
fields (of characteristic neither 2 nor 3); in a subsequent paper, the
case where $K$ is a number field, and the specific case $K=\Q$, will be
treated in detail. \endabstract

\endtopmatter 

%%%%%%%%%%%%%%%%%%%%%%%%%%%%%%%%%%%%%%%%%%%%%%%%%%%%%%%%%%%%%%%%%%%%%%

\document
%\openup 6pt \raggedbottom % for double spacing
%
%%%%%%%%%%%%%%%%%%%%%%%%%
%                       %
%   SECTION ONE         %
%                       %
%%%%%%%%%%%%%%%%%%%%%%%%%

\beginsection{\Intro}
\head\Intro. Introduction \endhead

Computing the rank and a basis for the group of rational points of an
elliptic curve over a number field is a highly non-trivial task, even
over the field $\Q$ of rational numbers.  This is particularly true
when the curve has no rational 2-torsion.  The only general method
which avoids extending the ground field goes back to Birch and
Swinnerton-Dyer \cite\BSD, and is based on classifying certain binary
quartic forms.  This method is described briefly in \cite\JCbook, and
in more detail in Serf's thesis \cite\PSthesis, where it is also
extended to real quadratic fields of class number one (see also
\cite\JCPS).  In this paper I will show how parts of this method may
be simplified and improved by using more classical invariant theory
and Galois theory than in the original treatment in \cite\BSD.  With
this approach it is hoped not only to make existing implementations of
the algorithm over $\Q$\ simpler and more efficient, but also to make
the job of extending the implementation to other number fields more
practical.

In this paper we restrict to general theory, valid over arbitrary
fields (of characteristic neither 2 nor 3); in a subsequent paper, the
case where $K$ is a number field, and the specific case $K=\Q$, will be
treated in detail from an algorithmic viewpoint.

I would like to thank Nigel Byott, Robin Chapman, Samir Siksek and
Nigel Smart for useful conversations.

\beginsection{\Invariants}
\head\Invariants. Invariant Theory for Binary Quartics \endhead

In this section we review some standard material on the invariant
theory of binary quartic forms.  Our references here are Hilbert's
lecture notes \cite\Hilbert\ and also the book \cite\Elliott.  In
these texts the ground field is never made explicit.  We will work
over an arbitrary field $K$ whose characteristic is not~2
or~3.  It will not be necessary to assume that $K$ is a number field,
although eventually this will be the case of most interest to us.

Let \neweq{\gxydef}
$$
   g(X,Y) = aX^4+bX^3Y+cX^2Y^2+dXY^3+eY^4 \tag\gxydef
$$
be a binary quartic form over $K$.  In the classical treatments, the
coefficients of the form would be denoted $a_0$, $4a_1$, $6a_2$, $4a_3$,
and $a_4$.  We have chosen the simpler notation to be consistent with
\cite\BSD, and also because for later purposes (when $K=\Q$, or a
number field) the
integrality of the coefficients $a$, $b$, $c$, $d$, $e$ will be
important.  However, it is still useful to define the {\it weights\/}
of the coefficients $a$, $b$, $c$, $d$ and $e$ as $0$, $1$, $2$, $3$
and $4$ respectively, as indicated by the subscripts in the traditional
notation.  We will also denote the corresponding inhomogeneous
polynomial by $g(X)=g(X,1)$, which is a quartic except in the
degenerate case when $a=0$.

The group $\GL(2,K)$ acts on the set of binary quartics via
$$\align
  A=\mat(\alpha,\beta;\gamma,\delta)\colon g(X,Y)&\mapsto
g({\alpha X+\beta Y},{\gamma X+\delta Y}) \\
&=a^*X^4+b^*X^3Y+c^*X^2Y^2+d^*XY^3+e^*Y^4.
\endalign
$$
The coefficients $a^*$, $b^*$, $c^*$, $d^*$ and~$e^*$ of the transform
of $g$ are linear combinations of $a$, $b$, $c$, $d$, $e$ with
coefficients which are polynomials in the matrix entries.  We call two
quartics $g_1$ and $g_2$ {\it equivalent\/} if they are in the same
orbit under this action, and write this as $g_1\sim g_2$.

An {\it invariant of weight $w$ and degree $n$} of the binary quartic
$g(X,Y)$ is a homogeneous polynomial $I$, of degree $n$ in the
variables $a$, $b$, $c$, $d$ and $e$, satisfying \neweq\invdef
$$
   I(a^*,b^*,c^*,d^*,e^*) = \det(A)^w I(a,b,c,d,e) \tag\invdef
$$
for all transformation matrices $A$ in $\GL(2,K)$.  The degree $n$ and
weight $w$ are related: $w=2n$.  (For invariants of forms $g$ of
general degree, the corresponding relation is $2w/n=\deg(g)$.)

Each term of an invariant of degree $n$, as well as being homogeneous
of degree $n$, is also {\it isobaric\/} of weight $w(=2n)$, in the
sense that each term of $I(a,b,c,d,e)$ has the same weight $w$.  
This follows from invariance under diagonal matrices.

Given an isobaric homogeneous form $I(a,b,c,d,e)$, the condition that
$I$ should be an invariant is that it should be annihilated by two
differential operators:
$$
  \Omega I = 4a\frac{\d I}{\d b} + 3b\frac{\d I}{\d c} 
           + 2c\frac{\d I}{\d d} +  d\frac{\d I}{\d e} = 0
$$
and
$$
  \Omega^* I = 4e\frac{\d I}{\d d} + 3d\frac{\d I}{\d c} 
             + 2c\frac{\d I}{\d b} +  b\frac{\d I}{\d a} = 0.
$$
The second condition is redundant if the weight and degree of $I$ are
related by $w=2n$.  These follow from invariance under matrices of
the form $\mat(1,\beta;0,1)$ and $\mat(1,0;\gamma,1)$ respectively.

The two basic invariants for quartics are \neweq\Idef
$$
   I = 12ae-3bd+c^2 \tag\Idef
$$
of degree 2 and weight 4, and the so-called {\it catalecticant} \neweq\Jdef
$$
  J =  72ace+9bcd-27ad^2-27eb^2-2c^3 \tag\Jdef
$$
of degree 3 and weight 6.  The invariants of degree $n$ form a vector
space whose basis consists of the monomials $I^rJ^s$ where $r,s\ge0$
and $2r+3s=n$.  In particular, $I$ and $J$ are algebraically
independent (this is easy to see, by specializing $a=1$, $b=c=0$).
The discriminant $\Delta$ of the quartic has degree 6 and weight 12,
hence must be a linear combination of $I^3$ and $J^2$; we will take
\neweq\Deltadef
$$
  \Delta = 4I^3-J^2 \tag\Deltadef
$$
which is 27 times the usual discriminant.  The condition for $g(X,Y)$
to have no repeated factors is of course $\Delta\not=0$, and we will
assume that this condition holds throughout.

A fundamental question to ask of a given field $K$ is: given two
values $I$ and $J$ in $K$ satisfying $4I^3-J^2\not=0$, find all
quartics in $K[X,Y]$ with invariants $I$ and $J$, up to
$\GL(2,K)$-equivalence.  For a number field such as $\Q$ we might also
take integral $I$ and $J$ and ask for all integral quartics
$g(X,Y)\in\Z[X,Y]$.  Even over $\Q$\ this question is highly
non-trivial; as we shall see, a good algorithmic answer to this
problem forms a substantial part of the process of full 2-descent on
elliptic curves.

\neweq\pdef
\neweq\rdef
\neweq\qdef
\neweq\semisyz
\neweq\covardef
\neweq\gfourdef
\neweq\gsixdef
\neweq\covarsyz

As well as invariants we will also need to consider two related kinds
of objects: seminvariants and covariants.  A {\it seminvariant\/} is a
form $S$ in the variables $a$, $b$, $c$, $d$ and $e$ which is isobaric
and homogeneous and satisfies $\Omega S=0$.  Thus all invariants are
also seminvariants; but we also find three essentially new
seminvariant quantities: these are $a$ (the leading coefficient, of
degree 1 and weight 0), 
$$
  p=3b^2-8ac \tag\pdef
$$
of degree 2 and weight 2, and 
$$
  r=b^3+8a^2d-4abc \tag\rdef
$$
of degree 3 and weight 3.  For future reference we will also introduce
the further seminvariant $q$ defined by
$$
  q = \frac13(p^2-16a^2I) = 3b^4-16ab^2c+16a^2c^2+16a^2bd-64a^3e. \tag\qdef
$$
(The notation $p$, $q$, $r$ is not standard in the literature, but
will be used consistently throughout this paper.)  Just as all
invariants are polynomials in $I$ and $J$, all seminvariants are
polynomials in $I$, $J$, $a$, $p$ and $r$; however these five are not
algebraically independent, but are related by a syzygy:
$$
   p^3-48a^2pI-64a^3J = 27r^2. \tag\semisyz
$$
(In general, a {\it syzygy\/} is an equation of algebraic dependence
between invariants, seminvariants or covariants.)  This syzygy, and
its extension to covariants below (\covarsyz), will play an important
role later.

Seminvariants are unchanged by the substitution $X\mapsto X+Y$; it
follows that if a seminvariant is expressed in terms of the roots
$x_i$ of $g$, it can be written as a function of the leading
coefficient $a$ and the differences $x_i-x_j$ of the roots.
Conversely, every homogeneous function of the roots which can be
expressed as a function of the differences between the roots is
seminvariant (if multiplied by a suitable power of the leading
coefficient $a$ to make it integral); such a function of the roots is
called an ``irrational'' seminvariant unless it is also symmetric in
the roots, when it is ``rational'' and hence is an actual seminvariant
in the sense defined here.  We will make use of this observation in
the next section.

Finally, a {\it covariant of order $w$} of the binary quartic is a
form $C(a,b,c,d,e,X,Y)$, homogeneous separately in $X,Y$ and in
$a,b,c,d,e$, satisfying the following transformation
law for all $A=\mat(\alpha,\beta;\gamma,\delta)\in\GL(2,K)$:
$$
   C(a^*,b^*,c^*,d^*,e^*,X,Y) = \det(A)^w
   C(a,b,c,d,e,\alpha X+\beta Y,\gamma X+\delta Y). \tag\covardef
$$
There is a one-one correspondence between seminvariants and
covariants: if $C$ is a covariant of order $w$ then the leading
coefficient $S(a,b,c,d,e)=C(a,b,c,d,e,1,0)$ is a seminvariant.
Conversely, every seminvariant $S$ is the leading coefficient of a
unique covariant $C$: one says that $S$ is the {\it source\/} of $C$.
If $S$ has degree $n$ (in the coefficients $a$, $b$, etc.) and weight
$w$, then the degree (in $X$, $Y$) of the associated covariant $C$ is
$d=4n-2w$; for invariants this is 0, and the associated covariant is
just the invariant itself.  In general, the covariant associated to
the seminvariant $S$ is
$$
   C(X,Y) = \sum_{i=0}^d \frac{(\Omega^*)^i(S)}{i!}X^{d-i}Y^i.
$$
The seminvariant $a$ is the source of the original form $g$, which is
trivially a covariant of itself of order~0.  The seminvariant $p$
is the source of a quartic covariant $g_4$:
$$\align
  g_4(X,Y) &= (3b^2 - 8ac) X^4 
+ 4 (bc - 6ad)             X^3Y 
+ 2 (2c^2 - 24ae - 3bd)    X^2Y^2 \tag\gfourdef\\
&\quad+ 4 (cd - 6be)            X  Y^3 
+   (3d^2-8ce)                Y^4,
\endalign
$$
while the seminvariant $r$ leads to the sextic covariant $g_6$:
$$\align
  g_6(X,Y) &= (b^3 + 8a^2d - 4abc)   X^6 
+ 2 (16a^2e + 2abd - 4ac^2 + b^2c)  X^5 Y \tag\gsixdef\\
&\quad+ 5 (8abe + b^2d - 4acd)            X^4 Y^2
+ 20(b^2e - ad^2)                   X^3 Y^3 \\
&\quad- 5 (8ade + bd^2 - 4bce)            X^2 Y^4 
- 2 (16ae^2 + 2bde - 4c^2e + cd^2 ) X   Y^5 \\
&\quad-   (d^3 + 8be^2 - 4cde)                Y^6.
\endalign
$$
The syzygy between the seminvariants extends to a syzygy between the
covariants:
$$
   27g_6^2 = g_4^3 - 48Ig^2g_4 - 64Jg^3. \tag\covarsyz
$$
This is an identity in $X$ and $Y$; substituting $(X,Y)=(1,0)$, we
recover (\semisyz).

\beginsection{\Galois}
\head\Galois. The Resolvent Cubic \endhead

We keep the notation of the previous section.  Traditionally, the
invariant theory of quartics can be used to derive the solution of
quartics by radicals, by reducing the problem to that of solving an
associated cubic equation, called the resolvent cubic.   We will need
to make the relation between a quartic and its resolvent cubic rather
explicit, and to describe the situation in terms of Galois theory.
This will lead us to a simple criterion for two quartics with the same
invariants to be $\GL(2,K)$-equivalent.

Let $I$ and $J$ be the invariants of a quartic $g$ defined over
$K$, such that $\Delta=4I^3-J^2\not=0$.  Suppose that the leading
coefficient $a$ of $g$ is nonzero.  Then $g$ factorises over the
algebraic closure $\Kbar$ into 4 linear factors:
$$
   g(X,Y) = a \prod_{j=1}^{4} (X-x_iY).
$$
Here the $x_i$ are the four roots of the associated inhomogeneous
quartic polynomial $g(X)=g(X,1)$.   We will usually exclude as
degenerate the case of quartics which have a root in $K$ itself; these
form precisely one orbit, and include quartics with $a=0$ (with a root
at infinity), under the $GL(2)$ action.

Associated to the quartic $g$, or rather to its pair of invariants
$I$, $J$, we have the cubic polynomial \neweq\rescub
$$
  F(X) = X^3 - 3IX + J \tag\rescub
$$
which has nonzero discriminant $27\Delta$.  We are most interested in
the case where $F(X)$ is irreducible over $K$; this is because in our
application to 2-descent on elliptic curves, this case will arise when
the curve has no $K$-rational 2-torsion.  Hence we will make this
assumption.  In the sequel, this assumption is not strictly necessary,
though some of the discussion would need to be reformulated if it did
not hold; the fields $L$ and $K(\phi)$ defined below would need to be
replaced by semisimple $K$-algebras, but essentially the same results
would hold.  This still leaves two distinct cases, according to
whether the Galois group of $F$ is or is not cyclic.  For simplicity
of exposition we will assume that we are in the generic case where the
Galois group is the full symmetric group $S_3$; the groups $S_4$ and
$S_3$ which appear below would need to be replaced by the alternating
groups $A_4$ and $A_3$ in the non-cyclic case, and appropriate degrees
of field extensions halved.

Let $\phi$ be a root of $F(X)$ in $\Kbar$, so that $\phi$ satisfies
$\phi^3=3I\phi-J$ and the field $K(\phi)$ has degree 3 over $K$, with
normal closure $L$ of degree~6 (in the non-cyclic case).  We denote by
$\phi'$ and $\phi''$ the conjugates of $\phi$ in $L$ and view
$\Gal(L/K)=S_3$ as acting by permutations on the set
$\{\phi,\phi',\phi''\}$.  Note that $\Tr_{K(\phi)/K}(\phi)=0$ since
$F(X)$ has no $X^2$ term.  We emphasise that neither $\phi$ nor the
field $L$ depend on the particular quartic $g$, but only on the pair
of invariants $I$, $J$.

Let $M=K(x_1,x_2,x_3,x_4)$ be the splitting field of $g$ over $K$.  By
sending $x_1$ to infinity a simple calculation shows that 
$g(X)$ is equivalent to $F(X)$ over
$K(x_1)$; it follows that the degree
$[M:K(x_1)]=6$ with $\Gal(M/K(x_1))\cong S_3$.   For trivial quartics
(with a root in $K$ itself) this is still true, with $K=K(x_1)$ and
$[M:K]=6$; in the non-trivial case, however, it follows that $g$ is
irreducible over $K$ and $[M:K]=24$ with $\Gal(M/K)\cong S_4$.  We
view this Galois group as acting on the set of roots $x_i$ by
permutation in the natural way, once we have fixed an ordering of the
roots $x_i$.

It also follows from this discussion that $L\subset M$, so
that $S_3=\Gal(L/K)$ is a quotient of $S_4=\Gal(M/K)$.  There is only
one normal subgroup $H$ of $S_4$ such that $S_4/H\cong S_3$, namely
the Klein 4-group $V_4$, defined in terms of permutations as
$$
   V_4 = \{\id,(1 2)(3 4),(1 3)(2 4),(1 4)(2 3)\}.
$$
($S_4$ acts by conjugation on the non-identity elements of $V_4$; this
gives the homomorphism $S_4\to S_3$ with $V_4$ as kernel.)

Using this explicit description of $\Gal(M/L)$ as a subgroup of
$\Gal(M/K)$, we may easily write down elements of $L$ in terms of the
roots $x_i$.  We define \neweq\zdefx
$$
   z = a^2(x_1+x_2-x_3-x_4)^2; \tag\zdefx
$$
then permutations of the $x_i$ take $z$ to one of three values: $z$
itself, and the conjugate quantities \neweq\zdashdefx
$$
  z'  = a^2(x_1-x_2+x_3-x_4)^2 \quad\text{and}\quad
  z'' = a^2(x_1-x_2-x_3+x_4)^2. \tag\zdashdefx
$$

Since $z$ is an integral function of the root differences, it is an
example of an irrational seminvariant, as introduced in the previous
section.  Symmetric functions of $z$, $z'$, $z''$ are therefore
rational seminvariants.  In particular, the coefficients of the
minimal polynomial of $z$ are seminvariants.
\newprop\zminpoly

\proclaim{Proposition \zminpoly} The minimal polynomial of $z$
(defined in (\zdefx)) is \neweq\zpoldef
$$\align
  h(Z) = (Z-z)(Z-z')(Z-z'') &= Z^3 - pZ^2 + qZ -r^2 \tag\zpoldef\\
                 &= \left(\frac{4a}{3}\right)^3F\left(\frac{3Z-p}{4a}\right).
  \endalign
$$
Hence $z\in K(\phi)$; explicitly, $z=\frac13(4a\phi+p)$, and moreover
$N_{K(\phi)/K}(z) = r^2$.
\endproclaim

\demo{Proof}
The first equality can be obtained by manipulation of symmetric
polynomials: the coefficients are seminvariants of degrees 2, 4 and~6.
The second equality comes from expanding $F((3Z-p)/(4a))$ and using
the syzygies (\qdef) and (\semisyz) relating $q$ and $r^2$ to the
other seminvariants.  The relation between $z$ and $\phi$ follows
immediately.  This is to be interpreted as a generic relation, since
both $z$ and $\phi$ are only defined up to conjugacy: if we fix a
numbering of the roots $x_j$ we thereby fix an ordering of $z$ and its
conjugates, and we then choose the ordering of $\phi$ and its
conjugates correspondingly.
\qed\enddemo

We will call the quantity $z$ the {\it cubic seminvariant\/}
associated to the quartic $g$.  There are two crucial properties of
the cubic seminvariant to notice: as an element of the cubic extension
$K(\phi)$ it is {\it linear\/} in $\phi$, in the sense that when
expressed in terms of the $K$-basis $1$, $\phi$, $\phi^2$ for
$K(\phi)$ it has no $\phi^2$ term.  Secondly, its norm is a {\it
square\/} in $K$.  The latter fact is essentially due to the syzygy
(\semisyz).

Given $z\in K(\phi)^*$ with conjugates $z'$, $z''$, such that the norm
$N(z)=zz'z''=r^2$ is a square in $K$, the normal closure of
$K(\sqrt{z})$ will be a field $M$ which is an $S_4$ extension of $K$
containing $L$ and with $\Gal(M/L)=V_4$.  (In the degenerate case, $z$
is itself a square, and $M=L$.)  We may then construct a quartic $g$
having $M$ as its splitting field and $z$ as its cubic seminvariant by
working backwards: $g$ is not uniquely defined, but only up to
translation and scaling.  Choosing $a$ nonzero and $b$ arbitrarily, we
have the following explicit formulas:
\neweq\xfromz
$$\aligned
4ax_1 &= + \sqrt{z} + \sqrt{z'} - \sqrt{z''} - b, \\
4ax_2 &= + \sqrt{z} - \sqrt{z'} + \sqrt{z''} - b, \\
4ax_3 &= - \sqrt{z} + \sqrt{z'} + \sqrt{z''} - b, \\
4ax_4 &= - \sqrt{z} - \sqrt{z'} - \sqrt{z''} - b. \\
\endaligned\tag\xfromz
$$
Here the square roots may be chosen in any way such that the product
$\sqrt{z}\sqrt{z'}\sqrt{z''}=r$ (as opposed to $-r$).  For
compatibility with (\zdashdefx), we have arranged (\xfromz) so that
$\sqrt{z}=a(x_1+x_2-x_3-x_4)$, $\sqrt{z'}=a(x_1-x_2+x_3-x_4)$ and
$\sqrt{z''}=a(-x_1+x_2+x_3-x_4)$.

Although the field $M=K(x_1,x_2,x_3,x_4)$ obtained thus will always
have $L$ as its cubic resolvent subfield, it is important to realise
that the quartic $g$ with the $x_i$ as its roots will {\it not\/}
necessarily have invariants $I$ and $J$.  This will only occur if the
element $z$ used, as well as having square norm, is linear in $\phi$.
This rather unnatural condition can be interpreted as follows.  
The condition that $z$ be linear in $\phi$ can be written as
$$
   \frac{z-z''}{z'-z''} = \frac{\phi-\phi''}{\phi'-\phi''},
$$
and this common value is simply the cross-ratio
$$
   \frac{(x_1-x_3)(x_2-x_4)}{(x_1-x_4)(x_2-x_3)};
$$
this may be readily checked by calculation from (\xfromz).  Hence by
requiring $z$ to be linear in $\phi$, we are simply specifying a fixed
value for the cross-ratio of the roots of the associated quartic $g$,
namely the value of the cross-ratio of the roots of $F(X)$ (including a
``root at infinity'').

The fields involved are shown in the diagram, where the degrees
indicated are for the non-trivial non-cyclic case (where $g$ has no
root in $K$, and $\Gal(L/K)\cong S_3$). 

\medskip
\vbox{
\settabs10\columns
\+&&$M=K(x_1,x_2,x_3,x_4)$                  \cr
\+&\hfill6&&\hfill4&&&&           $V_4$     \cr
\+&&&&$L=K(\phi,\phi',\phi'')$              \cr
\+&$K(x_1)$&&&2&&&                &$S_4$    \cr
\+&&&$K(z)=K(\phi)$                         \cr
\+&\hfill4\hfill&&3&&&&           $S_3$     \cr
\+&&$K$                                     \cr
}
\medskip

In order to have an unambiguous definition of $z$ applicable in all
cases, we will in fact use the equation \neweq\zdefphi
$$
   z = \frac{4a\phi + p}{3} \tag\zdefphi
$$
to define $z$ as an element of $K(\phi)$ for all quartics $g$ with
invariants $I$ and $J$, where $a$ and $p$ are the seminvariants
attached to $g$.  This includes degenerate cases, such as when $a=0$
(so that $g$ is in fact a cubic): then $p=3b^2$ and $z=b^2$ is
actually in the ground field $K$.  The fact that $z$ is a square here
is a special case of the following fundamental result.
\newprop\zequivprop

\proclaim{Proposition \zequivprop}  

\part1 $z$ is a square in $K(\phi)$ if and only if $g$ has a linear
factor in $K[X,Y]$ (that is, one of the roots $x_i$ is in 
$K\cup\{\infty\}$).

\part2 Let $g$ and $g^*$ be quartics with the same invariants $I$
and $J$, with cubic seminvariants $z$ and $z^*$.  Then 
$$
  g_1 \sim g_2 \iff zz^*\in(K(\phi)^*)^2.
$$
\endproclaim

\demo{Proof}
\part1 Let $h(Z)=Z^3-pZ^2+qZ-r^2$ be the minimum polynomial of $z$.
The condition that $z$ be a square in $K(\phi)$ is that $h(Z^2)$
should factorise over $K$ as $h(Z^2)=-h_0(Z)h_0(-Z)$.  Writing
$h_0(Z)=Z^3+uZ^2+vZ+r$ and equating coefficients, we find that
$v=(u^2-p)/2$, where $u$ satisfies the quartic equation
$$ 
   \tilde{g}(u) = (u^2-p)^2 -8ru -4q =0. 
$$ 
Manipulation now shows that 
$$
    \tilde{g}(u) = \frac{1}{a}g(u+b,-4a), 
$$ 
from which the result follows when $a\not=0$.  If $a=0$, we have
already observed that $z=b^2$, and $h(Z)=(Z-b^2)^2$ in this case.

\part2 Suppose that $g\sim g^*$ via a matrix $A$ in $\GL(2,K)$.
Since $\GL(2,K)$ is generated by matrices of the form
$$
   \mat(\alpha,0;0,\delta), \quad \mat(1,\beta;0,1)
   \quad\text{ and }\quad \mat(0,1;1,0),
$$
it suffices to show that $z=z^*$ (modulo squares) in these three
cases.  In the first case, $z^*=(\alpha/\delta)^2z$; in the second, $z
^*=z$ (clear from the definition). In the third case, direct
calculation shows that $zz^*=w^2$ where
$$
  9w = 2\phi^2-2c\phi+9bd-4c^2;
$$
here $b$, $c$, $d$ refer as usual to the coefficients of $g$.

For the converse, suppose that $zz^*$ is a square in $K(\phi)$, where
both $z$ and $z^*$ are linear in $\phi$.  The splitting field $M$ of
$g$ is the same as that of $g^*$, since it is the normal closure of
$K(\phi)(\sqrt{z})$.  Let the roots of $g$ be $x_i$, $i=1,\ldots,4$.
As observed above, the cross-ratio of the $x_i$ is equal to $\frac{z-z''}
{z'-z''} = \frac{\phi-\phi''}{\phi'-\phi''}$, and the
roots $y_i$ of $g^*$ have the same cross-ratio; hence there is a
matrix $A\in\GL(2,M)$, uniquely determined up to scalar
multiple, such that $A(x_i)=y_i$ for $i=1,\ldots,4$.

Finally we must show that $A$ can be taken to have entries in $K$, for
then it is easily seen that $g\sim g^*$ via $A^{-1}$.  We may scale $A$ so
that one of its entries is~1; then it suffices to show that for all
$\sigma\in\Gal(M/K)$ we have $A^\sigma=A$ (up to scalar multiple, and
hence exactly, since $1^\sigma=1$).  Now $\sigma$ acts on the $x_i$
via some permutation $\pi\in S_4$: $x_i^\sigma=x_{\pi(i)}$.  Using the
explicit expressions for the $x_i$ and $y_i$ in terms of $\sqrt{z}$,
$\sqrt{z ^*}$ and their conjugates as in (\xfromz), and the fact that
$z=w^2z^*$ for some $w\in K(\phi)$, it follows that $\sigma$ acts on
the $y_i$ via the {\it same\/} permutation: $y_i^\sigma=y_{\pi(i)}$.
Now applying $\sigma$ to the four equations $A(x_i)=y_i$, we obtain
$A^\sigma(x_{\pi(i)})=y_{\pi(i)}$ for all $i$, and hence (permuting
the equations), $A^\sigma(x_i)=y_i$ for all $i$.  By uniqueness of $A$
up to scalar multiple, we have $A^\sigma=A$ as required.
\qed\enddemo

Using this proposition, we can derive a simple test for whether a given
pair of quartics $g$, $g^*$ is equivalent.  We form the two cubic
seminvariants $z$, $z^*$ and test whether $zz^*$ is a square in
$K(\phi)$.  This condition turns out to be simply whether a third
quartic has a root in $K$, as in Proposition \zequivprop (1).
\newprop\zequivalg

\proclaim{Proposition \zequivalg} Let $g_1$, $g_2$ be quartics over the
field $K$, both having the same invariants $I$ and $J$.  Then $g_1\sim
g_2$ if and only if the quartic $u^4-2pu^2-8ru+s$ has a root in $K$,
where
$$\align
p &= (32a_1a_2I + p_1p_2)/3, \\
r &= r_1r_2, \\
\intertext{and}
s &= ( 64I(a_1^2p_2^2 + a_2^2p_1^2 + a_1a_2p_1p_2)
   - 256a_1a_2J(a_1p_2+a_2p_1) - p_1^2p_2^2) /27.
\endalign
$$
Here, $a_i$, $p_i$ and $r_i$ are the seminvariants attached to $g_i$
for $i=1,2$.
\endproclaim

\demo{Proof}
We compute the minimum polynomial $h(Z)$ of $z_1z_2$, as the characteristic
polynomial of the matrix $A_1A_2$, where $A_i$ is the characteristic
matrix of $z_i$:
$$
   A_i = \frac13\left(4a_i\pmatrix0&0&-J\cr1&0&3I\cr0&1&0\endpmatrix
       + p_i\pmatrix1&0&0\cr0&1&0\cr0&0&1\endpmatrix\right);
$$
here, $\pmatrix0&0&-J\cr1&0&3I\cr0&1&0\endpmatrix$ is the characteristic
matrix of $\phi$.  Writing $h(Z)=Z^3-pZ^2+qZ-r^2$, we have $p$ and $r$
as in the statement, and $s=p^2-4q$.  The condition that the roots
of $h(Z)$ be squares is, as in the proof of Proposition \zequivprop
(1), the condition that the given quartic $(u^2-p)^2-8ru-4q$ has a
root in $K$.
\qed\enddemo

We remark that this proposition gives a very simple, algebraic test
for equivalence of quartics, over any field.  Both the tests for
triviality and equivalence only rely on being able to determine
whether a certain quartic with coefficients in $K$ has a root in $K$.
This test is much simpler to implement than the test presented in
\cite\BSD\ and \cite\JCbook\ for $K=\Q$, and in \cite\PSthesis\ for
real quadratic fields.   In the real quadratic case the new test also
saves some computation time, particularly for curves of higher rank
(where there are more possible equivalences to check), and is expected
that the saving would be even greater for fields of higher degree.


\subhead Interpretation in terms of Galois cohomology \endsubhead

Fix a field $L$ which is Galois over $K$ with group either $S_3$ or
$A_3$.  Let $\phi\in L$ be of degree~3 over $K$, so that $L$ is the
Galois closure of $K(\phi)$.  There is a bijection between (a)
$S_4${} (respectively $A_4$) extensions $M$ of $K$ containing $L$
with $\Gal(M/L)\cong V_4$, and (b) nontrivial elements $z$ of the
group
$$
  H = \ker \left( N_{K(\phi)/K} : K(\phi)^*/\left(K(\phi)^*\right)^2 
              \to K^*/\left(K^*\right)^2 \right).
$$
The group $H$ may be interpreted as a Galois cohomology group, namely
\neweq\gcisom
$$
  H \cong H^1(\Gal(\Kbar/K),V_4) \tag\gcisom
$$
where the action of $\Gal(\Kbar/K)$ on $V_4$ is via its quotient
$\Gal(L/K)$ which acts on $V_4$ by permuting its nontrivial elements. 

We briefly indicate the construction of the isomorphism (\gcisom).
Given $z_1=z\in K(\phi)$ with square norm representing a nontrivial
element of $H$, with conjugates $z_2$ and $z_3$, for each
$\sigma\in\Gal(\Kbar/K)$ we have $z_i^\sigma =
z_{\overline{\sigma}(i)}$ where $\sigma\mapsto\overline{\sigma}$ is
the quotient map $\Gal(\Kbar/K)\to\Gal(L/K)$, and we have identified
$\Gal(L/K)$ with $S_3$ (or $A_3$).  Now for $i=1,2,3$, fix a square
root $\sqrt{z_i}\in\Kbar$; then for $\sigma\in\Gal(\Kbar/K)$ we have
$$
  (\sqrt{z_i})^\sigma =
  \epsilon_i(\sigma)\sqrt{z_{\overline{\sigma}(i)}}
$$ 
where $\epsilon_i(\sigma)=\pm1$ and
$\epsilon_1(\sigma)\epsilon_2(\sigma)\epsilon_3(\sigma)=+1$ since
$\sqrt{z_1}\sqrt{z_2}\sqrt{z_3}\in K$.   Now
$$
   \left\{(\epsilon_1,\epsilon_2,\epsilon_3)\in\{\pm1\}^3\mid
\epsilon_1\epsilon_2\epsilon_3=+1\right\} \cong V_4,
$$
and the map
$\sigma\mapsto(\epsilon_1(\sigma),\epsilon_2(\sigma),\epsilon_3(\sigma))$
is the desired $1$-cocycle in $H^1(\Gal(\Kbar/K),V_4)$.

Conversely, given a nontrivial $1$-cocycle in
$H^1(\Gal(\Kbar/K),V_4)$, its restriction to $\Gal(\Kbar/L)$ is just a
homomorphism $\Gal(\Kbar/L)\to V_4$, since $\Gal(\Kbar/L)$ acts
trivially on $V_4$ by definition, and is in fact a surjective
homomorphism.  Hence its kernel cuts out a $V_4$ extension $M$ of $L$
which is Galois over $K$.  This in turn determines a well-defined
class $z$ in $H$ as required.

For these cohomology computations it is worth noticing that the
restriction map
$$
H^1(\Gal(\Kbar/K),V_4) \to H^1(\Gal(\Kbar/L),V_4)
$$
is injective, since (by the restriction-inflation exact sequence) its
kernel is $H^1(S_3,V_4)$ which is trivial, as a
simple direct calculation shows.

\medskip

To summarise this section, we have shown that there exists a bijection
between (a) quartics $g$ over $K$ with invariants $I$, $J$, modulo
$\GL(2,K)$-equivalence; and (b) nonzero elements $z\in K(\phi)$ which
are linear in $\phi$ and whose norms are squares in $K$, modulo
squares in $K(\phi)$.  The bijection is defined by associating to a
quartic $g$ with seminvariants $a$ and $p$ the cubic seminvariant $z
=(4a\phi+p)/3$.  Each of these sets in turn can be identified with a
subset of the Galois cohomology group $H^1(\Gal(\Kbar/K),V_4)$,
depending on the specific generator $\phi$ for $K(\phi)$, or
equivalently on the specific invariants $I$ and $J$.  We will return
to this in Section \GStructure.

In the next section we will introduce the third ingredient, which
relates both these sets to the group of points on an elliptic curve
defined over $K$.


\beginsection{\Descent}
\head\Descent. 2-Descent on Elliptic Curves \endhead

We keep the notation of the previous sections: $I$ and $J$ are
elements of the field $K$ satisfying $\Delta=4I^3-J^2\not=0$, and
$F(X)=X^3-3IX+J$ is irreducible with root $\phi$.  Set
$\Ft(X)=-27F(-X/3)$, and let $\EIJ$ be the elliptic curve
\neweq\EIJdef
$$
   \EIJ:\quad Y^2 = \Ft(X) = X^3 -27IX-27J. \tag\EIJdef
$$
There is a close connection between (equivalence classes of) quartics
$g$ with invariants $I$, $J$ and arithmetic properties of $\EIJ(K)$.
Since the invariants $I$ and $J$ will remain fixed throughout, we will
sometimes drop the subscript and refer to the elliptic curve simply
as~$E$.

The syzygy (\semisyz) may be expressed as 
$$ 
   (27r)^2 = (4a)^3 \Ft\left(\frac{3p}{4a}\right), 
$$ 
so that 
$$
   (X,Y)=\left(\frac{3p}{4a},\frac{27r}{(4a)^{3/2}}\right)
$$ 
is a point on the curve $E$.  (We will also use projective
coordinates on $E$, in which this point is
$(X:Y:Z)=(6p\sqrt{a}:27r:8a\sqrt{a})$.)  This point is not $K$-rational
unless $a$ is a square in $K$.  This leads to the fundamental
question: is there a quartic equivalent to $g$ whose leading
coefficient is a square?  If this is the case, we call the quartic $g$
{\it soluble\/}.  

Associated to the quartic $g$ is the plane curve $\CC$:
\neweq\CCdef
$$ 
    \CC:\quad Y^2 = g(X) = aX^4+bX^3+cX^2+dX+e. \tag\CCdef 
$$ 
This affine curve is nonsingular, and has genus one. If $a\not=0$ it
has a double point at infinity, which can be desingularised by taking
the affine curve\neweq\CCdefa
$$
   \CC^*:\quad V^2 = g(1,U) = eU^4+dU^3+cU^2+bU+a \tag\CCdefa 
$$ 
and identifying the points $(X,Y)$ on $\CC$ with $X\not=0$ with the
points $(U,V)$ on $\CC^*$ with $U\not=0$ via $(U,V)=(1/X,Y/X^2)$.  The
double point at infinity on the projective closure of $\CC$ is
replaced by the two points $(0,\pm\sqrt{a})$ on $\CC^*$, which are
$K$-rational if and only if $a$ is a square in $K$.  For simplicity we
will use $\CC$ to denote the desingularised projective curve, bearing
in mind that it has two points at infinity which are rational if and
only if $a$ is a square.

The following result is now straightforward.
\newprop\ratpt

\proclaim{Proposition \ratpt} The curve $\CC$ has a $K$-rational point
if and only if there is a quartic equivalent to $g$ whose leading
coefficient is a square.
\endproclaim

\demo{Proof} If $a$ is a square then the points at infinity on $\CC$
are $K$-rational.  Conversely, if $\CC$ has a $K$-rational point, we
may apply a projective transformation to send its $X$-coordinate to
infinity, thereby replacing $g$ by an equivalent quartic whose leading
coefficient is a square.\qed
\enddemo

The seminvariant syzygy (\semisyz) only determined a rational point on
$\EIJ(K)$ when $a$ was a square.  Using the covariant syzygy
(\covarsyz), we can define a rational map $\CC\to E$ defined over
$K$. This can be derived by taking a rational point on $\CC(K)$,
mapping the $X$-coordinate to infinity, thus replacing the quartic $g$
by a quartic whose leading coefficient is a square, and writing down
the corresponding seminvariant syzygy.
\newprop\ratmapprop

\proclaim{Proposition \ratmapprop} The map 
$$
  \xi:\quad (x:y:z) \mapsto (6yzg_4(x,z) : 27g_6(x,z) : (2yz)^3)
$$
is a rational map from $\CC$ to $\EIJ$ of degree~$4$.
\endproclaim

\demo{Proof}
The covariant syzygy (\covarsyz) may be written
$$
   (27g_6(X,Z))^2 = (4g(X,Z))^3 \Ft\left(\frac{3g_4(X,Z)}{4g(X,Z)}\right);
$$
given $y^2z^2 = g(x,z)$ and substituting, this becomes
$$
   \left(\frac{27g_6(x,z)}{(2yz)^3}\right)^2 = 
                   \Ft\left(\frac{3g_4(x,z)}{(2yz)^2}\right), 
$$
so that $(6yzg_4(x,z) : 27g_6(x,z) : (2yz)^3) \in E(K)$ as required.

To see that the degree is~4, given $(X:Y:Z)$ on $E(K)$ in
projective coordinates, $(x:z)$ must be a solution to the quartic
equation $4Xg(x,z)-3Zg_4(x,z)$, and then $y$ is uniquely determined. 

Note that for $i=1,2,3,4$ we have $\xi((x_i:0:1)) = (0:1:0)$, the
point at infinity on $E$.
\qed\enddemo 

In our applications, we will only be interested in those quartics $K$
which are soluble, so that the curve $\CC$ has a $K$-rational point
and is thus itself an elliptic curve.  

For $i\in\{1,2,3,4\}$ we also have a map $\theta_i$ from $\CC$ to
$E$ which is a birational isomorphism defined over $K(x_i)$ .
Since $g\sim\Ft$ it is easy to see that a transformation
$A\in\GL(2,K(x_1))$ such that $A(x_1)=\infty$ takes the other roots
$x_j$ for $2\le j\le4$ to the roots of $\Ft$ in some order; these
roots are $-3\phi$ and its conjugates.  Hence $\theta_1$ takes
$(x_1,0)$ on $\CC$ to the point at infinity on $E$, and the other
points $(x_j,0)$ (for $j>1$) to the three points $(-3\phi,0)$ of
order~2 on $E$.  Similarly for the conjugate maps $\theta_j$.  
We set $\theta=\theta_1$.  The relation between $\theta$
and $\xi$ is given by the following result.

\newprop\comdiagprop
\proclaim{Proposition \comdiagprop} 
\part1 The following diagram, in which the horizontal map is
multiplication by~$2$ on $E$, commutes.  \neweq\comdiag 
$$
\matrix\format\r&\c\quad&\c\quad&\l\\
     & E      &\buildrel{[2]}\over\longrightarrow & E     \\
\\
\theta&\left\uparrow\vrule height10pt depth10pt width0pt\right. &\quad\longnearrow\xi   \\
\\
             &\CC                                                  \\
\endmatrix
\tag\comdiag
$$
\comment
\CD  E @>[2]>> E \\ @A\theta AA \\ \CC \\ \endCD
\endcomment

\part2
Let $P\in E(\Kbar)$ and let $[2]^{-1}(P) = \{Q_1,Q_2,Q_3,Q_4\}
\subset E(\Kbar)$; then $\xi^{-1}(P) = \{\theta^{-1}(Q_i)\mid 1\le
i\le 4\}$.  If in fact $P\in E(K)$, then (with a suitable numbering)
$Q_i$ is defined over $K(x_i)$ for $1\le i\le4$.

\part3
For each $\sigma\in\Gal(\Kbar/K)$, there exists $T_\sigma\in E[2]$
such that $\theta^\sigma(R)=\theta(R)+T_\sigma$ for all
$R\in\CC(\Kbar)$. 

\part4
The image of $\CC(K)$ under $\xi$ is a complete coset of $[2]E(K)$
in $E(K)$.
\endproclaim

\demo{Proof} 

\part1 Define $\mu=\xi\circ\theta^{-1}:E\to E$.  Then $\mu$ has
degree~4, and $\mu$ maps the four 2-torsion points $E[2]$ to~0, so
$\mu$ must be multiplication by~2 on $E$.

\part2 For $R\in\CC(\Kbar)$ we have $\xi(R)=P \iff [2]\circ\theta(R)=P
\iff \theta(R) = Q_i$ for some $i$.  If $P\in E(K)$ then if we set
$Q_i=\theta_i(P)$, we have $Q_i\in E(K(x_i))$ since $\theta_i$ is
defined over $K(x_i)$.  The result follows, since the four fields
$K(x_i)$ are distinct (under our permanent assumption that the cubic
polynomial $F(X)$ is irreducible over $K$).

\part3 Fix $\sigma\in G=\Gal(\Kbar/K)$.  Consider the map $\CC\to E$
defined by $R\mapsto\theta^\sigma(R)-\theta(R)$.  The image is
contained in $E[2]$ since $[2](\theta^\sigma(R)-\theta(R)) =
[2]\theta^\sigma(R) - [2]\theta(R) = \xi^\sigma(R) - \xi(R) = 0$,
since both $[2]$ and $\xi$ are defined over $K$.  But maps between
curves are either constant or surjective; it follows that $T_\sigma =
\theta^\sigma(R)-\theta(R)\in E[2]$ is independent of $R$.

\part4 Let $R_1$, $R_2\in\CC(K)$.  Then for all $\sigma\in G$,
$(\theta(R_2)-\theta(R_1))^\sigma - (\theta(R_2)-\theta(R_1)) =
(\theta^\sigma(R_2)-\theta(R_2)) - (\theta^\sigma(R_1)-\theta(R_1)) =
T_\sigma-T_\sigma = 0$ (using (3)).  Hence $\theta(R_2)-\theta(R_1)\in
E(K)$, so $\xi(R_2)-\xi(R_1) = [2](\theta(R_2)-\theta(R_1)) \in
2E(K)$.  Thus the image of $\CC(K)$ under $\xi$ is contained in a
single coset of $2E(K)$ in $E(K)$.

Conversely, if $R\in\CC(K)$ with $P=\xi(R)\in E(K)$, then let $Q\in
E(K)$ and set $R'=\theta^{-1}(\theta(R)+Q)$.  Then $\xi(R') =
[2](\theta(R)+Q) = \xi(R)+2Q = P+2Q$, and $R'\in\CC(K)$ since 
$\theta^\sigma(R') = \theta^\sigma\theta^{-1}(\theta(R)+Q) =
\theta(R)+Q+T_\sigma = \theta^\sigma(R) + Q  = (\theta(R)+Q)^\sigma =
(\theta(R'))^\sigma = \theta^\sigma((R')^\sigma)$, and hence
$R'=(R')^\sigma$ for all $\sigma\in G$.\qed
\enddemo

\remark{Remark} A diagram such as (\comdiag) is called a {\it
$2$-covering\/} of $\EIJ$.  There is a notion of equivalence of
2-coverings, which here corresponds exactly to replacing the quartic
$g$ defining $\CC$ with an equivalent quartic.  Thus there is an
injection from equivalence classes of quartics with invariants $I$ and
$J$ to 2-coverings of the elliptic curve $\EIJ$.  Unfortunately, this
is not in general a bijection: there exist 2-coverings which cannot be
represented by quartics in this way.  However, when $K$ is a number
field, then all 2-coverings which are everywhere locally soluble are
representable by quartics (see Lemma 1 of \cite\BSD), and this is the
case of most interest to us.  We discuss this question further in the
next section.
\endremark

Finally in this section, we have a result which will have important
implications for the efficient practical implementation of a
$2$-descent algorithm over number fields.

\newprop\extfield
\proclaim{Proposition \extfield}
With the same notation as above, let $K'$ be an extension field of
$K$.  Then there is a bijection between
\part{i} points $Q\in E(K')$ with $[2]Q=P$; and
\part{ii} roots of $g(X)$ in $K'$.
\endproclaim

\demo{Proof}
This follows immediately from Proposition \comdiagprop (2).
\qed
\enddemo

For example, if $K$ is a subfield of $\R$ (such as $\Q$) we can take
$K'=\R$ here.  If $\Delta<0$, then $E(\R)$ is connected and isomorphic
to the circle group, hence 2-divisible with one 2-torsion point.  It
follows that in this case all quartics will have exactly two real
roots.  These are the ``Type 3'' quartics of \cite\BSD.  On the other
hand, if $\Delta>0$ then $E(\R)$ has two components, the connected
component of the identity $E^0(\R)=[2]E(\R)$ and the ``egg-shaped''
component $E(\R)-E^0(\R) = E(\R)-[2]E(\R)$.  All the 2-torsion is real
in this case.  Thus the quartics are of two types here: ``Type 2''
quartics with four real roots, giving points on $E(K)\cap E^0(\R)$,
and ``Type 1'' quartics with no real roots, giving points on $E(K)-
E^0(\R)$.  If $E(K)\subset E^0(\R)$, then there will be no Type 1
quartics, while otherwise there is a bijection between the (soluble)
Type 1 quartics and the soluble Type 2 quartics.  We can make use of
this observation in a practical algorithm, where we search separately
for quartics of each type depending on the sign of the discriminant
$\Delta$.

Similar comments apply to the $p$-adic completions of $K$.  In the
sequel to this paper we will discuss such implementation details more
fully.

\beginsection{\GStructure}
\head\GStructure. Galois Cohomology and Group Structure \endhead

It is easy to see that the map $\sigma\mapsto T_\sigma$ used in the
proof of Proposition~\comdiagprop\ is a cocycle, representing an
element of the Galois cohomology group $H^1(\Gal(\Kbar/K),E[2])$, and
is in fact the image of $P=\xi(R)$ under the connecting homomorphism
$\delta$ in the long exact sequence of Galois cohomology:
$$
   0 \hookrightarrow E(K)/2E(K) \overset\delta\to\longrightarrow H^1(\Gal(\Kbar/K),E[2])
                                \longrightarrow H^1(\Gal(\Kbar/K),E)[2].
$$
This cocycle is independent of $R\in\CC(K)$; changing $\theta$ to a
different isomorphism $\CC\to E$ which makes (\comdiag) commute has the
effect of replacing the cocycle $T_\sigma$ by a cohomologous cocycle:
in fact, any such isomorphism must have the form $\theta^\tau$ for
some $\tau\in\Gal(\Kbar/K)$, and the effect of replacing $\theta$ by
$\theta^\tau$ is to replace $T_\sigma$ by $T_\sigma +
(T_\tau^\sigma-T_\tau)$.

There is a bijection between the set of equivalence classes of
2-coverings of $E$ and $H^1(\Gal(\Kbar/K),E[2])$.  In the application
to 2-descent, one is only interested in the subgroup of
$H^1(\Gal(\Kbar/K),E[2])$ coming from $K$-rational points on $E$ (the
image of $\delta$). These correspond to 2-coverings $\CC$ which are
soluble (meaning $\CC(K)\not=\emptyset$); such cocycles become trivial
in $H^1(\Gal(\Kbar/K),E)$, as is evident from their representation as
the coboundary $Q^\sigma-Q$ with $Q=\theta(R)\in E(\Kbar)$.  When $K$
is a number field, one can often only determine the 2-coverings which
are everywhere locally soluble (meaning $\CC(K_\P)\not=\emptyset$ for
all completions $K_\P$ of $K$ at primes $\P$ of $K$, including the
infinite primes).  As remarked above, these are all represented by
quartics.

It is not true in general that the subset of elements of
$H^1(\Gal(\Kbar/K),E[2])$ representable by quartics with a fixed pair
of invariants $I,J$ is a subgroup (see below for an example).  This
cohomology group does not depend on the particular elliptic curve
$E=\EIJ$, but rather only on its 2-division field $L$; for if $E_1$
and $E_2$ are two curves defined over $K$ with the same 2-division
field, then the 2-torsion subgroups $E_1[2]$ and $E_2[2]$ are
isomorphic as Galois modules for $\Gal(\Kbar/K)$, so we may identify
$H^1(\Gal(\Kbar/K),E_1[2])$ and $H^1(\Gal(\Kbar/K),E_2[2])$; however
the curves will (in general) have different invariants $I,J$, and the
subsets of those elements of $H^1(\Gal(\Kbar/K),E[2])$ which can be
represented by quartics with each pair of invariants will be
different.

The obstruction to an arbitrary 2-covering $\CC$ of a given curve $E$
being representable by a quartic is that, as an algebraic curve, $\CC$
may have no positive $K$-rational divisor of degree~2.  If $\CC$ has
such a divisor, then a straightforward application of the Riemann-Roch
Theorem shows that $\CC$ has an equation of the form $y^2=$ quartic;
see \cite\BSD, Lemma 2.  In \cite\BSD, Lemma 1 it is shown (though not
by explicit equations) that this obstruction is represented by the
non-existence of a $K$-rational point on a certain curve of genus~0
defined over $K$, associated with the 2-covering.  Using the Galois
theory developed in Section~\Galois, we can see this obstruction
explicitly.

To a 2-covering $\CC$ of $E$ we have associated an element of
$H^1(\Gal(\Kbar/K),E[2])$ and also a ``cubic seminvariant'' $z\in
K(\phi)$ of square norm, uniquely determined modulo squares.  Here the
generator $\phi$ of the cubic field $K(\phi)$ has trace~0 and
determines the invariants $I$ and~$J$ via its minimal equation
$\phi^3=3I\phi-J$.  For the 2-covering to be representable by a
quartic with invariants $I,J$, it is necessary and sufficient that we
may choose a representative for the coset $z\left(K(\phi)^*\right)^2$
which is linear in $\phi$.  Set $z=a+b\phi+c\phi^2$ and
$z_1=u+v\phi+w\phi^2$ with $a$, $b$, $c$, $u$, $v$ and~$w\in K$.
Expanding $zz_1^2$, the coefficient of $\phi^2$ is a quadratic form
$Q(u,v,w)$ with coefficients which are functions of $I,J,a,b,c$:
$$\multline
   Q(u,v,w) = cu^2 + (a+3cI)v^2 + (3I(a+3cI)-bJ)w^2 \\
+ 2buv +2(3bI-cJ)vw + 2(a+3cI)uw.
\endmultline
$$
Set $\alpha=3c^2I+ac-b^2$ and $\beta=ab-c^2J$, and suppose that
$N(z)=r^2$; then with the linear change of variables
$u_1=cu+bv+(3cI+a)w$, $v_1=\alpha v-\beta w$, $w_1=rw$, we find that 
$$
 \tilde{Q}(u_1,v_1,w_1) = \alpha c Q(u,v,w)
                        = \alpha u_1^2 + v_1^2 - cw_1^2. 
$$
We require a nontrivial solution $(u,v,w)\not=(0,0,0)$ to
$Q(u,v,w)=0$.  This equation is evidently the genus zero curve of
\cite\BSD, Lemma~1.  If $K$ is a number field and $z$ represents a
2-covering which is everywhere locally soluble, then $Q(u,v,w)=0$ will
have points everywhere locally, and hence globally by the Hasse
principle, so a solution will exist over $K$.  Thus we will be able to
find $z_1\in K(\phi)$ such that $zz_1^2$ is linear in $\phi$, from
which we may construct a quartic to represent the 2-covering as in
Section~\Galois.

We can now express the problem of whether the subset of
$H^1(\Gal(\Kbar/K),E[2])$, consisting of cocycles for which the
corresponding 2-covering can be represented by quartics with
invariants $I,J$, is closed under multiplication, in purely algebraic
terms.  Let $K(\phi)$ be a cubic extension of the field $K$.  Let $H$
be the subgroup of $K(\phi)^*/(K(\phi)^*)^2$ consisting of those
cosets whose representative elements $z$ have square norm in $K$.
From Section~\Galois, we know that there is a bijection between
(nontrivial elements of) $H$ and the set of $S_4$ (respectively,
$A_4$) extensions $M$ of $K$ containing the Galois closure $L$ of
$K(\phi)$, according as $\Gal(L/K)$ is isomorphic to $S_3$ or $A_3$
respectively.  We also have a bijection between $H$ and
$H^1(\Gal(\Kbar/K),V_4)$, where the action of $\Gal(\Kbar/K)$ on $V_4$
is via its $S_3$ (respectively $A_3$) quotient $\Gal(\Kbar/L)$ which
acts faithfully on $V_4$ by permuting its non-identity elements.

Fixing a generator $\phi$ for $K(\phi)$ with trace~0, we determine
elements $I$ and $J$ of $K$ such that $\phi^3=3I\phi-J$.  Then $M$ is
the splitting field of a quartic $g(X)\in K[X]$ with invariants $I,J$
if and only if it corresponds to an element of $H$ which has a
representative which is linear in $\phi$.  The question is then: is
the subset of such elements of $H$ a subgroup?  

%Fixing $\phi$ fixes $I$ and~$J$ and hence the elliptic curve $\EIJ$,
%and
%$$
%   H^1(\Gal(\Kbar/K),E[2]) \cong H^1(\Gal(\Kbar/K),V_4).
%$$
%

To see that the answer to this question may be negative, let
$\phi=\root3\of2$, and set $z_1=3(1+\phi)$ and $z_2=10(2+\phi)$.  Then
$z_1$ and $z_2$ have square norms $3^4$ and $10^4$ respectively.
Setting $z_3=z_1z_2 = 30(2+3\phi+\phi^2)$, we can try to adjust $z_3$
modulo squares to eliminate the $\phi^2$ term.  This leads to the
quadratic form
$$\align
  Q(u,v,w) &= u^2+6uv+2v^2+4uw+4vw+6w^2\\
           &= (u+3v+2w)^2 + 2(w-2v)^2 - 15v^2,\\
\endalign
$$
which is 2-adically and 5-adically insoluble.  Hence there are
two quartics over $\Q$, with invariants $I=0$ and $J=-2$, for which
there is no product with these invariants.  The associated elliptic
curve $E$ is $Y^2=X^3+54$ with infinite cyclic Mordell-Weil group $E(\Q)$;
the quartics are $g_1$ with coefficients
$\frac1{108}(243,0,-54,24,-1)$, which is soluble and leads to the
generator $(X,Y)=(3,9)$ of $E(\Q)$, and $g_2$ with coefficients
$\frac1{90}(675,0,-90,20,-1)$ which is insoluble in $\Q_2$ and $\Q_5$.

Hence, in general, the $\GL_2$-equivalence classes of quartics with a
fixed pair of invariants $I,J$ in $K$ cannot be made into an
elementary abelian 2-group.  However we do have a partial product,
which can be useful.

Let $g_1$, $g_2$ and~$g_3$ be three quartics all with the same
invariants $I$ and~$J$.  We say that $g_1*g_2=g_3$ if the associated
cubic seminvariants satisfy $z_1z_2=z_3\pmod{(K(\phi))^2}$.  Note that
by Proposition~\zequivprop, this relation is well-defined on
equivalence classes of quartics.  We can test the relation
$g_1*g_2=g_3$ in practice, since the condition $z_1z_2z_3=$ square is
equivalent to the existence of a root in $K$ of a certain fourth
quartic over $K$ (just as Proposition~\zequivalg\ gave a test for the
equivalence of quartics, following Proposition~\zequivprop). 

Given two quartics $g_1$ and $g_2$ with invariants $I$ and $J$, when
can we construct a ``product'' quartic $g_3$ with $g_1*g_2=g_3$?  If
both $g_1$ and $g_2$ are soluble, then one could map each to a point on
the elliptic curve $\EIJ$, add the points and construct the quartic
$g_3$ from their sum.  However, it is of interest to express this
partial group law purely algebraically, without reference to elliptic
curves.  As we have seen, this can be done if certain conditions on
the solubility of the corresponding homogeneous spaces hold.  More
generally, we can always form the associated ternary quadratic form
$Q(u,v,w)$, as in the example above, and determine whether it has a
zero.  This is done in the next proposition, where $Q(u,v,w)$ is
diagonalised explicitly, enabling certain cases to be dealt with
simply.

\newprop\qprod
\proclaim{Proposition \qprod}
Let $I,J\in K$ satisfy $4I^3-J^2\not=0$ and let $g_i(X,Y)\in K[X,Y]$
for $i=1,2$ be quartics with invariants $I$ and $J$.  Suppose that the
leading coefficients $a_1$, $a_2$ are equal modulo squares:
$a_1a_2\in(K^*)^2$.  Then there exists a quartic $g_3(X,Y)$ with
invariants $I$ and $J$ such that $g_3=g_1*g_2$.
\endproclaim

\remark{Remark} Since we are free to replace $g_1$ or $g_2$ by
equivalent quartics, we can also form $g_1*g_2$ provided that there
exist $(x_1,y_1)$, $(x_2,y_2)\in K\times K \setminus (0,0)$ such that
$g_1(x_1,y_1)g_2(x_2,y_2) \in (K^*)^2$.  \endremark

\demo{Proof} Let $z_i=(4a_i\phi+p_i)/3$ be the cubic seminvariant of
$g_i$ for $i=1,2$, where $\phi^3=3I\phi-J$ as usual.  The coefficient
of $\phi^2$ in $z_1z_2(u+v\phi+w\phi^2)^2$ is a ternary quadratic form
$Q(u,v,w)$, and it suffices to find a non-trivial solution to
$Q(u,v,w)=0$.  We have $N(z_i)=r_i^2$ for $i=1,2$ where $r_i\in K$.

Set 
$$\align
\alpha&=16(a_1a_2p_1p_2+a_1^2p_2^2+a_2^2p_1^2-48Ia_1^2a_2^2), \\
\beta &=4(a_1p_1p_2^2+a_2p_1^2p_2+64Ja_1^2a_2^2),\\
\noalign{\text{and}}
\gamma&=p_1^2p_2^2 + 48Ia_1a_2p_1p_2 + 64J(a_1^2a_2p_2+a_1a_2^2p_1),
\endalign
$$
and introduce new variables $\tilde{u}$, $\tilde{v}$, $\tilde{w}$
where
$$
\align
\tilde{u}&=16a_1a_2u+4(a_1p_2+a_2p_1)v+(p_1p_2+48Ia_1a_2)w,\\
\tilde{v}&=\alpha v+\beta w,\\
\noalign{\text{and}}
\tilde{w}&=108r_1r_2w.
\endalign
$$
The seminvariant syzygy implies that 
$$
   \beta^2-\alpha\gamma = 16a_1a_2(27r_1^2)(27r_2^2) = (108r_1r_2)^2a_1a_2.
$$
Using computer algebra we then find that 
$$
   16\alpha a_1a_2 Q(u,v,w) = \alpha\tilde{u}^2-\tilde{v}^2+a_1a_2\tilde{w}^2.
$$
Hence there is a nontrivial solution when $a_1a_2$ is a square.
\qed \enddemo





%%%%%%%%%%%%%%%%%%%%%%%%%%%%%%%%%%%%%%%%%%%%%%%%%%%%%%%%%%%%%%%%%%
% 
%
\Refs %      REFERENCES 
%
%%%%%%%%%%%%%%%%%%%%%%%%%%%%%%%%%%%%%%%%%%%%%%%%%%%%%%%%%%%%%%%%%%

\ref \no \BSD \by B. J. Birch and H. P. F. Swinnerton-Dyer
\paper Notes on Elliptic Curves I
\jour J. Reine Angew. Math.
\vol 212 \yr 1963 \pages 7--25
\endref

%\ref \no \Cassels \by J. W. S. Cassels
%\paper Survey article: Diophantine equations with special reference to
%elliptic curves
%\jour J. London Math. Soc.
%\vol 41 \yr 1966 \pages 193--291
%\endref

\ref \no \JCbook \by J. E. Cremona 
\book Algorithms for Modular Elliptic Curves
\publ Cambridge University Press
\yr 1992
\endref

\ref \no \JCPS \by J. E. Cremona and P. Serf
\paper Computing the rank of elliptic curves over real quadratic
fields of class number 1
\paperinfo preprint
\yr 1996
\endref

\ref \no \Elliott \by E. B. Elliott
\book An Introduction to the Algebra of Quantics (Second Edition)
\publ Oxford University Press
\yr 1913
\endref

\ref \no \Hilbert \by D. Hilbert
\book Theory of Algebraic Invariants
\publ Cambridge University Press
\yr 1993
\endref

\ref \no \PSthesis \by P. Serf
\book The rank of elliptic curves over real quadratic number fields of
class number 1
\bookinfo Doctoral Thesis, Universit\"at des Saarlandes
\yr 1995
\endref

\endRefs
\enddocument
